\chapter*{ЗАКЛЮЧЕНИЕ}
\addcontentsline{toc}{chapter}{ЗАКЛЮЧЕНИЕ}

В ходе работы были решены следующие задачи:
\begin{enumerate}
	\item Проведён сравнительный анализ способов реализации драйверов устройств.
	\item Сделан выбор способа реализации драйвера символьного дисплея.
	\item Проведён сравнительный анализ существующих интерфейсов взаимодействия с символьным жидкокристаллическим дисплеем.
	\item Сделан выбор интерфейса взаимодействия с символьным жидкокристаллическим дисплеем в соответствии с поставленной целью.
	\item Сделан выбор системных вызовов, поддерживаемых драйвером, в соответствии с поставленной целью.
	\item Разработаны алгоритмы и структура ПО.
	\item Реализован драйвер символьного дисплея для вывода информации о конкретном процессе как загружаемый модуль ядра Linux.
	\item Реализована программа уровня пользователя для взаимодействия с символьным дисплеем.
\end{enumerate}

При выполнении курсовой работы было принято решение реализовать драйвер символьного дисплея для вывода информации о процессе как модуль ядра Linux и использовать вспомогательный интерфейс I2C. Для получения данных о процессе по PID использована виртуальная файловая система proc.

Таким образом, была достигнута цель работы:  реализован программный комплекс, состоящий из  загружаемого модуля ядра Linux и программы уровня пользователя для вывода на символьный дисплей заданной приложением информации о конкретном процессе. В ходе проведения исследования продемонстрирована корректность работы разработанного ПО.