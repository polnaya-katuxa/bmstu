\begin{appendices}
\label{appendix:graph}
	\chapter{Исходный код}
	
	В листинге \ref{driver} приведен код разработанного модуля ядра~---~драйвера символьного дисплея.
	
	\begin{lstlisting}[label=driver,caption=Код модуля ядра~---~драйвера символьного дисплея]
#include <linux/module.h>
#include <linux/init.h>
#include <linux/fs.h>
#include <asm/io.h>
#include <linux/unistd.h>
#include <linux/delay.h>
#include <asm/uaccess.h>
#include <linux/miscdevice.h>
#include <linux/fcntl.h>
#include <linux/ioctl.h>
#include <linux/i2c-dev.h>
#include <linux/types.h>
#include <linux/delay.h>
#include <linux/i2c.h>
#include <uapi/linux/errno.h>
#include <asm/errno.h>

MODULE_LICENSE("GPL");
MODULE_AUTHOR("Karpova Ekaterina");

#define LCD_ADDRESS     (0x3e)
#define RGB_ADDRESS     (0xc0>>1)
#define COMMAND_REG 0x80
#define DATA_REG 0x40

#define REG_RED     0x04
#define REG_GREEN       0x03
#define REG_BLUE        0x02
#define REG_MODE1       0x00
#define REG_MODE2       0x01
#define REG_OUTPUT      0x08
#define LCD_CLEARDISPLAY  0x01
#define LCD_ENTRYMODESET  0x04
#define LCD_DISPLAYCONTROL  0x08
#define LCD_FUNCTIONSET  0x20

//#flags for display entry mode
#define LCD_ENTRYLEFT  0x02
#define LCD_ENTRYSHIFTDECREMENT  0x00

//#flags for display on/off control
#define LCD_DISPLAYON  0x04
#define LCD_CURSOROFF  0x00

//#flags for function set
#define LCD_4BITMODE  0x00
#define LCD_2LINE  0x08

struct file* file_i2c;

static inline int delay_microseconds(int value) {
    if (value > 1000) {
        msleep(value/1000);
    }
    udelay(value % 1000);
    return 0;
}

int write_byte_data(int reg, int data, const int addr) {
    if (vfs_ioctl(file_i2c, I2C_SLAVE, addr) < 0)
    {
        printk(KERN_DEBUG "+ failed to acquire bus access and/or talk to slave");
        return 1;
    }

    union i2c_smbus_data msg;
    msg.byte = data;

    struct i2c_smbus_ioctl_data req = {
        .read_write = 0,
        .command = reg,
        .size = 2,
        .data = &msg,
    };

    struct i2c_client *client = file_i2c->private_data;
    int res = i2c_smbus_xfer(client->adapter, client->addr, client->flags,
                         req.read_write, req.command, req.size, &msg);
    if (res < 0) {
        printk(KERN_DEBUG "+ failed to smbus_xfer %d", res);
        return 1;
    }

    return 0;
}

int command(int cmd) {
    if ((write_byte_data(COMMAND_REG, cmd, LCD_ADDRESS)) != 0)
    {
        printk(KERN_DEBUG "+ failed to write byte data");
        return 1;
    }

    return 0;
}

int print_byte(int data) {
    if ((write_byte_data(DATA_REG, data, LCD_ADDRESS)) != 0)
    {
        printk(KERN_DEBUG "+ failed to write byte data");
        return 1;
    }

    return 0;
}

int set_reg(int reg, int data) {
    if ((write_byte_data(reg, data, RGB_ADDRESS)) != 0)
    {
        printk(KERN_DEBUG "+ failed to write byte data");
        return 1;
    }

    return 0;
}

void set_cursor(int col, int row) {
    if (row == 0) {
        col |= 0x80;
    } else {
        col |= 0xc0;
    }

    if ((command(col)) != 0) {
        printk(KERN_DEBUG "+ failed to set cursor");
    }
}

int configure(void) {
    printk(KERN_DEBUG "+ set show_function 1st try");

    if ((command(LCD_4BITMODE | LCD_2LINE | LCD_FUNCTIONSET)) != 0) {
        printk(KERN_DEBUG "+ failed to set show_function 1st try");
        return 1;
    }

    delay_microseconds(100000);

    printk(KERN_DEBUG "+ set show_function 2nd try");

    if ((command(LCD_4BITMODE | LCD_2LINE | LCD_FUNCTIONSET)) != 0) {
        printk(KERN_DEBUG "+ failed to set show_function 2nd try");
        return 1;
    }

    delay_microseconds(100000);

    printk(KERN_DEBUG "+ set show_function 3rd try");

    if ((command(LCD_4BITMODE | LCD_2LINE | LCD_FUNCTIONSET)) != 0) {
        printk(KERN_DEBUG "+ failed to set show_function 3rd try");
        return 1;
    }

    printk(KERN_DEBUG "+ set show_function 4th try");

    if ((command(LCD_4BITMODE | LCD_2LINE | LCD_FUNCTIONSET)) != 0) {
        printk(KERN_DEBUG "+ failed to set show_function 4th try");
        return 1;
    }

    delay_microseconds(100000);

    printk(KERN_DEBUG "+ turn the display on with no cursor or blinking default");

    if ((command(LCD_DISPLAYON | LCD_CURSOROFF | LCD_DISPLAYCONTROL)) != 0) {
        printk(KERN_DEBUG "+ failed to turn the display on with no cursor or blinking default");
        return 1;
    }

    printk("+ write to send command clear");

    if ((command(LCD_CLEARDISPLAY)) != 0) {
        printk(KERN_DEBUG "+ failed to write to send command clear");
        return 1;
    }

    delay_microseconds(100000);

    printk(KERN_DEBUG "+ set text direction and entry mode");

    if ((command(LCD_ENTRYLEFT | LCD_ENTRYSHIFTDECREMENT | LCD_ENTRYMODESET)) != 0) {
        printk(KERN_DEBUG "+ failed to set text direction and entry mode");
        return 1;
    }

    printk(KERN_DEBUG "+ setting colors");

    set_reg(REG_MODE1, 0);
    set_reg(REG_OUTPUT, 0xFF);
    set_reg(REG_MODE2, 0x20);

    set_reg(REG_RED, 255);
    set_reg(REG_GREEN, 255);
    set_reg(REG_BLUE, 255);

    printk(KERN_DEBUG "+ ending configuration");

    return 0;
}

int clear(void) {
    if ((command(LCD_CLEARDISPLAY)) != 0) {
        printk(KERN_DEBUG "+ failed to write to send command");
        return 1;
    }

    return 0;
}

int print_string(char* str) {
    for (int i = 0; str[i] != 0; i++) {
        if (i > 0 && i % 16 == 0) {
            set_cursor(0, 1);
        }
        if ((print_byte(str[i])) != 0) {
            printk(KERN_DEBUG "+ Failed to write data");
            return 1;
        }
    }

    return 0;
}

int start(void) {
    int length;
    unsigned char buffer[60] = {0};

    printk(KERN_DEBUG "+ opening");

    char *filename = (char*)"/dev/i2c-5";
    file_i2c = filp_open(filename, O_RDWR, 0);
    if (!file_i2c) {
        printk(KERN_DEBUG "+ failed to open the i2c bus");
        return 1;
    }

    printk(KERN_DEBUG "+ configuring");

    if ((configure()) != 0) {
        printk(KERN_DEBUG "+ failed to configure");
        return 1;
    }

    printk(KERN_DEBUG "+ setting cursor");

    set_cursor(0, 0);

    delay_microseconds(100000);

    return 0;
}

static char *info_str = "LCD display driver.";

static ssize_t dev_read( struct file * file, char * buf, size_t count, loff_t *ppos) {
    int len = strlen(info_str);

    if(count < len) {
        return -EINVAL;
    }

    if (*ppos != 0) {
        return 0;
    }

    if (copy_to_user(buf, info_str, len)) {
        return -EINVAL;
    }

    *ppos = len;
    return len;
}

static int str_pos = 0;
static int col_pos = 0;

static ssize_t dev_write(struct file *file, const char *buf, size_t count, loff_t *ppos) {
    str_pos = 0;
    col_pos = 0;

    for (size_t i = 0; i < count; i++) {
        if ((col_pos==0) && (str_pos==0)) {
            clear();
        }

        set_cursor(col_pos, str_pos);
        delay_microseconds(10000);

        printk(KERN_DEBUG "+ cursor col %d row %d cur sym %c", col_pos, str_pos, buf[i]);

        if (buf[i] != '\n') {
            print_byte(buf[i]);
            col_pos++;
            delay_microseconds(10000);
        } else {
            col_pos=16;
        }

        if (col_pos == 16) {
            col_pos = 0;
            str_pos++;
            if (str_pos == 2) {
                str_pos = 0;
            }
        }
    }

    return count;
}

static const struct file_operations my_fops = {
        .owner  = THIS_MODULE,
        .read   = dev_read,
        .write  = dev_write,
};

static struct miscdevice my_dev = {
        MISC_DYNAMIC_MINOR,
        "proclcd",
        &my_fops
};

static int __init my_init(void)
{
    int ret;

    rc = misc_register(&my_dev);
    if (rc) {
        printk(KERN_ERR "+ unable to register misc device\n");
    }

    start();

    printk(KERN_DEBUG "+ module loaded");

    return ret;
}

static void __exit my_exit(void)
{
    misc_deregister( &my_dev );
    printk(KERN_DEBUG "+ module unloaded");
}

module_init(my_init);
module_exit(my_exit);

	\end{lstlisting}
	
	В листинге \ref{client} приведен код разработанного приложения уровня пользователя.
	
	\begin{lstlisting}[label=client,caption=Код разработанного приложения уровня пользователя]
#include <fcntl.h>
#include <unistd.h>
#include <string.h>
#include <dirent.h>
#include <errno.h>
#include <stdlib.h>
#include <stdio.h>
#include <stdint.h>
#include <signal.h>
#include "client.h"
#include <sys/stat.h>

#define DEV_FILE "/dev/proclcd"
#define _POSIX1_SOURCE 2
#define BUF_SIZE 102400
#define ANSI_COLOR_RED     "\x1b[31m"
#define ANSI_COLOR_GREEN   "\x1b[32m"
#define ANSI_COLOR_YELLOW  "\x1b[33m"
#define ANSI_COLOR_BLUE    "\x1b[34m"
#define ANSI_COLOR_MAGENTA "\x1b[35m"
#define ANSI_COLOR_CYAN    "\x1b[36m"
#define ANSI_COLOR_RESET   "\x1b[0m"

int pid = 1;
int opt = 0;
int stop = 0;

int existDir(const char * name)
{
    struct stat s;
    if (stat(name,&s)) return 0;
    return S_ISDIR(s.st_mode);
};

int print_menu() {
    int rc = EXIT_SUCCESS;

    printf("\n\n===============================\n");
    printf("\n" ANSI_COLOR_MAGENTA "PROCESS INFO ASSISTANT" ANSI_COLOR_RESET "\n");

    printf(ANSI_COLOR_BLUE "Choose info to view:" ANSI_COLOR_RESET "\n");
    printf(ANSI_COLOR_BLUE "0) " ANSI_COLOR_RESET "exit\n");
    printf(ANSI_COLOR_BLUE "1) " ANSI_COLOR_RESET "cmdline\n");
    printf(ANSI_COLOR_BLUE "2) " ANSI_COLOR_RESET "opened fd number\n");
    printf(ANSI_COLOR_BLUE "3) " ANSI_COLOR_RESET "thread number\n");
    printf(ANSI_COLOR_BLUE "4) " ANSI_COLOR_RESET "virtual memory size\n");
    printf(ANSI_COLOR_BLUE "5) " ANSI_COLOR_RESET "state\n");
    printf(ANSI_COLOR_BLUE "6) " ANSI_COLOR_RESET "executable filename\n");

    printf(ANSI_COLOR_BLUE "Input menu option: " ANSI_COLOR_RESET);
    if ((rc = scanf("%d", &opt)) != 1 || opt < 0 || opt > 6) {
        printf(ANSI_COLOR_RED "invalid menu option" ANSI_COLOR_RESET "\n");
        printf("\n===============================\n");
        return EXIT_FAILURE;
    }

    if (opt != 0) {
        printf(ANSI_COLOR_BLUE "Input process PID: " ANSI_COLOR_RESET);
        if ((rc = scanf("%d", &pid)) != 1) {
            printf(ANSI_COLOR_RED "invalid process pid" ANSI_COLOR_RESET "\n");
            printf("\n===============================\n");
            return EXIT_FAILURE;
        }

        char str[1024];
        sprintf(str, "/proc/%d", pid);

        if (existDir(str) == 0) {
            printf(ANSI_COLOR_RED "process doesn't exist" ANSI_COLOR_RESET "\n");
            printf("\n===============================\n");
            return EXIT_FAILURE;
        }
    }

    printf("\n===============================\n");
    return EXIT_SUCCESS;
}

void exit_handler() {
    opt = -1;
    printf("\n" ANSI_COLOR_YELLOW "program exiting" ANSI_COLOR_RESET "\n");
}

void cmdline_handler() {
    char buf[BUF_SIZE+1] = "\0";
    int len, i;
    FILE* f;

    int dev = open(DEV_FILE, O_RDWR);

    char path[PATH_MAX];
    snprintf(path, PATH_MAX, "/proc/%d/cmdline", pid);

    f = fopen(path, "r");

    while (stop == 0) {
        while ((len = fread(buf, 1, BUF_SIZE, f)) > 0)
        {
            buf[len] = '\0';
            write(dev, buf, len-1);
        }

        fseek(f, 0, SEEK_SET);
        sleep(3);
    }

    stop = 0;

    fclose(f);
    close(dev);
}

void fds_handler() {
    char path[PATH_MAX];
    snprintf(path, PATH_MAX, "/proc/%d/fd", pid);

    int dev = open(DEV_FILE, O_RDWR);

    struct dirent *d;
    DIR *dh = opendir(path);
    if (!dh)
    {
        perror("open task dir\n");
        exit(1);
    }

    long pos = telldir(dh);

    while (stop == 0) {
        int fd_num = 0;
        while ((d = readdir(dh)) != NULL)
        {
            if (d->d_name[0] == '.')
                continue;

            fd_num++;
        }

        char str[1024];
        sprintf(str, "%d", fd_num);
        write(dev, str, strlen(str));

        seekdir(dh,pos);

        sleep(3);
    }

    stop = 0;

    closedir(dh);
    close(dev);
}

void threads_handler() {
    char path[PATH_MAX];
    snprintf(path, PATH_MAX, "/proc/%d/task", pid);

    int dev = open(DEV_FILE, O_RDWR);

    struct dirent *d;
    DIR *dh = opendir(path);
    if (!dh)
    {
        perror("open task dir\n");
        exit(1);
    }

    long pos = telldir(dh);

    while (stop == 0) {
        int thread_num = 0;
        while ((d = readdir(dh)) != NULL)
        {
            if (d->d_name[0] == '.')
                continue;

            thread_num++;
        }

        char str[1024];
        sprintf(str, "%d", thread_num);
        write(dev, str, strlen(str));

        seekdir(dh,pos);

        sleep(3);
    }

    stop = 0;

    closedir(dh);
    close(dev);
}

void vm_handler() {
    char path[PATH_MAX];
    snprintf(path, PATH_MAX, "/proc/%d/statm", pid);

    int dev = open(DEV_FILE, O_RDWR);

    FILE *statm = fopen(path, "r");

    char buf[BUF_SIZE + 1] = "\0";
    int len, n;

    while (stop == 0) {
        fscanf(statm, "%d", &n);

        char str[1024];
        sprintf(str, "%d MB", n * sysconf(_SC_PAGE_SIZE) / 1024 / 1024);
        write(dev, str, strlen(str));

        fseek(statm, 0, SEEK_SET);
        sleep(3);
    }

    stop = 0;

    fclose(statm);
    close(dev);
}

void state_handler() {
    char path[PATH_MAX];
    snprintf(path, PATH_MAX, "/proc/%d/stat", pid);

    int dev = open(DEV_FILE, O_RDWR);

    FILE *stat = fopen(path, "r");

    char buf[BUF_SIZE + 1] = "\0";
    int len, n;
    char name[1024];
    char state;

    while (stop == 0) {
        fscanf(stat, "%d %s %c", &n, name, &state);

        char str[1024];
        if (state == 'R') {
            sprintf(str, "%c - runnable", state);
        } else if (state == 'D') {
            sprintf(str, "%c - uninterruptable", state);
        } else if (state == 'T') {
            sprintf(str, "%c - stopped", state);
        } else if (state == 'S') {
            sprintf(str, "%c - sleeping", state);
        } else if (state == 'Z') {
            sprintf(str, "%c - zombie", state);
        } else if (state == '<') {
            sprintf(str, "%c - negative nice", state);
        } else if (state == 'N') {
            sprintf(str, "%c - positive nice", state);
        }

        write(dev, str, strlen(str));

        fseek(stat, 0, SEEK_SET);
        sleep(3);
    }

    stop = 0;

    fclose(stat);
    close(dev);
}

void comm_handler() {
    char buf[BUF_SIZE+1] = "\0";
    int len, i;
    FILE* f;

    int dev = open(DEV_FILE, O_RDWR);

    char path[PATH_MAX];
    snprintf(path, PATH_MAX, "/proc/%d/comm", pid);

    f = fopen(path, "r");

    while (stop == 0) {
        while ((len = fread(buf, 1, BUF_SIZE, f)) > 0)
        {
            buf[len] = '\0';
            write(dev, buf, len-1);
        }

        fseek(f, 0, SEEK_SET);
        sleep(3);
    }

    stop = 0;

    fclose(f);
    close(dev);
}

void signal_handler(int signal)
{
    printf("\n" ANSI_COLOR_YELLOW "caught signal = %d" ANSI_COLOR_RESET "\n", signal);
    stop = 1;
}

int main(int argc, char **argv) {
    if ((signal(SIGINT, signal_handler) == SIG_ERR)) {
        perror("Can't attach handler\n");
        return EXIT_FAILURE;
    }

    if (argc != 1) {
        printf(ANSI_COLOR_RED "no arguments needed" ANSI_COLOR_RESET "\n");
        return EXIT_FAILURE;
    }

    void (*handlers[7])(void);
    handlers[0] = &exit_handler;
    handlers[1] = &cmdline_handler;
    handlers[2] = &fds_handler;
    handlers[3] = &threads_handler;
    handlers[4] = &vm_handler;
    handlers[5] = &state_handler;
    handlers[6] = &comm_handler;

    int rc;
    while (opt >= 0) {
        rc = print_menu();
        if (rc != EXIT_SUCCESS) {
            printf(ANSI_COLOR_RED "menu error" ANSI_COLOR_RESET "\n");
            opt = 0;
            pid = 1;
            stop = 0;
        } else {
            handlers[opt]();
        }
    }

    return EXIT_SUCCESS;
}

	\end{lstlisting}
	
\end{appendices}