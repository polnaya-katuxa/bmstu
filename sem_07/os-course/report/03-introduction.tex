\chapter*{ВВЕДЕНИЕ}
\addcontentsline{toc}{chapter}{ВВЕДЕНИЕ}

Драйверы устройств играют особую роль в ядре Linux \cite{bib1}. Каждый драйвер реализует определённый программный интерфейс взаимодействия с внешним устройством, скрывая от пользователя подробности работы устройства. Сопоставление вызовов, представленных в интерфейсе, с операциями, специфичными для устройства~---~основная задача драйвера устройства в Linux-системах. Реализация драйвера как модуля ядра Linux позволяет расширять функциональность ядра по работе с внешними устройствами во время работы системы. Часто вместе с драйвером предоставляется клиентская библиотека, реализующая возможности, не являющиеся частью интерфейса самого драйвера.

Некоторые драйверы также работают с дополнительными наборами функций ядра для данного типа устройств~\cite{bib1}. Например, с интерфейсами USB, I2C, UART, SPI и другими. Наиболее распространёнными интерфейсами для работы с символьными дисплеями являются I2C, SPI и UART.