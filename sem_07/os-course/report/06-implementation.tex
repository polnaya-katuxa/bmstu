\chapter{Технологический раздел}

\section{Выбор языка и среды программирования}

Для реализации загружаемого модуля был выбран язык Си, так как необходимо написать загружаемый модуль ядра, и ядро Linux реализовано на языке C. Операционная система Linux позволяет писать загружаемые модули ядра на Rust и на Cи, но Rust обладает определёнными ограничениями в написании модулей ядра.

В качестве среды разработки выбрана CLion: запуск программного комплекса осуществлялся на компьютере, к которому было выполнено подключение по ssh, и данная среда поддерживает передачу данных по ssh из графического интерфейса.

\section{Реализация загружаемого модуля ядра~---~драйвера символьного дисплея}

В листинге \ref{lst:mod} представлены реализации функции инициализации и завершения работы загружаемого модуля ядра~---~драйвера символьного дисплея.

\begin{lstlisting}[label=lst:mod,caption=Реализация загружаемого модуля ядра~---~драйвера символьного дисплея]
static const struct file_operations my_fops = {
        .owner  = THIS_MODULE,
        .read   = dev_read,
        .write  = dev_write,
};

static struct miscdevice my_dev = {
        MISC_DYNAMIC_MINOR,
        "proclcd",
        &my_fops
};

static int __init my_init(void)
{
    int ret;

    rc = misc_register(&my_dev);
    if (rc) {
        printk(KERN_ERR "+ unable to register misc device\n");
    }

    start();

    printk(KERN_DEBUG "+ module loaded");

    return ret;
}

static void __exit my_exit(void)
{
    misc_deregister( &my_dev );
    printk(KERN_DEBUG "+ module unloaded");
}

module_init(my_init);
module_exit(my_exit);
\end{lstlisting}

В листинге \ref{lst:conf} представлена реализация функции установления связи с шиной I2C и настройки соединения с дисплеем.

\begin{lstlisting}[label=lst:conf,caption=Реализация функции установления связи с шиной I2C и настройки соединения с дисплеем]
int start(void) {
    int length;
    unsigned char buffer[60] = {0};

    printk(KERN_DEBUG "+ opening");

    char *filename = (char*)"/dev/i2c-5";
    file_i2c = filp_open(filename, O_RDWR, 0);
    if (!file_i2c) {
        printk(KERN_DEBUG "+ failed to open the i2c bus");
        return 1;
    }

    printk(KERN_DEBUG "+ configuring");

    if ((configure()) != 0) {
        printk(KERN_DEBUG "+ failed to configure");
        return 1;
    }

    printk(KERN_DEBUG "+ setting cursor");

    set_cursor(0, 0);

    delay_microseconds(100000);

    return 0;
}
\end{lstlisting}

В листинге \ref{lst:write_bytes} представлена реализация функции вывода данных на дисплей.

\begin{lstlisting}[label=lst:write_bytes,caption=Реализация функции вывода данных на дисплей]
static ssize_t dev_write(struct file *file, const char *buf, size_t count, loff_t *ppos) {
    char data[1024];
    if (copy_from_user(data, buf, strlen(buf))) {
        return -EINVAL;
    }

    str_pos = 0;
    col_pos = 0;

    for (int i = 0; i < count; i++) {
        if ((col_pos==0) && (str_pos==0)) {
            clear();
        }

        set_cursor(col_pos, str_pos);
        delay_microseconds(10000);

        printk(KERN_DEBUG "+ cursor col %d row %d cur sym %c", col_pos, str_pos, data[i]);

        if (buf[i] != '\n') {
            print_byte(data[i]);
            col_pos++;
            delay_microseconds(10000);
        } else {
            col_pos=16;
        }

        if (col_pos == 16) {
            col_pos = 0;
            str_pos++;
            if (str_pos == 2) {
                str_pos = 0;
            }
        }
    }

    return count;
}
\end{lstlisting}

В листинге \ref{lst:write_byte} представлена реализация функции передачи байта данных с помощью интерфейса I2C.

\begin{lstlisting}[label=lst:write_byte,caption=Реализация функции передачи байта данных с помощью интерфейса I2C]
int write_byte_data(int reg, int data, const int addr) {
    if (vfs_ioctl(file_i2c, I2C_SLAVE, addr) < 0)
    {
        printk(KERN_DEBUG "+ failed to acquire bus access and/or talk to slave");
        return 1;
    }

    union i2c_smbus_data msg;
    msg.byte = data;

    struct i2c_smbus_ioctl_data req = {
        .read_write = 0,
        .command = reg,
        .size = 2,
        .data = &msg,
    };

    struct i2c_client *client = file_i2c->private_data;
    int res = i2c_smbus_xfer(client->adapter, client->addr, client->flags,
                         req.read_write, req.command, req.size, &msg);
    if (res < 0) {
        printk(KERN_DEBUG "+ failed to smbus_xfer %d", res);
        return 1;
    }

    return 0;
}
\end{lstlisting}

В листинге \ref{lst:make} представлен Makefile загружаемого модуля ядра.

\begin{lstlisting}[label=lst:make,caption=Makefile загружаемого модуля ядра]
ifneq ($(KERNELRELEASE),)
	obj-m := my_driver.o
else
	CURRENT = $(shell uname -r)
	KDIR = /lib/modules/$(CURRENT)/build
	PWD = $(shell pwd)

default:
	$(MAKE) -C $(KDIR) M=$(PWD) modules

clean:
	rm -rf .tmp_versions
	rm *.ko
	rm *.o
	rm *.mod.c
	rm *.symvers
	rm *.order

endif
\end{lstlisting}

\section{Реализация клиентского приложения}

В листинге \ref{lst:proc} представлена реализация функции-обработчика запроса пользователя для вывода из proc данных о потребляемой процессом виртуальной памяти.

\begin{lstlisting}[label=lst:proc,caption=Реализация функции-обработчика запроса пользователя для вывода из proc данных о потребляемой процессом виртуальной памяти]
void vm_handler() {
    char path[PATH_MAX];
    snprintf(path, PATH_MAX, "/proc/%d/statm", pid);

    int dev = open(DEV_FILE, O_RDWR);

    FILE *statm = fopen(path, "r");

    char buf[BUF_SIZE + 1] = "\0";
    int len, n;

    while (stop == 0) {
        fscanf(statm, "%d", &n);

        char str[1024];
        sprintf(str, "%d MB", n * sysconf(_SC_PAGE_SIZE) / 1024 / 1024);
        write(dev, str, strlen(str));

        fseek(statm, 0, SEEK_SET);
        sleep(3);
    }

    stop = 0;

    fclose(statm);
    close(dev);
}
\end{lstlisting}

\section*{Вывод}
Был обоснован выбор языка и среды программирования и приведены листинги кода. Представлены листинги функций инициализации и завершения работы драйвера дисплея, функции установления связи с шиной I2C и настройки соединения с дисплеем. Также приведены листинги общей функции вывода данных на дисплей и передачи байта данных с помощью интерфейса I2C.






