\chapter*{ВВЕДЕНИЕ}
\addcontentsline{toc}{chapter}{ВВЕДЕНИЕ}

Криптография~---~наука о методах обеспечения целостности данных, аутентификации, шифрования. По мере образования информационного общества, криптография становится одним из основных инструментов, обеспечивающих конфиденциальность, доверие, авторизацию, электронные платежи, корпоративную безопасность и бесчисленное множество других важных вещей \cite{bib1}. <<Энигма>> является криптографической машиной, которая была создана в 1920-х для военных нужд \cite{bib2}.

Целью данной лабораторной работы является программная реализация аналога шифровальной машины <<Энигма>> на языке Си.

Для достижения поставленной цели необходимо выполнить следующие задачи.

\begin{enumerate}[label=\arabic*)]
	\item изучить алгоритм работы шифровальной машины <<Энигма>>;
	\item спроектировать алгоритм работы шифровальной машины <<Энигма>>;
	\item реализовать алгоритм работы шифровальной машины <<Энигма>> на языке Си;
	\item протестировать реализацию алгоритма работы шифровальной машины  <<Энигма>>.
\end{enumerate}