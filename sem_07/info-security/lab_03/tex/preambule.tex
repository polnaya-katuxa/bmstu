\documentclass[a4paper,14pt]{extarticle}

\usepackage{cmap} % Улучшенный поиск русских слов в полученном pdf-файле
\usepackage[T2A]{fontenc} % Поддержка русских букв
\usepackage[utf8]{inputenc} % Кодировка utf8
\usepackage[english,russian]{babel} % Языки: русский, английский

\usepackage[14pt]{extsizes} % Задание 14-размера шрифта
\usepackage[left=3cm,right=1cm,top=2cm,bottom=2cm]{geometry} % Задание геометрии листа

\usepackage[unicode,pdftex]{hyperref} % Ссылки в pdf
\hypersetup{hidelinks}

\usepackage{setspace}
\onehalfspacing % Полуторный интервал

\frenchspacing
\usepackage{indentfirst} % Красная строка

% Настройка нумерации объектов
\counterwithin{figure}{section}
\counterwithin{table}{section}
%\numberwithin{equation}{section}


\usepackage{enumitem} % Настройка оформления списков
\setlist{nosep} 
\setlist[enumerate,1]{label={\arabic*)}}

\renewcommand{\labelitemi}{---}
\renewcommand{\labelitemii}{---}

\usepackage{titlesec} % Оформление заголовков
\titleformat{\section}[block]
{\bfseries\large}
{\thesection}
{1em}
{}

\titleformat{name=\section,numberless}[block]
{\bfseries\large\centering}
{}
{1em}
{}

\titleformat{\subsection}[hang]
{\bfseries\normalsize}
{\thesubsection}
{1em}{}

\titleformat{\subsubsection}[hang]
{\bfseries\normalsize}
{\thesubsubsection}
{1em}{}

% Математические пакеты
\usepackage{amsmath} 
\usepackage{amssymb}

\usepackage{caption} % Подпись картинок и таблиц
\captionsetup{labelsep=endash} % Разделитель между номером и текстом краткое тире и пробел
\captionsetup[figure]{name={Рисунок}} % Изменяет имя для всех фигур на "Рисунок"
%\captionsetup[figure]{justification=centering}
%\usepackage[justification=centering]{caption} % Настройка подписей float объектов
\captionsetup[table]{justification=raggedright, singlelinecheck=false}
\captionsetup[lstlisting]{justification=raggedright, singlelinecheck=false}

% дополнительно к таблицам
\usepackage{makecell} % удобный перенос строки в таблице
\usepackage{longtable} % многостраничные таблицы
\usepackage{multirow} % объединение строк и столбцов


% Вставка рисунков
\usepackage{graphicx} 

\newcommand{\img}[3] {
	\begin{figure}[H]
		\center{\includegraphics[height=#1]{img/#2}}
		 \ifthenelse{ \equal{#3}{} }
		{}
		{\caption{#3}}
		\label{img:#2}
	\end{figure}
}

\usepackage{csvsimple} % генерация и фильтрация таблиц из csv файлов
\usepackage{float} % настройка флоат-объектов

%\usepackage[shortcuts]{extdash}


% Оформление листингов
\usepackage{listings}
\usepackage{xcolor} % Добавление цветов 

\lstdefinestyle{mystyle}{ % Опеределение стиля 
%	language=C++,
	backgroundcolor=\color{white},
	basicstyle=\footnotesize\ttfamily,
	keywordstyle=\color{blue},
	stringstyle=\color{red},
	commentstyle=\color{gray},
	numbers=left,
	numberstyle=\tiny,
	stepnumber=1,
	numbersep=5pt,
	frame=single,
	tabsize=4,
	captionpos=t,
	breaklines=true,
	breakatwhitespace=true,
	xleftmargin=10pt
}
\lstset{extendedchars=true, texcl=true}
\lstset{style=mystyle}

% Настройка списка литературы
\addto\captionsrussian{\renewcommand{\refname}{СПИСОК ИСПОЛЬЗОВАННЫХ ИСТОЧНИКОВ}}

\makeatletter
\def\@biblabel#1{#1. }
\makeatother

% вставка многостраничных pdf-документов
\usepackage{pdfpages}

% настройка приложений
\usepackage[titletoc, title]{appendix} %добавление приложений в оглавление