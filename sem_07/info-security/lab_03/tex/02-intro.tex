%\pagenumbering{arabic}
\section*{ВВЕДЕНИЕ}
\phantomsection
\addcontentsline{toc}{section}{ВВЕДЕНИЕ}

AES — симметричный алгоритм блочного шифрования, принятый в качестве стандарта шифрования правительством США по результатам конкурса AES. Этот алгоритм хорошо проанализирован и сейчас широко используется, как это было с его предшественником DES. Национальный институт стандартов и технологий США опубликовал спецификацию AES 26 ноября 2001 года после пятилетнего периода, в ходе которого были созданы и оценены 15 кандидатур. 26 мая 2002 года AES был объявлен стандартом шифрования. По состоянию на 2009 год AES является одним из самых распространённых алгоритмов симметричного шифрования.

\textbf{Целью} данной лабораторной работы является разработка программного обеспечения, позволяющего шифровать и дешифровать произвольный файл по алгоритму симметричного шифрования AES, а именно OFB — режим обратной связи по выходу.
Для достижения цели необходимо решить следующие задачи:
\begin{itemize}
	\item изучить алгоритм симметричного шифрования AES;
	\item реализовать алгоритм симметричного шифрования AES;
	\item реализовать алгоритм режим OFB.
\end{itemize}


\newpage