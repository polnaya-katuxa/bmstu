\chapter{Теоретический раздел}

Обычно под термином <<случайные числа>> подразумевают последовательность независимых случайных величин с заданным распределением.

Равномерным распределением на конечном множестве чисел называется такое распределение, при котором любое из возможных чисел имеет одинаковую вероятность появления. Если не задано определенное распределение на конечном множестве чисел, то принято считать его равномерным.

\begin{equation}
	f(x) = \begin{cases}
        1, x \in (a; b), \\
        0, \text{иначе}.
    \end{cases}
\end{equation}

Существует несколько категорий способов генерации случайных чисел. Одними из них являются табличный и алгоритмический способы.

Табличный способ заключается в использовании специальным образом составленных таблиц. Генерация чисел производится следующим способом: вычисляется начальная позиция в файле, после чего происходит чтение нужного количества цифр числа из файла. При этом переход на следующую позицию возможен множеством способов: переходом на следующую позицию, на следующую строку в столбце, на ближайшую четную позицию.

Алгоритмический способ заключается в генерации последовательности чисел детерминированным образом, в котором каждый следующий элемент последовательности зависит от предыдущего. Таким образом, фактически последовательность не является случайной, но выглядит таковой для пользователя. Числа, полученные алгоритмическим способом, называются псевдослучайными.

Для проверки того, насколько последовательность можно считать случайной, существуют специальные критерии. В данной работе реализованы критерий Колмогорова-Смирнова и сериальной корреляции.

Критерий Колмогорова-Смирнова заключается в расчете разности теоретической функции распределения случайной величины и эмпирической функции распределения, полученной опытным путем. Для оценки используется следующий параметр:

\begin{equation}
	K_n = \sqrt{n} \cdot \max{|F(x) - F_n(x)|}.
\end{equation}

Кроме того, был использован критерий сериальной корреляции. Расчитывается коэффициент корреляции между соседними группами измерений. Больший коэффициент корреляции означает зависимость, близкую к линейной.