\chapter{Теоретический раздел}

Случайный процесс называют марковским, если во всякий момент времени
вероятность нахождения системы в некотором состоянии после него не зависит от состояния системы до него.

Предельной вероятностью нахождения системы в \( i \)-ом состоянии называют
число
\[
    p_i = \lim_{t \rightarrow +\infty} p_i(t).
\]

Предельная вероятность \( p_i \) нахождения системы в \( i \)-ом состоянии
может быть найдена путём решения системы линейных алгебраических уравнений
\[
    \begin{cases}
        \sum_{j = 1, j \ne i}^n p_j \lambda_{ji}
            = p_i \sum_{j = 1, j \ne i}^n \lambda_{ij}; \\
        \sum_{k = 1}^n p_k = 1.
    \end{cases}
\]