\section{Ход выполнения работы}

Все задания практикума выполнялись по варианту 11.

Выполнить визуализацию неориентированного графа, представленного в формате tsv. Каждая строчка файла представляет собой описание ребра, сотоящее из трех чисел (Вершина,Вершина,Вес).


В листинге~\ref{lst1} представлен код программы по индивидуальному варианту из файла $host\_main.cpp$.

\begin{lstlisting}[caption={Код программы по индивидуальному варианту host\_main.cpp}, label=lst1, style=Go]
#ifdef MY_GRAPH

    __foreach_core(group, core)
	{
		lnh_inst.gpc[group][core]->start_async(__event__(delete_graph));
	}


	unsigned int* host2gpc_ext_buffer[LNH_GROUPS_COUNT][LNH_MAX_CORES_IN_GROUP];
	unsigned int messages_count = 0;
	unsigned int u, v, w;

	__foreach_core(group, core)
	{
		host2gpc_ext_buffer[group][core] = (unsigned int*)lnh_inst.gpc[group][core]->external_memory_create_buffer(16 * 1048576 * sizeof(int));
		offs = 0;

		std::ifstream file(argv[3], std::ios::in);

		if (!file.is_open())
		{
			std::cout << "Error opening file." << std::endl;
			
			return EXIT_FAILURE;
		}

		for (std::string line; std::getline(file, line); ) 
		{
			std::vector<std::string> tokens = split(line, '\t');

			if (tokens.size() != 2)
			{
				std::cout << "Incorrect tokens count: expected 2, got " << tokens.size() << "." << std::endl;

				return EXIT_FAILURE;
			}

			u = std::stoul(tokens[0]);
			v = std::stoul(tokens[1]);
			w = 1;
			
			EDGE(u, v, w);
			EDGE(v, u, w);
			messages_count += 2;
		}

		lnh_inst.gpc[group][core]->external_memory_sync_to_device(0, 3 * sizeof(unsigned int)*messages_count);
	}
	__foreach_core(group, core)
	{
		lnh_inst.gpc[group][core]->start_async(__event__(insert_edges));
	}
	__foreach_core(group, core) {
		long long tmp = lnh_inst.gpc[group][core]->external_memory_address();
		lnh_inst.gpc[group][core]->mq_send((unsigned int)tmp);
	}
	__foreach_core(group, core) {
		lnh_inst.gpc[group][core]->mq_send(3 * sizeof(int)*messages_count);
	}


	__foreach_core(group, core)
	{
		lnh_inst.gpc[group][core]->finish();
	}
	printf("Data graph created!\n");

#endif
\end{lstlisting}
\par В рисунке 2.1 представлена команда для сборки проекта.
\img{0.05\textwidth}{2.png}{команда для сборки проекта}
\par В рисунке 2.2 представлен результат запуска.
\img{0.5\textwidth}{3.png}{результат запуска}
\par В рисунке 2.3 представлена команда для запуска сервера bokeh.
\img{0.15\textwidth}{4.png}{команда для запуска сервера bokeh}
\par В рисунке 2.4 результат визуализации.
\img{0.5\textwidth}{plot.jpeg}{результат визуализации}
\newpage