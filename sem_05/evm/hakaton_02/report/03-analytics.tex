\section{Аналитическая часть}

\subsection{Конвейер визуализации графов}
Визуализация графа — это графическое представление вершин и ребер графа. Визуализация строится на основе исходного графа, но направлена на получение дополнительных атрибутов вершин и ребер: размера, цвета, координат вершин, толщины и геометрии ребер. Помимо этого, в задачи визуализации входит определение масштаба представления визуализации. Для различных по своей природе графов, могут быть более применимы различные варианты визуализации. Таким образом задачи, входящие в последовательность подготовки графа к визуализации, формулируются исходя из эстетических и эвристических критериев. Графы можно визуализировать, используя:
\begin{itemize}
	\item 2D графическую сцену - наиболее часто применяемый случай, обладающий приемлемой вычислительной сложностью;
	\item 3D графическую сцену - такой вариант позволяет выполнять перемещение камеры наблюдения, что увеличивает возможное количество визуализируемых вершин;
	\item Иерархическое представление - граф представляется в виде иерархически вложенных подграфов (уровней), что позволяет более наглядно представить тесно связанные компоненты первоначального графа.
\end{itemize}
\par Для визуализации графов было определено несколько показателей качества, позволяющие получить количественные оценки эстетики и удобства графического представления. Алгоритмы раскладки, в большинстве случаев, нацелены на оптимизацию следующих показателей:
\begin{itemize}
	\item Меньшее количество пересечений ребер: выравнивание вершин и ребер для получения наименьшего количества пересечений ребер делает визуализацию более понятной и менее запутанной.
	\item Минимум наложений вершин и рёбер.
	\item Распределение вершин и/или рёбер равномерно.
	\item Более тесное расположение смежных вершин.
	\item Формирование сообществ вершин из сильно связанных групп вершин.
\end{itemize}
\par Для больших графов метод выделения сообществ является предпочтительным, так как позволяет выявить скрытые кластеры вершин, показать центральную вершину сообщества, минимизировать количество внешних связей между сообществами.
Таким образом визуализация графов представляет собой многостадийный алгоритмический процесс. Кратко стадии процесса визуализации представлены на следующем рисунке.
\img{0.15\textwidth}{1.png}{Схема визуализации графов}
\subsection{Использование библиотеки шаблонов leonhard-x64-xrt-v2 для обработки графов}
\par Для реализация алгоритмов обработки графов необходимо представить операции над множествами (в том числе, множествами вершин и ребер графа) в виде набора команд дискретной математики DISC. Все команды обработки структур данных изменяют регистр статуса, по которому можно определить, было ли выполнение команды успешным (Регистр LNH\_STATE, бит SPU\_ERROR\_FLAG). Результаты, влияющие на работу программы, должны быть учтены в общем алгоритме. После завершения основания команд, основанных на поиске (SEARCH, DELETE, MAX, MIN, NEXT, PREV, NSM, NGR) в очередь данных попадают ключ и значение найденных записей (KEY, VALUE), которые могут быть использованы в алгоритме программного ядра CPE riscv32. Для команд И-ИЛИ-НЕ (пересечение,объединение,дополнение) передаются операнды номеров структур (R,A,B). Операнд R указывает на номер структуры, в которой будет сохранен результат. Структуры A и B используются в И-ИЛИ-НЕ операциях и срезах в качестве исходных.
\newpage