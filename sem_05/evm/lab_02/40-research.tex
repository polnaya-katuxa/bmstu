\chapter*{Программа по варианту}

\subsection*{Исследуемая программа}

Код программы представлен в листинте \ref{lst:v11}

\begin{code}
\caption{Код программы 6 варианта}
\label{lst:v11}
\begin{minted}{text}
        .section .text
        .globl _start;
        len = 8 #Размер массива
        enroll = 2 #Количество обрабатываемых элементов
        за одну итерацию
	elem_sz = 4 #Размер одного элемента массива

_start:
        addi x20, x0, len/enroll
        la x1, _x
	add x31, x0, x0
lp:
        lw x2, 0(x1)
        lw x3, 4(x1) # !
        addi x1, x1, elem_sz*enroll
        addi x20, x20, -1
        add x31, x31, x2
        add x31, x31, x3
        bne x20, x0, lp
        addi x31, x31, 1
lp2: j lp2

        .section .data
_x:     .4byte 0x1
        .4byte 0x2
        .4byte 0x3
        .4byte 0x4
\end{minted}
\end{code}

\begin{code}
\caption{Код программы 6 варианта (продолжение)}
\label{lst:v12}
\begin{minted}{text}
        .4byte 0x5
        .4byte 0x6
        .4byte 0x7
        .4byte 0x8
\end{minted}
\end{code}


Дизассемблерный код представлен на листинге \ref{lst:v22}.

\begin{code}
\caption{Дизассемблированный код 6 варианта }
\label{lst:v22}

\begin{minted}{text}
Disassembly of section .text:
80000000 <_start>:
80000000:    00400a13    addi     x20,x0,4
80000004:    00000097    auipc    x1,0x0
80000008:    03008093    addi     x1,x1,48 # 80000034 <_x>
8000000c:    00000fb3    add      x31,x0,x0
80000010 <lp>:

80000010:    0000a103    lw      x2,0(x1)
80000014:    0040a183    lw      x3,4(x1)
80000018:    00808093    addi    x1,x1,8
8000001c:    fffa0a13    addi    x20,x20,-1
80000020:    002f8fb3    add     x31,x31,x2
80000024:    003f8fb3    add     x31,x31,x3
80000028:    fe0a14e3    bne     x20,x0,80000010 <lp>
8000002c:    001f8f93    addi    x31,x31,1
80000030 <lp2>:

80000030:    0000006f    jal     x0,80000030 <lp2>
\end{minted}
\end{code}

\newpage

\begin{code}
\caption{Дизассемблированный код 6 варианта (продолжение)}
\label{lst:v23}

\begin{minted}{text}
Disassembly of section .data:
80000034 <_x>:
80000034:    0001    c.addi  x0,0
80000036:    0000    c.unimp
80000038:    0002    c.slli64        x0
8000003a:    0000    c.unimp
8000003c:    00000003	lb    x0,0(x0) # 0 <enroll-0x2>
80000040:    0004    0x4
80000042:    0000    c.unimp
80000044:    0005    c.addi  x0,1
80000046:    0000    c.unimp
80000048:    0006    c.slli  x0,0x1
8000004a:    0000    c.unimp
8000004c:    00000007    0x7
80000050:    0008    0x8
        ...

\end{minted}
\end{code}

\clearpage
Можно сказать, что данная программа эквивалентна следующему псевдокоду на языке C, представленному на листинге \ref{lst:v33}.

\begin{code}
\caption{Псевдокод программы 6 варианта}
\label{lst:v33}

\begin{minted}{c}
// суммирует все элементы массива и прибавляет 1
len = 8; // длина массива = 8
enroll = 2; // сколько элементов обработаем за итерацию
elem_sz = 4; // размер элемента

x[len] = {1,2,3,4,5,6,7,8} // массив

x20 = len / enroll; // количество итераций
x1 = &x; // адрес массива
x31 = 0;

do {
    x2 = x1[0]; // 0 элемент
    x3 = x1[1]; // 4 элемент 

    x1 += elem_sz * enroll; // сдвиг по массиву

    x20 -= 1 // уменьшение количества итераций
    x31 += x2  
    x31 += x3 
} while (x20 != 0);

x31 += 1 // Результат = 37

while (1) {}; // бесконечный цикл
\end{minted}
\end{code}



\subsubsection*{Трасса работы программы}
Трасса работы представлена на рисунке \ref{img:8.png}.
\img{90mm}{8.png}{Трасса выполнения программы}
\clearpage

\subsubsection*{Результат работы программы}
Результат работы представлен на рисунке \ref{img:4.jpeg}.
\img{22mm}{4.jpeg}{Результат работы программы} 

\subsubsection*{Временные диаграммы}
Временные диаграммы сигналов, соответствующих всем стадиям выполнения команды, обозначенной в тексте программы символом \#! (lw x3, 4(x1)) представлены на рисунках \ref{img:5.jpeg}~---~\ref{img:7.jpeg}.
\img{120mm}{5.jpeg}{Временные диаграммы сигналов - F \& ID} 
\img{77mm}{6.jpeg}{Временные диаграммы сигналов - D \& M} 
\img{77mm}{7.jpeg}{Временные диаграммы сигналов - M} 

\subsection*{Вывод и предложение по оптимизации}
Как видно на трассе работы программы, задержек и простоев не вознакает, следовательно нет более оптимального варианта для данной программы.
\clearpage
