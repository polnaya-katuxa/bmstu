\chapter*{Общая программа для всех вариантов}

\subsection*{Исследуемая программа}

Код программы представлен в листингах \ref{lst:v1}~--~\ref{lst:v1p}.

\begin{code}
\caption{Код программы для всех вариантов}
\label{lst:v1}

\begin{minted}{text}
	.section .text
	.globl _start;
	len = 8 
	enroll = 4 
	elem_sz = 4 
_start:
	addi x20, x0, len/enroll
	la x1, _x
loop:	
	lw x2, 0(x1)
	add x31, x31, x2
	lw x2, 4(x1)
	add x31, x31, x2
	lw x2, 8(x1)
	add x31, x31, x2
	lw x2, 12(x1)
	add x31, x31, x2
	addi x1, x1, elem_sz*enroll
	addi x20, x20, -1
	bne x20, x0, loop
	addi x31, x31, 1
forever: j forever
	
	.section .data
_x:	.4byte 0x1
	.4byte 0x2
	.4byte 0x3
\end{minted}
\end{code}


\begin{code}
\caption{Код программы для всех вариантов (продолжение)}
\label{lst:v1p}

\begin{minted}{text}
	.4byte 0x4
	.4byte 0x5
	.4byte 0x6
	.4byte 0x7
	.4byte 0x8
\end{minted}
\end{code}


Дизассемблерный код представлен на листинге \ref{lst:v2}.

\begin{code}
\caption{Дизассемблированный код общей программы}
\label{lst:v2}

\begin{minted}{text}
Disassembly of section .text:

80000000 <_start>:
80000000:    00200a13    addi    x20,x0,2
80000004:    00000097    auipc   x1,0x0
80000008:    03c08093    addi    x1,x1,60 # 80000040 <_x>

8000000c <lp>:
8000000c:    0000a103    lw      x2,0(x1)
80000010:    002f8fb3    add     x31,x31,x2
80000014:    0040a183    lw      x3,4(x1)
80000018:    003f8fb3    add     x31,x31,x3
8000001c:    0080a203    lw      x4,8(x1)
80000020:    00c0a283    lw      x5,12(x1)
80000024:    004f8fb3    add     x31,x31,x4
80000028:    005f8fb3    add     x31,x31,x5
8000002c:    01008093    addi    x1,x1,16
80000030:    fffa0a13    addi    x20,x20,-1
80000034:    fc0a1ce3    bne     x20,x0,8000000c <lp>
80000038:    001f8f93    addi    x31,x31,1

8000003c <lp2>:
8000003c:    0000006f    jal     x0,8000003c <lp2>
\end{minted}
\end{code}

\clearpage

Можно сказать, что данная программа эквивалентна следующему псевдокоду на языке C, представленному на листинге \ref{lst:v3}.

\begin{code}
\caption{Псевдокод общей программы}
\label{lst:v3}

\begin{minted}{c}
#define len 8
#define enroll 4
#define elem_sz 4
int _x[]={1,2,3,4,5,6,7,8};
void _start() {
	int x20 = len/enroll;
	int *x1 = _x;
	
	do {
		int x2 = x1[0];
		x31 += x2;
		x2 = x1[1];
		x31 += x2;
		x2 = x1[2];
		x31 += x2;
		x2 = x1[3];
		x31 += x2;
		x1 += enroll;
		x20--;
	} while(x20 != 0);
	x31++;
	while(1){}
}
\end{minted}
\end{code}

\newpage
\img{100mm}{9.png}{Трасса выполнения программы}


