\setcounter{page}{3}
\chapter*{Введение}
\addcontentsline{toc}{chapter}{Введение} 
В процессе реализации ПО зачастую появляется необходимость обработки одного набора данных в несколько этапов. Если эти этапы связаны между собой по данным, то есть данные, полученные на предыдущем этапе, являются входными для следующего, то увеличить эффективность поможет использование конвейерная обработка данных \cite{item13}. 

В работе рассмотрены два типа конвейерной обработки данных: линейная и параллельная. При линейной конвейеризации одновременно в системе с конвейером может находиться только одна заявка: пока её обработка не завершена, другие заявки не могут попасть в конвейер. Тогда частым являением будет простой линий конвейера. При параллельной конвейеризации время простоя линий сокращается в связи с независимостью работы каждой линии от работы остальных.

Цель работы: получение навыков программирования, тестирования полученного программного продукта и проведения замеров времени выполнения по результатам работы программы на примере конвейерной обработки данных из документов.\\

Задачи работы:
\begin{enumerate}[label={\arabic*)}]
	\item изучение теоретических основ конвейерной обработки данных;
	\item описание алгоритмов обработки данных, реализованных на разных этапах конвейера~---~токенизации текста, установки правил токенизации текста и сортировки токенов в алфавитном порядке;
	\item реализация данных алгоритмов;
	\item реализация линейных конвейерных вычислений с не менее чем тремя линиями;
	\item реализация параллельных конвейерных вычислений с не менее чем тремя линиями;
	\item проведение замеров времени работы (в мкс) данных алгоритмов; 
	\item получение графической зависимости замеряемой величины от количества заявок, предоставляемых на вход конвейеру;
	\item проведение сравнительного анализа двух представленных реализаций конвейерных вычислений на основе полученной зависимости.
\end{enumerate}

\newpage