\chapter*{Заключение}
\addcontentsline{toc}{chapter}{Заключение} 
Из проведённых замеров времени работы реализаций последовательной и параллельной конвейерной обработки данных можно сделать следующие выводы.
Реализация параллельных конвейерных вычислений выполняется быстрее реализации линейных конвейерных вычислений в 1.26 раза при 8 заявках.
При реализации параллельных конвейерных вычислений возникает ситуация, когда на двух линиях простоя в очереди практически нет~---~общее время простоя на первой линии составляет 491052 микросекунд против 78 микросекунды на второй линии. Это обусловлено тем, что первый обработчик конвейера выполняет обработку данных дольше, чем другие. 
Среднее время заявки в системе на синхронном конвейере в 1.12 раз больше, чем на асинхронном. Такое поведение является следствием отсутствия простоя во всех очередях, кроме первичной, в случае отсутствия параллельности вычислений. Но, несмотря на это, реализация параллельной конвейерной обработки в целом будет работать быстрее, так как многие вычисления для разных заявок происходят одновременно.

В ходе выполнения лабораторной работы была достигнута поставленная цель: были получены навыки программирования, тестирования полученного программного продукта и проведения замеров времени выполнения и потребляемой памяти по результатам работы программы на примере конвейерной обработки данных из документов.

В процессе выполнения лабораторной работы были также реализованы все поставленные задачи, а именно:
\begin{itemize}
	\item были изучены теоретические основы конвейерной обработки данных;
	\item были описаны алгоритмы обработки данных, реализованных на этапах конвейера~---~токенизации текста, установки правил токенизации текста и сортировки токенов в алфавитном порядке;
	\item была выполнена программная реализация данных алгоритмов;
	\item была выполнена программная реализация линейных конвейерных вычислений с не менее чем тремя линиями;
	\item была выполнена программная реализация параллельных конвейерных вычислений с не менее чем тремя линиями;
	\item были проведены замеры времени работы (в мкс) для данных алгоритмов;
	\item была получена зависимость замеряемой величины от количества заявок, предоставляемых на вход конвейеру;
	\item был проведен сравнительный анализ двух представленных реализаций конвейерных вычислений на основе полученной зависимости.
\end{itemize}

\newpage