\chapter{Технологическая часть}

В данном разделе будет представлена реализация алгоритмов конвейерных вычислений, токенизации, применения правил к токенам и сортировки токенов по алфавиту. Также будут указаны обязательные требования к ПО, средства реализации алгоритмов и результаты проведённого тестирования программы.

\section{Требования к ПО}
Для программы выделен перечень требований:
\begin{itemize}
	\item программой предоставляется интерфейс в формате меню с возможностью ввода пути к директории с обрабатываемыми документами и количества обрабатываемых документов (заявок);
	\item программой обрабатываются документы, находящиеся в указанной директории;
	\item программой принимается путь к конфигурационному файлу с правилами токенизации в качестве параметра при запуске;
	\item программой предлагается повторно выполнить ввод при невалидном выборе пункта меню;
	\item программой производится аварийное завершение с текстом об ошибке при иных ошибках;
	\item программой проводится модульное тестирование функций реализации конвейерных вычислений, токенизации, применения правил к токенам;
	\item программой производятся замеры времени выполнения;
	\item программой сортируются токены по возрастанию в лексикографическом порядке;
	\item программой допускается только ввод файлов, текст которых содержит буквы русского алфавита, арабские цифры и знаки препинания.
\end{itemize}

\section{Средства реализации}
Для реализации данной работы был выбран язык программирования Go \cite{web_item2}. Выбор обусловлен наличием в $Go$ библиотек для тестирования ПО и проведения замеров времени выполнения в микросекундах \cite{web_item12}\cite{web_item15}, а также необходимых для реализации поставленных цели и задач средств. Также, $Go$ предоставляет возможность реализовывать потокобезопасные очереди и конвейеры при использовании каналов типа $chan$ и параллельных операций $goroutine$, которые могут выполняться независимо от функции, в которой они запущены \cite{item16}. В качестве среды разработки была выбрана $GoLand$ \cite{web_item4}.

\section{Реализация алгоритмов}
В листингах \ref{code:lin1}~--~\ref{code:goroutines2} представлены реализации функций линейных и параллельных конвейерных вычислений, и подпрограмм токенизации и применения правил к токенам в листингах \ref{code:token}~--~\ref{code:rule2}. 

\begin{code}
\caption{Листинг функции линейных конвейерных вычислений (начало)}
\label{code:lin1}
\begin{minted}{go}
func LaunchLinear(docs []document.Document, rules []rule.Rule) 
[]document.Document {
	result := make([]document.Document, 0)
	tasks := make([]Task, 0)

	for _, v := range docs {
		input := make(chan Task, len(docs))
		tokenized := make(chan Task, len(docs))
		ruled := make(chan Task, len(docs))
		sorted := make(chan Task, len(docs))

		task := Task{
			Doc:       v,
			TimeFirst: time.Now(),
		}
		input <- task
		close(input)
\end{minted}
\end{code}

\newpage

\begin{code}
\caption{Листинг функции линейных конвейерных вычислений (окончание листинга \ref{code:lin1})}
\label{code:lin2}
\begin{minted}{go}
		tokeniser(input, tokenized)
		ruler(tokenized, ruled, rules)
		sorter(ruled, sorted)

		res := <-sorted

		result = append(result, res.Doc)
		tasks = append(tasks, res)
	}

	LogTasks(tasks)

	return result
}
\end{minted}
\end{code}

\begin{code}
\caption{Листинг функции параллельных конвейерных вычислений (начало)}
\label{code:par1}
\begin{minted}{go}
func LaunchParallel(docs []document.Document, rules []rule.Rule) 
[]document.Document {
	result := make([]document.Document, 0)
	tasks := make([]Task, 0)
	input := make(chan Task, len(docs))
	tokenized := make(chan Task, len(docs))
	ruled := make(chan Task, len(docs))
	sorted := make(chan Task, len(docs))
	for _, v := range docs {
		task := Task{
			Doc:       v,
			TimeFirst: time.Now(),
		}
		input <- task
	}
\end{minted}
\end{code}

\begin{code}
\caption{Листинг функции параллельных конвейерных вычислений (окончание листинга \ref{code:par1})}
\label{code:par2}
\begin{minted}{go}
	go tokeniser(input, tokenized)
	go ruler(tokenized, ruled, rules)
	go sorter(ruled, sorted)
	close(input)

	for v := range sorted {
		result = append(result, v.Doc)
		tasks = append(tasks, v)
	}
	LogTasks(tasks)
	return result
}
\end{minted}
\end{code}

\begin{code}
\caption{Листинг функций-обработчиков (начало)}
\label{code:goroutines1}
\begin{minted}{go}
func tokeniser(in <-chan Task, out chan<- Task) {
	for v := range in {
		v.TimeStart1 = time.Now()
		v.Doc.Tokenize()
		v.TimeEnd1 = time.Now()
		out <- v
	}
	close(out)
}

func ruler(in <-chan Task, out chan<- Task, rules []rule.Rule) {
	for v := range in {
		v.TimeStart2 = time.Now()
		v.Doc.ApplyRules(rules)
		v.TimeEnd2 = time.Now()
		out <- v
	}
	close(out)
}
\end{minted}
\end{code}

\begin{code}
\caption{Листинг функций-обработчиков (окончание листинга \ref{code:goroutines1})}
\label{code:goroutines2}
\begin{minted}{go}
func sorter(in <-chan Task, out chan<- Task) {
	for v := range in {
		v.TimeStart3 = time.Now()
		sort.Strings(v.Doc.Tokens)
		v.TimeEnd3 = time.Now()
		out <- v
	}
	close(out)
}
\end{minted}
\end{code}

\begin{code}
\caption{Листинг функции токенизации}
\label{code:token}
\begin{minted}{go}
func (doc *Document) Tokenize() {
	doc.Tokens = nil
	re := regexp.MustCompile(`([^0-9а-яА-ЯЁё-])`) //|[+]
	splitted := re.Split(doc.Text, -1)
	for i := range splitted {
		if splitted[i] != "" && splitted[i] != "-" {
			doc.Tokens = append(doc.Tokens,
			strings.ToLower(splitted[i]))
		}
	}
}
\end{minted}
\end{code}

\begin{code}
\caption{Листинг функции применения правил к токенам (начало)}
\label{code:rule1}
\begin{minted}{go}
func (doc *Document) ApplyRules(rules []rule.Rule) error {
	if len(doc.Tokens) == 0 {
		return errors.New("no tokens")
	}
	if len(rules) == 0 {
		return nil
	}
	ruled := make([]string, 0)
\end{minted}
\end{code}

\begin{code}
\caption{Листинг функции применения правил к токенам (окончание листинга \ref{code:rule1})}
\label{code:rule2}
\begin{minted}{go}
	for i := 0; i < len(doc.Tokens); i++ {
		ruleInd := -1

		for j := 0; j < len(rules) && ruleInd < 0; j++ {
			ruleInd = j

			for k := 0; k < len(rules[j].Option) &&
			ruleInd >= 0; k++ {
				if i+k >= len(doc.Tokens) ||
				doc.Tokens[i+k] != rules[j].Option[k] {
					ruleInd = -1
				}
			}
		}

		if ruleInd < 0 {
			ruled = append(ruled, doc.Tokens[i])
		} else {
			ruled = append(ruled, rules[ruleInd].Standard)
			i += len(rules[ruleInd].Option) - 1
		}
	}

	doc.Tokens = ruled

	return nil
}
\end{minted}
\end{code}

\newpage

\section{Тестирование}
В таблице \ref{table:tests} представлены тесты для конвейерной обработки данных. Все тесты пройдены успешно. Документы для тестов либо составлялись вручную, либо копировались со случайных сайтов.

\begin{table}[H]
  \caption{\label{table:tests} Тесты для конвейерной обработки данных}
  \begin{center}
    \begin{tabular}{|c|c|c|}
      \hline
      % &  &  & \\
      %\cline{3-5}
      № &  Количество заявок & Результат \\ \hline
      1 & 0 & Сообщение об ошибке \\ \hline
      2 & -3 & Сообщение об ошибке \\ \hline
      3 & 1 & Журналирование конвейерных вычислений \\ \hline
      4 & 3 & Журналирование конвейерных вычислений  \\ \hline
      5 & 8 & Журналирование конвейерных вычислений\\ \hline
      6 & 10 & Журналирование конвейерных вычислений \\ \hline
    \end{tabular}
  \end{center}
\end{table}

\newpage