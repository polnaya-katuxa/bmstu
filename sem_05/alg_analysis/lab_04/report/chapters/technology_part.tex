\chapter{Технологическая часть}

В данном разделе будет представлена реализация алгоритма обратной трассировки лучей. Также будут указаны требования к ПО, средства реализации алгоритмов и результаты проведённого тестирования программы.

\section{Требования к ПО}
Для автоматизации работы программы и использования сценария необходимо реализовать передачу в программу аргументов командной строки и их распознавание самой программой. Были предусмотрены следующие флаги:

\begin{itemize}
	\item флаг «-d» - передача в программу значения глубины рекурсивных погружений, за флагом следует положительное целочисленное значение;
	\item флаги «-i», «-o» - передача в программу имени входного или выходного файла, за флагом следует строка с именем файла;
	\item флаг «-n» - передача в программу количества выделяемых потоков;
	\item флаг «-m» - указание программе о том, что проводятся замеры, при этом обязательно наличие параметров «-d» и «-i», а «-o» становится необязательным, так как полученное изображение в данном случае не используется;
	\item при стандартном запуске программы, когда важно полученное изображение, передача параметров «-d», «-i» и «-o» является обязательной, в случае отсутствия обязательных параметров программой генерируется ненулевой код возврата.
\end{itemize}

\section{Средства реализации}
В качестве языка программирования для реализации данной лабораторной работы был выбран язык программирования $C++$ \cite{web_item4}. Данный выбор обусловлен наличием инструментов для реализации многопоточности. Также был выбран фреймворк $Qt$ \cite{web_item2} в связи с наличием необходимых для работы с компьютерной графикой библиотек и встроенных средств. Для передачи сведений об объектах сцены в программу использовались файлы с расширением $txt$, $obj$, $mtl$, как наиболее широко применяемые в сфере компьютерной графики.

\section{Реализация алгоритмов}
В листингах \ref{code:ray1}~--~\ref{code:ray3} представлена реализация алгоритма обратной трассировки лучей. В листингах \ref{code:ray_one}~--~\ref{code:ray_many3} представлены реализации однопоточного и многопоточного синтеза изображения.

\begin{code}
\caption{Листинг функции реализации алгоритма обратной трассировки лучей (начало)}
\label{code:ray1}
\begin{minted}{c++}
QColor Drawing::castRay(const sight_t& sight, const int& depth)
{
    double intensity = 0.0;
    double specIntensity = 0.0;
    std::tuple<bool, double, QVector3D, int> res = sceneIntersect(sight);
    if (depth > this->depth || !std::get<0>(res))
        return QColor(0, 0, 0);

    double t = std::get<1>(res);
    QVector3D norm = std::get<2>(res);
    int closest = std::get<3>(res);
    QVector3D cam = sight.cam;
    QVector3D dir = sight.dir;
    QVector3D intersection = cam + dir * t;

    QVector3D reflectDir = reflect(dir, norm).normalized();
    sight_t sightRefl = {
        .cam = intersection,
        .dir = reflectDir
    };
    QColor reflectCol = castRay(sightRefl, depth + 1);
    
    QVector3D refractDir = refract(dir, norm, 
    figures[closest]->getMaterial().getRefCoef()).normalized();
    sight_t sightRefr = {
        .cam = intersection,
        .dir = refractDir
    };
\end{minted}
\end{code}

\begin{code}
\caption{Листинг функции реализации алгоритма обратной трассировки лучей (продолжение листинга \ref{code:ray1})}
\label{code:ray2}
\begin{minted}{c++}
    QColor refractCol = castRay(sightRefr, depth + 1);
    
    for (size_t k = 0; k < lightSources.size(); k++) {
        QVector3D light = intersection - lightSources[k].getPos();
        light = light.normalized();
        double t1 = 0.0;
        bool flag = false;
        sight_t sightLight = {
            .cam = lightSources[k].getPos(),
            .dir = light
        };
        std::tuple<bool, double, QVector3D> res1 = 
        figures[closest]->rayIntersection(sightLight, lightSources);

        t1 = std::get<1>(res1);
        std::tuple<bool, double, QVector3D, int> res2 = 
        sceneIntersect(sightLight);
        if (std::get<1>(res2) < t1)
            flag = true;

        if (!flag) {
            intensity += lightSources[k].getIntensity() * 
            std::max(0.f, QVector3D::dotProduct(norm, (-1) * light));

            QVector3D mirrored = reflect(light, norm);
            mirrored = mirrored.normalized();
            double angle = QVector3D::dotProduct(mirrored, dir * (-1));
            specIntensity += powf(std::max(0.0, angle), 
            figures[closest]->getMaterial().getSpecCoef()) * 
            lightSources[k].getIntensity();
        }
    }
\end{minted}
\end{code}

\begin{code}
\caption{Листинг функции реализации алгоритма обратной трассировки лучей (окончание листинга \ref{code:ray2})}
\label{code:ray3}
\begin{minted}{c++}
    QColor diffColor = figures[closest]->getMaterial().getDiffColor();
    QColor specColor = figures[closest]->getMaterial().getSpecColor();

    QColor color = getColor(intensity, specIntensity, 
    figures[closest]->getMaterial().getAlbedo(), diffColor, 
    specColor, reflectCol, refractCol);

    return color;
}
\end{minted}
\end{code}

\begin{code}
\caption{Листинг функции реализации однопоточного синтеза изображения}
\label{code:ray_one}
\begin{minted}{c++}
std::shared_ptr<QImage> Drawing::drawFigures()
{
    std::shared_ptr<QImage> image = std::make_shared<QImage>(canvasWidth, 
    canvasHeight, QImage::Format_RGB32);
    image->fill(Qt::black);
    QVector3D cam = QVector3D(0, 0, 3000);
    sight_t sight = {
        .cam = cam 
    };
    
    for (int y = 0; y < canvasHeight; y++)
        for (int x = 0; x < canvasWidth; x++) {
            QVector3D pix = QVector3D(x, y, 200);
            QVector3D dir = (pix - cam).normalized();
            sight.dir = dir;
            QColor refColor = castRay(sight, 0);
            image->setPixel(x, y, qRgb(refColor.red(), refColor.green(),
            refColor.blue()));
        }

    return image;
}
\end{minted}
\end{code}

\newpage

\begin{code}
\caption{Листинг функции реализации многопоточного синтеза изображения (начало)}
\label{code:ray_many1}
\begin{minted}{c++}
std::shared_ptr<QImage> Drawing::drawFigures(int numThreads)
{
    std::shared_ptr<QImage> image = std::make_shared<QImage>(canvasWidth,
    canvasHeight, QImage::Format_RGB32);
    image->fill(Qt::black);

    QVector3D cam = QVector3D(0, 0, 3000);
    sight_t sight = {
        .cam = cam
    };

    int size = canvasHeight * canvasWidth;
    int cur = 0;
    int win = (size / numThreads);
    std::vector<QVector3D> limits;
    limits.push_back(QVector3D(0,0,200));

    std::vector<std::vector<uint>> colors(canvasHeight,
    std::vector<uint>(canvasWidth, 0));

    if (numThreads == 0) {
        limits.push_back(QVector3D(canvasWidth, canvasHeight, 200));
        drawThread(limits[0], limits[1], sight, colors);
    } else {
        for (int i = 0; i < numThreads; i++) {
            cur += win;
            limits.push_back(QVector3D(cur % canvasWidth,
            cur / canvasWidth, 200));
        }
        limits[limits.size() - 1] = QVector3D(canvasWidth,
        canvasHeight, 200);
\end{minted}
\end{code}

\begin{code}
\caption{Листинг функции реализации многопоточного синтеза изображения (продолжение листинга \ref{code:ray_many1})}
\label{code:ray_many2}
\begin{minted}{c++}
        std::vector<std::thread> threads(numThreads);
        
        for (int i = 0; i < numThreads; i++) {
            threads[i] = std::thread(&Drawing::drawThread, this,
            std::ref(limits[i]), std::ref(limits[i+1]), sight, 
            std::ref(colors));
        }
        for (int i = 0; i < numThreads; i++)
            threads[i].join();
    }

    for (int y = 0; y < canvasHeight; y++) {
        for (int x = 0; x < canvasWidth; x++) {
            image->setPixel(x, y, colors[y][x]);
        }
    }

    return image;
}
\end{minted}
\end{code}

\begin{code}
\caption{Листинг функции реализации многопоточного синтеза изображения (продолжение листинга \ref{code:ray_many2})}
\label{code:ray_many3}
\begin{minted}{c++}
void Drawing::drawThread(QVector3D &preLim, QVector3D &lim, sight_t sight,
std::vector<std::vector<uint>> &image)
{
    int startX = 0, endX = canvasWidth;

    int startY = preLim.y();
    if (preLim.x() != 0 || preLim.y() != 0) {
        startY--;
    }
\end{minted}
\end{code}

\begin{code}
\caption{Листинг функции реализации многопоточного синтеза изображения (продолжение листинга \ref{code:ray_many3})}
\label{code:ray_many4}
\begin{minted}{c++}
    for (int y = startY; y < lim.y(); y++)
    {
        if (y == startY) {
            startX = preLim.x();
        } else {
            startX = 0;
        }

        if (y == lim.y() - 1) {
            endX = lim.x();
        } else {
            endX = canvasWidth;
        }

        for (int x = startX; x < endX; x++) {
            QVector3D pix = QVector3D(x, y, 200);
            QVector3D dir = (pix - sight.cam).normalized();

            sight.dir = dir;
            QColor refColor = castRay(sight, 0);
            image[y][x] = qRgb(refColor.red(), refColor.green(),
            refColor.blue());
        }
    }
}
\end{minted}
\end{code}

\section{Тестирование}
На рисунках \ref{img:p1}~---~\ref{img:p3} представлены тесты для алгоритма обратной трассировки лучей. Тестирование проводилось по проверке корректности получаемых изображений, по методологии белого ящика. Для передачи сведений об объектах сцены в программу использовались файлы с расширением $txt$, $obj$, $mtl$. Все тесты пройдены успешно.

\begin{figure}[h!]
    \centering
    \includegraphics[width=1\linewidth]{p1.pdf}
    \caption{Изображение-тест №1}
    \label{img:p1}
\end{figure}

\newpage

\begin{figure}[h!]
    \centering
    \includegraphics[width=1\linewidth]{p2.pdf}
    \caption{Изображение-тест №2}
    \label{img:p2}
\end{figure}

\newpage

\begin{figure}[h!]
    \centering
    \includegraphics[width=1\linewidth]{p3.pdf}
    \caption{Изображение-тест №3}
    \label{img:p3}
\end{figure}

\newpage