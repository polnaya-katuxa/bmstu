\chapter{Аналитическая часть}
В данном разделе будут рассмотрены теоретические основы алгоритма обратной трассировки лучей.

\section{Алгоритм обратной трассировки лучей}
Алгоритм обратной трассировки лучей работает в пространстве изображения \cite{web_item2}. Предполагается, что сцена уже преобразована в это пространство.

Наблюдатель находится на положительной полуоси $OZ$. Картинная плоскость, т.е. растр, перпендикулярна оси $OZ$. Каждый луч проходит через пиксель растра до сцены. Траектория каждого луча отслеживается, чтобы определить, какие именно объекты сцены, если таковые существуют, пересекаются с данным лучом. Необходимо проверить пересечение каждого объекта сцены с каждым лучом. Пересечение с максимальным значением $z$ представляет видимую поверхность для данного пикселя.

Для дальнейшего определения цвета пикселя рассматриваются лучи от точки пересечения луча наблюдения с объектом к каждому источнику света. Если на пути к источнику света луч пересекает иной объект, то свет от того источника не учитывается в расчёте цвета данного пиксела. Если для луча от наблюдателя (камеры) не найдено объектов, с которыми он пересекается, пиксел закрашивается цветом фона.

Для расчёта отражений (преломлений) луча, встретившего на своей траектории объект, используются физические законы (равенство угла падения и отражения, закон Снеллиуса), рассчитывается направление луча отраженного (преломлённого). Найденная точка пересечения теперь считается точкой наблюдения, и описанный алгоритм испускания луча повторяется столько раз, сколько составляет максимальная глубина рекурсивных погружений.

Метод прямой трассировки предполагает построение траекторий лучей от всех источников освещения ко всем точкам всех объектов сцены. Тогда достаточно часто будут выполняться расчеты для лучей, которые не попадут в камеру, результаты которых не будут учтены. Поэтому данный вариант алгоритма считается неэффективным.

Алгоритм трассировки лучей подразумевает множество вычислений, поэтому синтез изображения происходит долго. Возможной является многопоточная реализация алгоритма, при которой вычисление цвета каждого пиксела может быть выполнено параллельно. Это позволит ускорить синтез изображения.

\section{Алгоритм обратной трассировки лучей в однопоточной реализации}
При работе с одним потоком при реализации алгоритма трассировки лучей осуществляется обход всех пикселей растра изображения. В каждый пиксель испускается луч из положения наблюдателя, и соответствующий пиксель закрашивается рассчитанным цветом.

Однопоточная реализация алгоритма работает достаточно долго, так как требуется произвести большое количество вычислений: поиск пересечений со всеми объектами сцены луча наблюдателя на каждом пикселе, определение теней, рекурсивный расчёт траектории отражённого и преломлённого лучей и т.д.

\section{Алгоритм обратной трассировки лучей в многопоточной реализации}
При работе с одном потоком при реализации алгоритма трассировки лучей растр делится на участки (окна), размер которых равен:
\begin{equation}
	window = \frac{size}{numThreads},
\end{equation}
где size~---~общее количество пикселов растра, а numThreads~---~количество выделяемых потоков.

Создаётся специальный массив, хранящий последний входящий в окно пиксел для каждого выделенного участка. Размер всех окон одинаков, кроме последнего, границей которого всегда является последний (крайний правый нижний) пиксел растра.

При проходе по окну все строки, кроме последней для данного окна, обрабатываются целиком, а последняя~---~до указанной во вспомогательном массиве крайней координаты обрабатываемых пикселов по ширине растра.

Каждое окно обрабатывается так же, как и весь растр при однопоточной реализации алгоритма, но за счёт параллельности вычислений алгоритм работает быстрее.

\newpage