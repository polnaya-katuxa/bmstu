\setcounter{page}{3}
\chapter*{Введение}
\addcontentsline{toc}{chapter}{Введение} 
Поток выполнения~---~наименьшая единица обработки, исполнение которой назначает ядро операционной системы. Также поток можно определить как часть кода, которая может выполняться параллельно с другими частями кода.

Потоки разделяют адресное пространство процесса, а именно код, данные и его контекст. Фактический порядок выполнения потоков зависит от количества процессоров (ядер): если их несколько, то потоки будут действительно исполняться параллельно, а если нет~---~квазипараллельно, то есть будет происходить поочерёдное выделение квантов времени потокам.

К достоинствам многопоточности относятся:
\begin{itemize}
	\item отзывчивость~---~обеспечение быстрой реакции одного потока во время блокировки или занятости других;
	\item разделение ресурсов~---~выполнение нескольких задач одновременно в одном адресном пространстве;
	\item экономичность~---~переключение контекста происходит быстрее, чем при создании нового процесса;
	\item масштабируемость~---~использование мультипроцессорной или однопроцессорной архитектуры.
\end{itemize}

Цель работы: получение навыков реализации многопоточности для программного продукта, сравнение быстродействия для однопоточной и многопоточной реализаций алгоритма обратной трассировки лучей.\\

Задачи работы:
\begin{enumerate}[label={\arabic*)}]
	\item изучение алгоритма обратной трассировки лучей;
	\item разработать последовательную и параллельную версии алгоритма обратной трассировки лучей;
	\item разработка схем данных алгоритмов;
	\item программная реализация данного алгоритма с использованием одного потока;
	\item программная реализация данного алгоритма с использованием вспомогательных потоков;
	\item проведение замеров времени работы (в мс) данных реализаций алгоритма; 
	\item получение зависимости замеряемых величин от количества потоков;
	\item проведение сравнительного анализа данных реализаций алгоритма на основе полученных зависимостей.
\end{enumerate}

\newpage