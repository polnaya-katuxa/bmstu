\setcounter{page}{3}
\chapter*{Введение}
\addcontentsline{toc}{chapter}{Введение} 
В сфере обработки информации существует множество разнообразных задач, для решения которых требуется найти наиболее оптимальный алгоритм. Решение многих подобных задач значительно упрощается при использовании сортировки данных, считающейся наиболее фундаментальной задачей при изучении алгоритмов.

Сортировка~---~это процесс упорядочения некоторого множества элементов, на котором определены отношения порядка:
\begin{itemize}
	\item >;
	\item <;
	\item >=;
	\item <=.
\end{itemize}

Алгоритмы сортировки имеют большое практическое применение. Их можно встретить почти везде, где речь идет об обработке и хранении больших объемов информации, например, в обработке баз данных или математических программах.

В данной лабораторной работе алгоритм сортировки будет рассматриваться как алгоритм для упорядочивания элементов в массиве. 

На данный момент определено множество алгоритмов сортировки данных в массиве, и они  постоянно оптимизируются. Алгоритмы сортировки формируют отдельный класс алгоритмов.

Цель работы: получение навыков программирования, тестирования полученного программного продукта и проведения замеров времени выполнения и потребляемой памяти по результатам работы программы на примере реализации алгоритмов сортировки.\\

Задачи работы:
\begin{enumerate}[label={\arabic*)}]
	\item изучение алгоритмов блинной и быстрой сортировок и сортировки бусинами;
	\item разработка схем данных алгоритмов и анализ их трудоёмкости;
	\item реализация данных алгоритмов;
	\item проведение замеров потребления памяти (в байтах) для данных алгоритмов; 
	\item проведение замеров времени работы (в нс) данных алгоритмов; 
	\item получение графической зависимости замеряемых величин от длины массива, предоставляемого на вход алгоритмам;
	\item проведение сравнительного анализа данных алгоритмов на основе полученных зависимостей.
\end{enumerate}

\newpage