\chapter{Технологическая часть}

В данном разделе будет представлена реализация алгоритмов сортировки. Также будут указаны обязательные требования к ПО, средства реализации алгоритмов и результаты проведённого тестирования программы.

\section{Требования к ПО}
Для программы выделен перечень требований:
\begin{itemize}
	\item программа предоставляет интерфейс в формате меню с возможностью ввода и изменения обрабатываемого массива, завершения работы с программой и выбора используемого для обработки введённого массива алгоритма сортировки;
	\item предлагается повторно выполнить ввод при невалидном выборе пункта меню;
	\item производится аварийное завершение с текстом об ошибке при иных ошибках;
	\item проводится модульное тестирование функций сортировки;
	\item производятся замеры времени выполнения и потребления памяти функциями сортировки;
	\item вызывается исключение при вводе пустого массива;
	\item сортируются массивы только по возрастанию;
	\item допускается только сортировка массивов натуральных чисел.
\end{itemize}

\section{Средства реализации}
Для реализации данной работы был выбран язык программирования Go \cite{web_item2}. Выбор обусловлен наличием в $Go$ библиотек для тестирования ПО и проведения замеров времени выполнения в наносекундах \cite{web_item11}, а также необходимых для реализации поставленных цели и задач средств. В качестве среды разработки была выбрана $GoLand$ \cite{web_item4}.

\section{Реализация алгоритмов}
В листингах \ref{code:pancake}~--~\ref{code:bead} представлены реализации алгоритмов сортировок: блинной, быстрой и сортировки бусинами, и некоторых нужных подпрограмм в листингах \ref{code:max}~--~\ref{code:flip}. 

\begin{code}
\caption{Листинг функции поиска индекса максимума массива}
\label{code:max}
\begin{minted}{go}
func getIndMax(arr []int) int {
	iMax := 0
	for i := range arr {
		if arr[i] > arr[iMax] {
			iMax = i
		}
	}
	return iMax
}
\end{minted}
\end{code}

\begin{code}
\caption{Листинг функции поиска переворота массива}
\label{code:flip}
\begin{minted}{go}
func flip(arr []int) {
	n := len(arr)
	for left := 0; left < n; left++ {
		arr[left], arr[n-1] = arr[n-1], arr[left]
		n--
	}
}
\end{minted}
\end{code}

\begin{code}
\caption{Листинг алгоритма блинной сортировки}
\label{code:pancake}
\begin{minted}{go}
func Pancakesort(arr []int) {
	for n := len(arr); n > 1; n-- {
		iMax := getIndMax(arr[:n])
		if iMax != n-1 {
			flip(arr[:(iMax + 1)])
			flip(arr[:n])
		}
	}
}
\end{minted}
\end{code}

\begin{code}
\caption{Листинг алгоритма быстрой сортировки}
\label{code:quick}
\begin{minted}{go}
func quicksort(arr []int) {
	n := len(arr)
	less := make([]int, 0, n)
	greater := make([]int, 0, n)
	pivot := 0

	if n > 1 {
		pivot = arr[n/2]

		for i, v := range arr {
			if i != n/2 {
				if v < pivot {
					less = append(less, v)
				} else {
					greater = append(greater, v)
				}
			}
		}

		quicksort(less)
		quicksort(greater)
		
		l := len(less)
		g := len(greater)

		for i := 0; i < l; i++ {
			arr[i] = less[i]
		}
		arr[l] = pivot
		for i := l + 1; i < l+g+1; i++ {
			arr[i] = greater[i-l-1]
		}
	}
}
\end{minted}
\end{code}

\begin{code}
\caption{Листинг алгоритма сортировки бусинами}
\label{code:bead}
\begin{minted}{go}
func Beadsort(arr []int) {
	n := len(arr)
	max := arr[getIndMax(arr)]
	m := make([][]int, n)
	for i := 0; i < n; i++ {
		m[i] = make([]int, max)
	}
	for i := 0; i < n; i++ {
		for j := 0; j < arr[i]; j++ {
			m[i][j]++
		}
	}

	for j := 0; j < max; j++ {
		beadsInColumn := 0
		for i := 0; i < n; i++ {
			if m[i][j] == 1 {
				beadsInColumn++
				m[i][j] = 0
			}
		}
		for i := n - beadsInColumn; i < n; i++ {
			m[i][j] = 1
		}
	}

	for i := 0; i < n; i++ {
		beadsInRow := 0
		for j := 0; j < max; j++ {
			beadsInRow += m[i][j]
		}
		arr[i] = beadsInRow
	}
}
\end{minted}
\end{code}

\section{Тестирование}
В таблице  представлены тесты для алгоритмов блинной, быстрой и бусинной сортировки. Тестирование проводилось по методологии чёрного ящика. Все тесты пройдены успешно.

\begin{table}[H]
  \caption{\label{table:tests} Тесты для алгоритмов блинной, быстрой и бусинной сортировки}
  \begin{center}
    \begin{tabular}{|c|c|c|}
      \hline
      % &  &  & \\
      %\cline{3-5}
      № &  Входной массив & Результат \\ \hline
      1 & 4, 2, 7, 5, 8, 1, 6 & 1, 2, 4, 5, 6, 7, 8 \\ \hline
      2 & 4, 2, 2, 5, 2, 1, 6 & 1, 2, 2, 2, 4, 5, 6 \\ \hline
      3 & 2, 4, 5, 6, 8, 10 & 2, 4, 5, 6, 8, 10 \\ \hline
      4 & 10, 8, 6, 5, 4, 2 & 2, 4, 5, 6, 8, 10  \\ \hline
      5 & 2, 2, 2, 2, 2 & 2, 2, 2, 2, 2\\ \hline
      6 & 2 & 2 \\ \hline
      7 & 4, 2, 7, 5, 8, 10, 6, 21, 7, 3, 11, 9, 1 & 1, 2, 3, 4, 5, 6, 7, 7, 8, 9, 10, 11, 21 \\ \hline
    \end{tabular}
  \end{center}
\end{table}

\newpage