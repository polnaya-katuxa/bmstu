\chapter{Аналитическая часть}
В данном разделе будут рассмотрены теоретические основы алгоритмов блинной, быстрой и гравитационной сортировок. Все алгоритмы трактуются так, как будто сортировка выполняется по возрастанию.

\section{Алгоритм блинной сортировки}
Блинная сортировка~---~это класс алгоритмов, в которых допускается единственная операция~---~переворот элементов последовательности до заданного индекса. Соответственно минимизируемым параметром при оптимизации является количество произведённых переворотов. Это отличает данный класс алгоритмов от других, в которых к минимуму сводится количество сравнений.

 Процесс можно визуально представить как стопку блинов, которую тасуют путём взятия нескольких блинов сверху и их переворачивания. На каждой итерации находится максимум текущего массива, и часть массива от начала до максимума включая переворачивается. Так текущий максимум становится первым элементом в массиве. Затем выполняется переворот всего массива так, что максимум оказывается в конце массива. На каждой следующей итерации количество анализируемых элементов массива уменьшается на один, так как максимум уже занял свою позицию~---~последнюю в массиве.

\section{Алгоритм быстрой сортировки}
Быстрая сортировка является существенно улучшенным вариантом алгоритма сортировки с помощью прямого обмена. Этот алгоритм является одним из самых быстрых известных универсальных алгоритмов сортировки массивов. 

Основные этапы алгоритма можно изложить так:
\begin{enumerate}[label={\arabic*)}]
	\item выбирается опорный элемент массива (в разных реализациях в качестве опорных выбираются различные элементы~---~последний, первый, серединный);
	\item массив делится на две (элементы больше опорного и элементы меньше опорного, элементы равные опорному добавляются в любую из двух частей) или три части (элементы больше опорного, элементы меньше опорного и элементы равные опорному);
	\item первые два шага рекурсивно применяются к выделенным из массива группам элементов, если группа имеет больше одного элемента;
	\item отсортированные группы соединяются в один массив в порядке: меньшие опорного, большие опорного, в случае разбиения на две части, и в порядке: меньшие опорного, равные опорному, большие опорного, в случае разбиения на три части.
\end{enumerate}

\section{Алгоритм сортировки бусинами (гравитационный)}
Алгоритм сортировки бусинами (гравитационный алгоритм)~---~алгоритм сортировки, который может быть применён только к массиву натуральных чисел.

В процессе работы алгоритма будет заполнена матрица, имеющая размерность равную  $<\text{количество элементов массива}>\times<\text{значение максимального элемента массива}>$, нулями или единицами, в соответствии с величиной чисел, являющихся элементами сортируемого массива. Каждая строка такой матрицы будет представлять собой элемент массива с индексом, равным индексу строки в матрице, где сумма записанных подряд с первого элемента строки единиц будет равняться данному элементу массива. Эти единицы называются бусинами.

Алгоритм гравитационной сортировки может быть сравнен с тем, как бусины падают вниз на параллельных шестах (столбцах матрицы), например как в абаке, однако каждый из шестов может иметь разное количество бусин. 

После того, как все строки заполнены, все бусины нужно опустить вниз по шестам, как будто под действием гравитации. Тогда последний ряд будет представлять собой самое большое число списка, а первый — наименьшее. В результате суммирования единиц каждой строки будут получены элементы исходного массива в порядке возрастания от первой строки матрицы к последней. Их запись в исходный массив в порядке от первой строки матрицы к последней позволит получить исходный массив, отсортированный по возрастанию.

\newpage