\setcounter{page}{3}
\chapter*{Введение}
\addcontentsline{toc}{chapter}{Введение} 
В современном мире постоянно растёт объём доступной информации. С информацией необходимо взаимодействовать~---~получать, анализировать, хранить, осуществлять её поиск, транслировать. 

Если определённые блоки информации можно идентифицировать по определённому единичному значению, то подходящим методом хранения информации будет словарь. Словарь, или ассоциативный массив, означает абстрактный тип данных формата «ключ-значение». То есть одному заданному признаку ставятся в соответствие данные, определённым образом соотносящиеся с данным признаком.

Такой вариант хранения данных пригоден, если требуется осуществлять их периодический поиск, поэтому на основе словаря можно сформировать своеобразную поисковую систему. Однако, пользовательский ввод зачастую неточен, предлагаемые категории признака абстрактны и требуют конкретизации. Подобная экспертная информация трудно формализуема в рамках стандартных математических подходов. В связи с этим, необходимо воспользоваться правилами нечёткой логики \cite{item8}.

Цель работы: получение навыков программирования, тестирования полученного программного продукта и поиска по словарю при ограничении на значение признака, заданного при помощи лингвистической переменной.\\

Задачи работы:
\begin{enumerate}[label={\arabic*)}]
	\item выбор объекта и признака объекта для разделения на категории;
	\item формализация выбранного признака и определение категорий;
	\item составление анкеты для респондентов;
	\item проведение опроса среди респондентов по составленной анкете;
	\item построение функции принадлежности термам числовых значений при- знака, описываемого лингвистической переменной, на основе проведения статистического анализа мнений респондентов;
	\item описание допустимых формулировок при формировании запроса;
	\item описание алгоритма поиска в словаре сущностей, удовлетворяющих ограничению, заданному в запросе на ограниченном естественном языке, и алгоритма определения соответствия введённого запроса распознаваемым вариантам; 
	\item описание структуры данных словаря;
	\item реализация выбранного алгоритма поиска в словаре;
	\item анализ применимости предложенного алгоритма.
\end{enumerate}

\newpage