\chapter{Аналитическая часть}
В данном разделе будут рассмотрены применяемые правила нечёткой логики, представлена формализация объекта и выбранного признака классификации и описан алгоритм бинарного поиска в словаре.

\section{Формализация объекта и выбранного признака классификации}
Словарь~---~абстрактный тип данных формата «ключ-значение». То есть одному заданному признаку ставятся в соответствие данные, определённым образом соотносящиеся с данным признаком. 

В данной работе ключом словаря является количество шерсти у кошки на см$^2$ тела. Иными словами, степень пушистости кошки. Значением является структура, хранящая информацию о породе кошки, её пушистости и ссылку на изображение кошки. При поиске значения в словаре осуществляется поиск по ключу. Данные в словаре хранятся в отсортированном порядке.

\subsection{Применяемые правила нечёткой логики}
Представление экспертных знаний в конкретной предметной области пред- полагает определение конечного множества лингвистических переменных (в данной работе~---~одна, пушистость кошки), термов для лингвистической переменной, построение функции принадлежности и определение диапазонов значений признака, соответствующих каждому терму.

Задаётся множество термов
\begin{equation}
	T^{'}_i = (t_1, t_2, ..., t_m),
\end{equation}
где $m$~---~количество термов.

Также задаётся множество значений величины, которую характеризует лингвистическая переменная:
\begin{equation}
	X = (x_1, x_2, ..., x_n)
\end{equation}
где $n$~---~количество значений.

Для каждого $i = \overline{1,m}, j = \overline{1,n}$ требуется определить значение $\mu_{t_i}(x_j)$~---~степени функции принадлежности элементов множества $X$ к элементам из множества $T_i^{'}$.

В данной работе данное значение определяется при помощи анализа статистической выборки по результатам опроса респондентов (экспертов).

Тогда степень функции принадлежности элементов множества $X$ к элементам из множества $T_i^{'}$ можно рассчитать по формуле
\begin{equation}
	\mu_{t_i}(x_j) = \frac{1}{k} \cdot \sum\limits_{k=1}^K a^{k}_{ji},
\end{equation}
где $K$~---~количество экспертов, $a_{nm}^k$~---~бинарная экспертная оценка $k$-м экспертом у элемента $x_j$ свойств нечёткого множества $T_i^{'}$, $i = \overline{1,m}$.

По полученным значениям для каждого терма строится функция принадлежности, для которой по оси абсцисс откладывается значение величин из $X$, а по оси ординат~---~значение степени функции принадлежности элементов множества $X$ к элементам из множества $T_i^{'}$.

Терм будет определять те значения из диапазона, для которых график его функции принадлежности будет расположен выше остальных.

\subsection{Алгоритм бинарного поиска в словаре}
Бинарный поиск широко используется в информатике, вычислительной математике и программировании \cite{item9}. Суть данного алгоритма~---~нахождение в массиве (в условиях данной лабораторной работы~---~в словаре) элемента по ключу. Изначальная отсортированность данных~---~важное условие работы алгоритма. 

В данной работе бинарный поиск используется последовательно два раза~---~для определения начального и конечного значений  для диапазона элементов из множества $X$ для терма.

В данном алгоритме применяется последовательное деление массива и его частей на половины на каждой итерации цикла, то есть, постепенное приближение к искомому значению. Значение считается найденным при схождении границ обзора массива, которые в начале работы алгоритма устанавливаются равными 0 и индексу последнего элемента в наборе соответственно. Вторым случаем является завершение работы алгоритма при пересечении границ обзора, что случается, когда в наборе нет искомого элемента. В контексте данной работы такая ситуация также валидна. Тогда в качестве результата выбирается большее из значений границ, если ведётся поиск начального значения для диапазона элементов множества $X$, и меньшее, если конечного.

\newpage