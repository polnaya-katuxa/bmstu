\chapter*{Заключение}
\addcontentsline{toc}{chapter}{Заключение} 
Алгоритм бинарного поиска применительно к задаче поиска эелемента в словаре по ключу достаточно эффективен, так как имеет временную сложность $O(\log{2}n)$ \cite{item11}. Однако, ограничением эффективности данного алгоритма является эффективность предварительно примененной сортировки: если выбранный алгоритм сортировки неэффективен, то в целом поиск по словарю также займёт большое количество времени, что будет неэффективно.

В ходе выполнения лабораторной работы была достигнута поставленная цель: были получены навыки программирования, тестирования полученного программного продукта и поиска по словарю при ограничении на значение признака, заданного при помощи лингвистической переменной.

В процессе выполнения лабораторной работы были также реализованы все поставленные задачи, а именно:
\begin{itemize}
	\item был произведён выбор объекта и признака объекта для разделения на категории;
	\item была выполнена формализация выбранного признака и определение категорий;
	\item была составлена анкета для респондентов;
	\item был проведён опрос среди респондентов по составленной анкете;
	\item была построена функция принадлежности термам числовых значений при- знака, описываемого лингвистической переменной, на основе проведения статистического анализа мнений респондентов;
	\item были описаны допустимые формулировки при формировании запроса;
	\item был описан алгоритм поиска в словаре сущностей, удовлетворяющих ограничению, заданному в запросе на ограниченном естественном языке, и алгоритм определения соответствия введённого запроса распознаваемым вариантам; 
	\item была описана структура данных словаря;
	\item были реализованы описанные алгоритмы;
	\item был проведён анализ применимости предложенного алгоритма.
\end{itemize}

\newpage