\chapter{Экспериментальная часть}

В данном разделе описаны проведённые замеры и представлены результаты исследования. Также будут уточнены характеристики устройства, на котором проводились замеры времени выполнения и потребляемой памяти.

\section{Технические характеристики}
Технические характеристики устройства, на котором выполнялось тестирование \cite{web_item5}:
\begin{itemize}
	\item операционная система macOS Monterey 12.4;
	\item 8 ГБ оперативной памяти;
	\item процессор Apple M2 (базовая частота~---~2400 МГц, но поддержка технологии Turbo Boost позволяет достигать частоты в 3500 МГц \cite{web_item10}).
\end{itemize}

\section{Измерение времени выполнения реализаций алгоритмов}
Тестирование реализаций алгоритмов производилось при помощи встроенных в Go средств, а именно «бенчмарков» ($benchmarks$ из пакета $testing$ стандартной библиотеки $Go$ \cite{web_item3}), представляющих собой тесты производительности. По умолчанию производятся только замеры процессорного времени выполнения в наносекундах \cite{web_item11}\cite{web_item12}, но при добавлении ключа $-benchmem$ также выполняется замер потребления памяти и количества аллокаций памяти.

Значение $N$ динамически изменяется для достижения стабильного результата при различных условиях, но гарантируется, что каждый «бенчмарк» будет выполняться хотя бы одну секунду. Для замеров использовались строки равной длины, генерирующиеся  случайным образом из строчных и прописных букв латинского алфавита перед началом выполнения «бенчмарков». Результаты тестирования возвращаются в структуре специального вида. Пример такой структуры представлен в листинге \ref{code:go_bench_struct}.

\newpage

\begin{code}
\caption{Листинг структуры результата «бенчмарка»}
\label{code:go_bench_struct}

\begin{minted}{go}
testing.BenchmarkResult{N:120000, T:1200000000, Bytes:0, MemAllocs:0x0, 
MemBytes:0x0, Extra:map[string]float64{}}
\end{minted}
\end{code}

В листинге \ref{code:go_bench} представлен пример реализации «бенчмарка», где $alg.function$~---~объект типа функция (в данной реализации~---~функция, описывающая один из алгоритмов поиска редакционного расстояния).
\begin{code}
\caption{Листинг примера реализации «бенчмарка»}
\label{code:go_bench}

\begin{minted}{go}
func NewBenchmark(s1, s2 string, alg algorithm) func(*testing.B) {
	return func(b *testing.B) {
		for j := 0; j < b.N; j++ {
			alg.function(s1, s2)
		}
	}
}
\end{minted}
\end{code}

Результаты замеров времени выполнения (в нс.) приведены в таблице \ref{table:benchmarks_time}. Сокращение Д~--~Л обозначает алгоритм вычисления расстояния Дамерау~--~Левенштейна, (м)~---~матричная реализация, (р)~---~рекурсивная, (рк)~---~рекурсивная с кэшем. В таблице для значений, для которых замеры не выполнялись, в поле результата находится «---». Замеры времени для реализации рекурсивного алгоритма без кэша выполнены для первых семи значений длин строк, полученные результаты показывают различия в характере роста затрачиваемого реализациями времени: наибольшая скорость роста наблюдается для рекурсивного алгоритма поиска расстояния Дамерау~--~Левенштейна без кэша, что подтверждает теоретическую оценку быстродействия (см. п. \ref{section:lev_matr}).

\begin{table}[H]
  \caption{\label{table:benchmarks_time} Результаты замеров времени (нс.)}
  \begin{center}
    \begin{tabular}{
    |S[table-format=4.0]
    |S[table-format=10.0]
    |S[table-format=10.0]
    |S[table-format=10.0]
    |S[table-format=10.0]|
    }
      \hline
      {Длина строк} & {Левентшейн (м)} & {Д~--~Л (м)} & {Д~--~Л (р)} & {Д~--~Л (рк)} \\ \hline
      1 & 67 & 66 & 21 & 72 \\ \hline
      2 & 102 & 107 & 66 & 126 \\ \hline
      3 & 136 & 144 & 261 & 188 \\ \hline
      5 & 244 & 268 & 6600 & 402 \\ \hline
      6 & 297 & 325 & 35055 & 545 \\ \hline
      7 & 350 & 388 & 189077 & 684 \\ \hline
      10 & 614 & 703 & 31517684 & 1319 \\ \hline
      15 & 1229 & 1463 & {---} & 2782 \\ \hline
      20 & 2162 & 2565 & {---} & 4934 \\ \hline
      30 & 5031 & 5820 & {---} & 10738 \\ \hline
      50 & 14009 & 15811 & {---} & 35395 \\ \hline
    \end{tabular}
  \end{center}
\end{table}

На рисунке \ref{img:graph_time} приведен график, отображающий зависимость времени работы реализаций алгоритмов от длин строк для всех реализаций алгоритмов, кроме рекурсивной реализации алгоритма поиска расстояния Дамерау--Левенштейна. 

\noindent
\begin{table}[h!]
  \centering
  \begin{tabular}{p{1\linewidth}}
    \centering
    \includegraphics[width=0.75\linewidth]{../data/time_new.pdf}
    \captionof{figure}{Зависимость времени работы реализаций алгоритмов вычисления редакционного расстояния от длины строк}
    \label{img:graph_time}
  \end{tabular}
\end{table}

На рисунке \ref{img:graph_time2} приведен график, отображающий зависимость времени работы реализаций алгоритмов от длин строк для всех реализаций алгоритмов при входных строках с длиной не более 10 символов. 

\noindent
\begin{table}[h!]
  \centering
  \begin{tabular}{p{1\linewidth}}
    \centering
    \includegraphics[width=0.8\linewidth]{../data/time_new2.pdf}
    \captionof{figure}{Зависимость времени работы реализаций алгоритмов вычисления редакционного расстояния от длины строк (не более 10 символов)}
    \label{img:graph_time2}
  \end{tabular}
\end{table}

\section{Измерение объёма потребляемой реализациями алгоритмов памяти}
Измерение объёма потребляемой памяти производилось посредством создания собственного программного модуля memory, использующего функции пакета unsafe языка $Go$ \cite{web_item3}. Реализация основного функционала модуля, а также пример функции расчёта памяти, затрачиваемой на алгоритм поиска расстояния Левенштейна, и пример использования указанных функций приведены в Приложении А.

Результаты замеров потребляемой памяти (в байтах) приведены в таблице \ref{table:benchmarks_mem}. Сокращение Д~--~Л обозначает алгоритм вычисления расстояния Дамерау~--~Левенштейна, (м)~---~матричная реализация, (р)~---~рекурсивная, (рк)~---~рекурсивная с кэшем. В таблице для значений, для которых тестирование не выполнялось, в поле результата находится «---». Замеры памяти для реализации рекурсивного алгоритма без кэша выполнены для первых семи значений длин строк, полученные результаты показывают различия в характере роста потребляемой реализациями памяти: наибольшая скорость роста наблюдается для рекурсивного алгоритма поиска расстояния Дамерау~--~Левенштейна с кэшем, что подтверждает теоретическую оценку потребления памяти (см. п. \ref{section:dam_lev_cache}).

\begin{table}[h]
  \caption{\label{table:benchmarks_mem} Результаты замеров потребляемой памяти (в байтах)}
  \begin{center}
    \begin{tabular}{
    |S[table-format=4.0]
    |S[table-format=10.0]
    |S[table-format=10.0]
    |S[table-format=10.0]
    |S[table-format=10.0]|
    }
      \hline
      {Длина строк} & {Левентшейн (м)} & {Д~--~Л (м)} & {Д~--~Л (р)} & {Д~--~Л (рк)} \\ \hline
      1 & 330 & 346 & 418 & 602 \\ \hline
      2 & 404 & 420 & 748 & 1044 \\ \hline
      3 & 494 & 510 & 1078 & 1502 \\ \hline
      5 & 722 & 738 & 1738 & 2466 \\ \hline
      6 & 860 & 876 & 2068 & 2972 \\ \hline
      7 & 1014 & 1030 & 2398 & 3494 \\ \hline
      10 & 1572 & 1588 & 3388 & 5156 \\ \hline
      15 & 2822 & 2838 & {---} & 8246 \\ \hline
      20 & 4472 & 4488 & {---} & 11736 \\ \hline
      30 & 8972 & 8988 & {---} & 19916 \\ \hline
      50 & 22772 & 22788 & {---} & 41076 \\ \hline
    \end{tabular}
  \end{center}
\end{table}

На рисунке \ref{img:graph_mem} приведён график, отображающий зависимость потребляемой памяти от длин строк для реализаций алгоритмов поиска расстояния Левенштейна и Дамерау~--~Левенштейна на длинах строк не более 10 символов, так как результаты замеров памяти схожи, что не позволяет увидеть различия при большем масштабе графика. На рисунке \ref{img:graph_mem2} приведён график, отображающий зависимость потребляемой памяти от длин строк для всех алгоритмов на длинах строк не более 10 символов.

\newpage

\noindent
\begin{figure}[t!]
	\centering
    \includegraphics[width=0.8\linewidth]{../data/memory_new.pdf}
    \caption{Зависимость потребляемой памяти при вычислении редакционного расстояния от длины строк}
    \label{img:graph_mem}
\end{figure}

\noindent
\begin{figure}[t!]
	\centering
    \includegraphics[width=0.8\linewidth]{../data/memory_new2.pdf}
    \caption{Зависимость потребляемой памяти при вычислении редакционного расстояния от длины строк}
    \label{img:graph_mem2}
\end{figure}

\newpage