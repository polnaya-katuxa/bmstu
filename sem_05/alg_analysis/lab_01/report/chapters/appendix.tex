\setcounter{chapter}{5}
\setcounter{listing}{0}
\chapter*{Приложение А}
\addcontentsline{toc}{chapter}{Приложение А} 
\label{appendix}

В ходе выполнения лабораторной работы в соответствии с поставленными задачами было необходимо произвести замеры потребляемой при выполнении функций, реализующих заданные алгоритмы, памяти. В связи с этим был разработан программный модуль $memory$ для измерения потребляемой функцией памяти в байтах (замеры реализованы только для функций, представляющих алгоритмы по заданию лабораторной работы). 

«Бенчмарки», использованные для измерения затрачиваемого на исполнение функции времени, предоставляют возможность измерить только память, выделяемую на куче, что не соответствует поставленной задаче. 

При вызове функции в языке $Go$ для неё выделяется область собственного стека, что особенно критично при рекурсивных реализациях. Соответственно, необходимо иметь возможность измерять потребляемую память и на стеке, так как в ином случае не будут получены реалистичные результаты замеров и построить зависимость, отражающую действительное потребление памяти в зависимости от длины строк, будет невозможно.

В листингах \ref{code:mem_module1}~---~\ref{code:mem_module3} приведена реализация основных структур и базового функционала модуля $memory$ для измерения потребляемой функцией памяти в байтах.

\begin{code}
\caption{Листинг основных структур и базового функционала модуля $memory$ для измерения потребляемой функцией памяти в байтах (начало)}
\label{code:mem_module1}

\begin{minted}{go}
package algorithms

import (
	"unsafe"
)

var MemoryInfo Metrics

type Metrics struct {
	current int
	max     int
}
\end{minted}
\end{code}

\newpage

\begin{code}
\caption{Листинг основных структур и базового функционала модуля $memory$ для измерения потребляемой функцией памяти в байтах (продолжение листинга \ref{code:mem_module1})}
\label{code:mem_module2}

\begin{minted}{go}
// сброс рассчитанных значений
func (m *Metrics) Reset() {
	m.current = 0
	m.max = 0
}

// добавление значения к общей сумме потребляемой памяти,
// обновление максимума
func (m *Metrics) Add(v int) {
	m.current += v
	if m.current > m.max {
		m.max = m.current
	}
}

// вычитание значения из общей суммы потребляемой памяти
func (m *Metrics) Done(v int) {
	m.current -= v
}

// получение значения макс. потребления памяти за выполнение функции
func (m *Metrics) Max() int64 {
	return int64(m.max)
}

// получение размера типа данных
// пример вызова: sizeOf[int]()
func sizeOf[T any]() int {
	var v T
	return int(unsafe.Sizeof(v))
}
\end{minted}
\end{code}

\begin{code}
\caption{Листинг основных структур и базового функционала модуля $memory$ для измерения потребляемой функцией памяти в байтах (окончание листинга \ref{code:mem_module2})}
\label{code:mem_module3}

\begin{minted}{go}
// получение полного размера среза (заголовок + элементы)
// пример вызова: sizeOfArray[int](10)
func sizeOfArray[T any](n int) int {
	return sizeOf[[]T]() + n*sizeOf[T]()
}
\end{minted}
\end{code}

В листингах \ref{code:mem_module_lev1}~--~\ref{code:mem_module_lev2} приведена реализация одной из функций (в данном случае функции, реализующей алгоритм поиска расстояния Левенштейна), вычисляющих потребление памяти конкретной функцией, модуля $memory$.

\begin{code}
\caption{Листинг функции, вычисляющей потребление памяти функцией, реализующей алгоритм поиска расстояния Левенштейна (начало)}
\label{code:mem_module_lev1}

\begin{minted}{go}
func memoryLevenshtein(s1, s2 string) int {
	n1 := len(s1)
	n2 := len(s2)

	s01 := sizeOf[string]() + n1*sizeOf[byte]()
	s02 := sizeOf[string]() + n2*sizeOf[byte]()

	n1r := len([]rune(s1))
	n2r := len([]rune(s2))

	sr01 := sizeOfArray[rune](n1r)
	sr02 := sizeOfArray[rune](n2r)

	res := sizeOf[int]()

	loop1 := 2 * sizeOf[int]()

	m := sizeOf[[][]int]() + sizeOfArray[int](n2r+1)*(n1r+1)
	
	loop2 := 5 * sizeOf[int]()
\end{minted}
\end{code}

\begin{code}
\caption{Листинг функции, вычисляющей потребление памяти функцией, реализующей алгоритм поиска расстояния Левенштейна (окончание листинга \ref{code:mem_module_lev1})}
\label{code:mem_module_lev2}

\begin{minted}{go}
	minFunc := sizeOfArray[int](3) + sizeOf[int]()*3

	return s01 + s02 + sr01 + sr02 + res + m + loop1 + loop2 + minFunc
}
\end{minted}
\end{code}

В листинге \ref{code:mem_module_use} приведен пример использования функций модуля $memory$ в реализации алгоритма поиска расстояния Левенштейна.

\begin{code}
\caption{Листинг использования функций модуля $memory$ в реализации алгоритма поиска расстояния Левенштейна}
\label{code:mem_module_use}

\begin{minted}{go}
func Levenshtein(s1, s2 string) int {
	MemoryInfo.Reset()
	MemoryInfo.Add(memoryLevenshtein(s1, s2))
	defer MemoryInfo.Done(memoryLevenshtein(s1, s2))
	...
}
\end{minted}
\end{code}

Таким образом, данный модуль позволяет измерить полное потребление памяти функциями, реализующими алгоритмы поиска редакционного расстояния, и получить реалистичные данные по результатам измерений.


