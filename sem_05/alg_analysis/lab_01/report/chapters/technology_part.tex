\chapter{Технологическая часть}

В данном разделе будет представлена реализация алгоритмов поиска редакционного расстояния. Также будут указаны обязательные требования к ПО, средства реализации алгоритмов и результаты проведённого тестирования программы.

\section{Требования к ПО}
Для программы выделен перечень требований:
\begin{itemize}
	\item предоставляется интерфейс в формате меню с возможностью ввода и изменения обрабатываемых строк, завершения работы с программой и выбора используемого для обработки введённых строк алгоритма поиска редакционного расстояния;
	\item предлагается повторно выполнить ввод при невалидном выборе пункта меню;
	\item производится аварийное завершение с текстом об ошибке при иных ошибках;
	\item проводится модульное тестирование функций реализующих поиск редакционного расстояния;
	\item производятся замеры времени выполнения и потребления памяти функциями поиска редакционного расстояния;
	\item допускается ввод строк в любой раскладке.
\end{itemize}

\section{Средства реализации}
Для реализации данной работы был выбран язык программирования Go \cite{web_item2}. Выбор обусловлен наличием в $Go$ библиотек для тестирования ПО и проведения замеров времени выполнения в наносекундах, а также необходимых для реализации поставленных цели и задач средств. В качестве среды разработки была выбрана $GoLand$ \cite{web_item4}.

\section{Реализация алгоритмов}
В листингах \ref{code:go_lev}~--~\ref{code:go_rec_dam_lev_cache3} представлены различные реализации алгоритмов нахождения расстояний Левенштейна и Дамерау~--~Левенштейна.

\newpage

\begin{code}
\caption{Листинг матричной реализации алгоритма нахождения расстояния Левенштейна}
\label{code:go_lev}

\begin{minted}{go}
func Levenshtein(s1, s2 string) int {
	s1Rune := []rune(s1)
	s2Rune := []rune(s2)
	m := make([][]int, len(s1Rune)+1)
	for i := range m {
		m[i] = make([]int, len(s2Rune)+1)
		m[i][0] = i
	}
	for j := range m[0] {
		m[0][j] = j
	}
	
	for i := 1; i < len(m); i++ {
		for j := 1; j < len(m[i]); j++ {
			insertOpt := m[i][j-1] + 1
			deleteOpt := m[i-1][j] + 1
			replaceOpt := m[i-1][j-1]
			
			if s1Rune[i-1] != s2Rune[j-1] {
				replaceOpt += 1
			}
			m[i][j] = min(insertOpt, deleteOpt, replaceOpt)
		}
	}
	if Print {
		PrintMatrix(m)
	}

	return m[len(m)-1][len(m[0])-1]
}
\end{minted}
\end{code}

\begin{code}
\caption{Листинг матричной реализации алгоритма нахождения расстояния Дамерау~--~Левенштейна (начало)}
\label{code:go_dam_lev}

\begin{minted}{go}
func DamerauLevenshtein(s1, s2 string) int {
	s1Rune := []rune(s1)
	s2Rune := []rune(s2)

	m := make([][]int, len(s1Rune)+1)
	for i := range m {
		m[i] = make([]int, len(s2Rune)+1)
		m[i][0] = i
	}
	for j := range m[0] {
		m[0][j] = j
	}

	for i := 1; i < len(m); i++ {
		for j := 1; j < len(m[i]); j++ {
			insertOpt := m[i][j-1] + 1
			deleteOpt := m[i-1][j] + 1
			replaceOpt := m[i-1][j-1]
			substituteOpt := math.MaxInt

			if s1Rune[i-1] != s2Rune[j-1] {
				replaceOpt += 1
			}

			if i > 1 && j > 1 {
				if s1Rune[i-1] == s2Rune[j-2] && 
				s1Rune[i-2] == s2Rune[j-1] {
					substituteOpt = m[i-2][j-2] + 1
				}
			}

			
\end{minted}
\end{code}

\begin{code}
\caption{Листинг матричной реализации алгоритма нахождения расстояния Дамерау~--~Левенштейна (окончание листинга \ref{code:go_dam_lev})}
\label{code:go_dam_lev2}

\begin{minted}{go}
			m[i][j] = min(insertOpt, deleteOpt, replaceOpt, 
			substituteOpt)
		}
	}

	if Print {
		PrintMatrix(m)
	}

	return m[len(m)-1][len(m[0])-1]
}
			
\end{minted}
\end{code}


\begin{code}
\caption{Листинг рекурсивной реализации алгоритма нахождения расстояния Дамерау~--~Левенштейна (начало)}
\label{code:go_rec_dam_lev}

\begin{minted}{go}
func RecursiveDamerauLevenshtein(s1, s2 string) int {
	s1Rune := []rune(s1)
	s2Rune := []rune(s2)

	return recursiveDamerauLevenshtein(s1Rune, s2Rune)
}

func recursiveDamerauLevenshtein(s1, s2 []rune) int {

	if len(s1) == 0 && len(s2) == 0 {
		return 0
	} else if len(s1) > 0 && len(s2) == 0 {
		return len(s1)
	} else if len(s1) == 0 && len(s2) > 0 {
		return len(s2)
	} else {
		e := 0
\end{minted}
\end{code}

\begin{code}
\caption{Листинг рекурсивной реализации алгоритма нахождения расстояния Дамерау~--~Левенштейна (окончание листинга \ref{code:go_rec_dam_lev})}
\label{code:go_rec_dam_lev2}

\begin{minted}{go}
		if s1[len(s1)-1] != s2[len(s2)-1] {
			e += 1
		}

		if len(s1) > 1 && len(s2) > 1 &&
			(s1[len(s1)-1] == s2[len(s2)-2] && 
			s1[len(s1)-2] == s2[len(s2)-1]) {
			return min(recursiveDamerauLevenshtein(s1, 
			s2[:len(s2)-1])+1,
				recursiveDamerauLevenshtein(s1[:len(s1)-1], 
				s2)+1,
				recursiveDamerauLevenshtein(s1[:len(s1)-1], 
				s2[:len(s2)-1])+e,
				recursiveDamerauLevenshtein(s1[:len(s1)-2], 
				s2[:len(s2)-2])+1)
		} else {
			return min(recursiveDamerauLevenshtein(s1, 
			s2[:len(s2)-1])+1,
				recursiveDamerauLevenshtein(s1[:len(s1)-1], 
				s2)+1,
				recursiveDamerauLevenshtein(s1[:len(s1)-1], 
				s2[:len(s2)-1])+e)
		}
	}
}
\end{minted}
\end{code}

\newpage

\begin{code}
\caption{Листинг рекурсивной реализации алгоритма нахождения расстояния Дамерау~--~Левенштейна с кэшем (начало)}
\label{code:go_rec_dam_lev_cache}

\begin{minted}{go}
func RecursiveDamerauLevenshteinCached(s1, s2 string) int {
	s1Rune := []rune(s1)
	s2Rune := []rune(s2)

	cache := make([][]int, len(s1)+1)
	for i := range cache {
		cache[i] = make([]int, len(s2)+1)
		cache[i][0] = i
	}
	for j := range cache[0] {
		cache[0][j] = j
	}

	for i := 1; i < len(cache); i++ {
		for j := 1; j < len(cache[i]); j++ {
			cache[i][j] = math.MaxInt
		}
	}

	res := recursiveDamerauLevenshteinCached(s1Rune, s2Rune, cache)

	if Print {
		PrintMatrix(cache)
	}

	return res
}

func recursiveDamerauLevenshteinCached(s1, s2 []rune, cache [][]int) int {
	if cache[len(s1)][len(s2)] == math.MaxInt {
		if len(s1) == 0 && len(s2) == 0 {
\end{minted}
\end{code}

\begin{code}
\caption{Листинг рекурсивной реализации алгоритма нахождения расстояния Дамерау~--~Левенштейна с кэшем (продолжение листинга \ref{code:go_rec_dam_lev_cache})}
\label{code:go_rec_dam_lev_cache2}

\begin{minted}{go}
			cache[len(s1)][len(s2)] = 0
		} else if len(s1) > 0 && len(s2) == 0 {
			cache[len(s1)][len(s2)] = len(s1)
		} else if len(s1) == 0 && len(s2) > 0 {
			cache[len(s1)][len(s2)] = len(s2)
		} else {
			e := 0
			if s1[len(s1)-1] != s2[len(s2)-1] {
				e += 1
			}
			if len(s1) > 1 && len(s2) > 1 &&
				(s1[len(s1)-1] == s2[len(s2)-2] && 
				s1[len(s1)-2] == s2[len(s2)-1]) {
				cache[len(s1)][len(s2)] = min(
				    recursiveDamerauLevenshteinCached(
				    s1, s2[:len(s2)-1], cache)+1,
					recursiveDamerauLevenshteinCached(
					s1[:len(s1)-1], s2, cache)+1,
					recursiveDamerauLevenshteinCached(
					s1[:len(s1)-1], s2[:len(s2)-1], 
					cache)+e,
					recursiveDamerauLevenshteinCached(
					s1[:len(s1)-2], s2[:len(s2)-2], 
					cache)+1)
			} else {
				cache[len(s1)][len(s2)] = min(
				    recursiveDamerauLevenshteinCached(
				    s1, s2[:len(s2)-1], cache)+1,
					recursiveDamerauLevenshteinCached(
					s1[:len(s1)-1], s2, cache)+1,
					recursiveDamerauLevenshteinCached(
					s1[:len(s1)-1], s2[:len(s2)-1], 
					cache)+e)
\end{minted}
\end{code}

\begin{code}
\caption{Листинг рекурсивной реализации алгоритма нахождения расстояния Дамерау~--~Левенштейна с кэшем (окончание листинга \ref{code:go_rec_dam_lev_cache2})}
\label{code:go_rec_dam_lev_cache3}

\begin{minted}{go}
			}
		}
	}
	return cache[len(s1)][len(s2)]
}
\end{minted}
\end{code}

\section{Тестирование}
В таблице  представлены тесты для алгоритмов нахождения расстояний Левенштейна и Дамерау~--~Левенштейна. Тестирование проводилось по методологии чёрного ящика. Все тесты пройдены успешно.

\begin{table}[H]
  \caption{\label{table:tests} Тесты для алгоритмов нахождения расстояний Левенштейна и Дамерау~--~Левенштейна}
  \begin{center}
    \begin{tabular}{|c|c|c|c|c|}
      \hline
       &  &  & \multicolumn{2}{c|}{Ожидаемый результат} \\
      \cline{4-5}
      \raisebox{1.5ex}[0cm][0cm]{№} & \raisebox{1.5ex}[0cm][0cm]{$s_1$} & \raisebox{1.5ex}[0cm][0cm]{$s_2$} 
      &  Левентшейн & Дамерау~--~Левенштейн \\ \hline
      1 & $\varnothing$ & $\varnothing$ & 0 & 0 \\ \hline
      2 & abc & aba & 1 & 1 \\ \hline
      3 & text & tetx & 2 & 1 \\ \hline
      4 & скат & кот & 2 & 2 \\ \hline
      5 & $\varnothing$ & ababab & 6 & 6 \\ \hline
      6 & acaca & $\varnothing$ & 5 & 5 \\ \hline
      7 & $\varnothing$ & пюрешка & 7 & 7 \\ \hline
      8 & сосиска & $\varnothing$ & 7 & 7 \\ \hline
      9 & кеотон & кетоно & 2 & 2 \\ \hline
      10 & мышь & мiшь & 1 & 1 \\ \hline
      11 & кушнявка & укшняква & 4 & 2 \\ \hline
      12 & привет & рпвите & 4 & 3 \\ \hline
    \end{tabular}
  \end{center}
\end{table}

\newpage