\setcounter{page}{3}
\chapter*{Введение}
\addcontentsline{toc}{chapter}{Введение} 

Расстояние Левенштейна (редакционное расстояние)~---~это минимальное количество операций вставки одного символа, удаления одного символа и замены одного символа на другой (редакционных операций), необходимых для превращения одной последовательности символов в другую.

Редакционные операции можно обозначить следующим образом:
\begin{itemize}
	\item $i$ (insert)~---~вставка одного символа;
	\item $d$ (delete)~---~удаление одного символа;
	\item $r$ (replace)~---~замена одного символа на другой.
\end{itemize}

Расстояние Дамерау~--~Левенштейна~---~это минимальное количество операций вставки одного символа, удаления одного символа, замены одного символа на другой и транспозиции двух соседних символов (редакционных операций), необходимых для превращения одной последовательности символов в другую. Можно назвать модифицированным расстоянием Левенштейна.

Таким образом, вводится одно обозначение в дополнение к перечисленным выше: \\
$t$ (transposition)~---~транспозиция двух соседних символов.\\

Расстояние Левенштейна и расстояние Дамерау~--~Левенштейна находят применение в:
\begin{itemize}
	\item компьютерной лингвистике;
	\item биоинформатике (сравнение генов);
	\item базах данных \cite{web_item1};
	\item распознавании рукописных символов \cite{web_item1}.
\end{itemize}

\newpage

Цель работы: получение навыков программирования, тестирования полученного программного продукта и проведения замеров времени выполнения и потребляемой памяти по результатам работы программы на примере решения задачи о редакционном расстоянии.\\

Задачи работы:
\begin{enumerate}[label={\arabic*)}]
	\item изучение расстояний Левенштейна и Дамерау~--~Левенштейна;
	\item изучение алгоритма вычисления расстояния Левенштейна, Дамерау~--~Левенштейна, рекурсивного алгоритма вычисления расстояния Дамерау~--~Левенштейна и модификации последнего с использованием кэша;
	\item разработка нерекурсивных алгоритмов поиска расстояний Левенштейна и  Дамерау~--~Левенштейна, рекурсивного алгоритм поиска расстояния Дамерау~--~Левенштейна и модификации последнего с использованием кэша;
	\item реализация данных алгоритмов;
	\item проведение замеров потребления памяти (в байтах) для данных алгоритмов; 
	\item проведение замеров времени работы (в нс) данных алгоритмов; 
	\item получение графической зависимости замеряемых величин от длины последовательности символов, предоставляемой на вход алгоритмам;
	\item проведение сравнительного анализа данных алгоритмов на основе полученных зависимостей.
\end{enumerate}

\newpage