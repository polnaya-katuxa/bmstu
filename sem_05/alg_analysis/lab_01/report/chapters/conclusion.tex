\chapter*{Заключение}
\addcontentsline{toc}{chapter}{Заключение} 
В результате проведения замеров времени выполнения и потребляемой памяти были сформулированы нижеперечисленные выводы.

Если сравнивать алгоритмы поиска редакционного расстояния по времени выполнения, то очевидно, что рекурсивный алгоритм занимает много больше времени, чем реализации, использующие матрицу результатов. Уже при длине строк равной 6 рекурсивный алгоритм поиска расстояния Дамерау~--~Левенштейна работает примерно в 107 раз дольше, чем аналогичный матричный. При длине строк, равной 10, он работает дольше в 44833 раза. 

Если сравнивать две матричные реализации алгоритмов~---~поиска расстояния Левенштейна и Дамерау~--~Левенштейна~---~время работы второго будет превосходить время работы первого, что связано с наличием дополнительных проверок. 

Если сравнивать две рекурсивные реализации алгоритма поиска расстояния Дамерау~--~Левенштейна, время работы алгоритма без кэша будет превосходить время работы алгоритма с кэшем, что связано с необходимостью повторных вычислений при работе в отсутствие матрицы промежуточных результатов.

Если сравнивать реализации алгоритмов по объёму потребляемой памяти, то рекурсивные алгоритмы уступают матричным при любых длинах строк в связи с тем, что при вызове функции выделяется область собственного стека для нее. Так как в рекурсивных методах много вызовов функций, то потребляемая память выходит больше, чем в матричных, в которых стек выделяется на один вызов функции. Так, при длине строк в 10 символов рекурсивная реализация алгоритма поиска расстояния Дамерау~--~Левенштейна потребляет в 2,1 раза больше байт памяти, чем матричная реализация.

Алгоритм с кэшированием потребляет в среднем в 1,5 раза больше памяти, чем аналогичный алгоритм без кэширования, так как требуется дополнительная память для хранения матрицы.

Потребление памяти матричными реализациями алгоритмов сравнимо, так как разницу составляют только единичные переменные (например, дополнительная переменная с ценой операции транспозиции в алгоритме поиска расстояния Дамерау~--~Левенштейна на текущем шаге, которой нет в алгоритме поиска расстояния Левенштейна, в связи с отсутствием в нём данной операции).

\newpage

В ходе выполнения лабораторной работы была достигнута поставленная цель: были получены навыки программирования, тестирования полученного программного продукта и проведения замеров по результатам работы программы на примере решения задачи о редакционном расстоянии.

В процессе выполнения лабораторной работы были также реализованы все поставленные задачи, а именно:
\begin{itemize}
	\item были изучены расстояния Левенштейна и Дамерау~--~Левенштейна;
	\item были разработаны нерекурсивные алгоритмы поиска расстояний Левенштейна и Дамерау~--~Левенштейна, рекурсивный алгоритм поиска расстояния Дамерау~--~Левенштейна и модификация последнего с использованием кэша;
	\item была выполнена программная реализация данных алгоритмов;
	\item были проведены замеры потребления памяти (в байтах) и времени работы (в нс) для данных алгоритмов;
	\item была получена графическая зависимость измеренных величин от длины последовательности символов, предоставляемой на вход алгоритмам;
	\item был проведен сравнительный анализ данных алгоритмов на основе полученных зависимостей.
\end{itemize}

\newpage