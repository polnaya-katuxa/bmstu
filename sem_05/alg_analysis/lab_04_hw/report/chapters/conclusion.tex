\chapter*{Заключение}
\addcontentsline{toc}{chapter}{Заключение} 
В результате проведения замеров были сформулированы нижеперечисленные выводы.

Наибольшую эффективность работы алгоритм достигает при выделении 8 потоков, так как это соответствует числу логических ядер процессора устройства, на котором проводились замеры (8 ядер). Если потоков выделялось больше, в большинстве случаев алгоритм работал медленнее, чем на 8 ядрах. Например, при 32 потоках при размере изображения $512 \times 512$ алгоритм работал в 1.17 раз медленнее, чем при 18 потоках. Также все алгоритмы работают дольше при выделении 1 потока, чем при отсутствии  вспомогательных потоков, так как требуется время на выделение такового. В среднем разница составляла до 10 мс.

При увеличении размера изображения для рендера время работы алгоритма росло. Например, на 4 потоках при изображении размером $128 \times 128$ алгоритм работал в 8,6 раз быстрее, чем на размере изображения $384 \times 384$.

В среднем для любого размера изображения алгоритма работал на 8 потоках в 5~--~6 раз быстрее, чем при однопоточной реализации. Время работы реализации алгоритма постепенно снижалось при повышении числа потоков до 8.

Таким образом, рекомендуется использовать число потоков, равное числу логических процессоров (ядер).

В ходе выполнения лабораторной работы была достигнута поставленная цель: были получены навыки реализации многопоточности для программного продукта, было проведено сравнение быстродействия для однопоточной и многопоточной реализаций алгоритма обратной трассировки лучей.\\

В процессе выполнения лабораторной работы были также реализованы все поставленные задачи, а именно:
\begin{enumerate}[label={\arabic*)}]
	\item был изучен алгоритм обратной трассировки лучей;
	\item были разработаны последовательная и параллельная версии алгоритма обратной трассировки лучей;
	\item была проведена разработка схем данных алгоритмов;
	\item была выполнена программная реализация данного алгоритма с использованием одного потока;
	\item была выполнена программная реализация данного алгоритма с использованием вспомогательных потоков;
	\item был проведён замер времени работы (в мс) данных реализаций алгоритма; 
	\item была получена зависимость замеряемых величин от количества потоков;
	\item был проведён сравнительный анализ данных реализаций алгоритма на основе полученных зависимостей.
\end{enumerate}

\newpage