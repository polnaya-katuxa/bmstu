\chapter{Технологическая часть}

В данном разделе будет представлена реализация алгоритмов умножения матриц. Также будут указаны обязательные требования к ПО, средства реализации алгоритмов и результаты проведённого тестирования программы.

\section{Требования к ПО}
Для программы выделен перечень требований:
\begin{itemize}
	\item предоставляется интерфейс в формате меню с возможностью ввода и изменения обрабатываемых матриц, завершения работы с программой и выбора используемого для умножения введённых матриц алгоритма;
	\item предлагается повторно выполнить ввод при невалидном выборе пункта меню;
	\item производится аварийное завершение с текстом об ошибке при иных ошибках;
	\item проводится модульное тестирование функций умножения матриц;
	\item производятся замеры времени выполнения и потребления памяти функциями умножения матриц;
	\item производятся расчёты только для матриц, содержащих целые числа.
\end{itemize}

\section{Средства реализации}
Для реализации данной работы был выбран язык программирования Go \cite{web_item2}. Выбор обусловлен наличием в $Go$ библиотек для тестирования ПО и проведения замеров времени выполнения и потребления памяти функциями, а также необходимых для реализации поставленных цели и задач средств. В качестве среды разработки была выбрана $GoLand$ \cite{web_item4}.

\section{Реализация алгоритмов}
В листингах \ref{code:stand}~--~\ref{code:winBetter2} представлены реализации алгоритмов умножения матриц: стандартного, Винограда и оптимизированного алгоритма Винограда, и некоторых нужных подпрограмм в листингах \ref{code:rc}~--~\ref{code:ccBetter}. 

\newpage

\begin{code}
\caption{Листинг функции реализации стандартного умножения матриц}
\label{code:stand}
\begin{minted}{go}
func Mul(m1, m2 Matrix) (Matrix, error) {
	if m1.N != m2.M {
		return Matrix{}, errors.New("mul sizes error")
	}

	res := CreateEmpty(m1.M, m2.N)

	for i := 0; i < res.M; i++ {
		for j := 0; j < res.N; j++ {
			for k := 0; k < m1.N; k++ {
				res.Data[i][j] = res.Data[i][j] + 
				m1.Data[i][k]*m2.Data[k][j]
			}
		}
	}

	return res, nil
}
\end{minted}
\end{code}

\begin{code}
\caption{Листинг функции реализации алгоритма Винограда умножения матриц}
\label{code:win}
\begin{minted}{go}
func WinogradMulMatrix(m1 matrix.Matrix, m2 matrix.Matrix) 
(matrix.Matrix, error) {
	if m1.N != m2.M {
		return matrix.Matrix{}, errors.New("mul sizes error")
	}

	res := matrix.CreateEmpty(m1.M, m2.N)

	rowCoefs := rowCoefs(m1)
	columnCoefs := columnCoefs(m2)
\end{minted}
\end{code}

\begin{code}
\caption{Листинг функции реализации алгоритма Винограда умножения матриц (продолжение листинга \ref{code:win})}
\label{code:win2}
\begin{minted}{go}
	for i := 0; i < res.M; i++ {
		for j := 0; j < res.N; j++ {
			res.Data[i][j] = res.Data[i][j] - rowCoefs[i] -
			 columnCoefs[j]

			for k := 0; k < m1.N/2; k++ {
				res.Data[i][j] = res.Data[i][j] + 
				(m1.Data[i][2*k]+m2.Data[2*k+1][j])*
					(m1.Data[i][2*k+1]+m2.Data[2*k][j])
			}
		}
	}

	if m1.N%2 != 0 {
		for i := 0; i < res.M; i++ {
			for j := 0; j < res.N; j++ {
				res.Data[i][j] = res.Data[i][j] + 
				m1.Data[i][m1.N-1]*m2.Data[m1.N-1][j]
			}
		}
	}

	return res, nil
}
\end{minted}
\end{code}

\begin{code}
\caption{Листинг функции реализации оптимизированного алгоритма Винограда умножения матриц}
\label{code:winBetter}
\begin{minted}{go}
func WinogradBetterMulMatrix(m1 matrix.Matrix, m2 matrix.Matrix) 
(matrix.Matrix, error) {
	if m1.N != m2.M {
		return matrix.Matrix{}, errors.New("mul sizes error")
	}
	res := matrix.CreateEmpty(m1.M, m2.N)
	rowCoefs := betterRowCoefs(m1)
	columnCoefs := betterColumnCoefs(m2)
\end{minted}
\end{code}

\begin{code}
\caption{Листинг функции реализации оптимизированного алгоритма Винограда умножения матриц (продолжение листинга \ref{code:winBetter})}
\label{code:winBetter2}
\begin{minted}{go}
	isOdd := m1.N%2 != 0
	halfN := m1.N >> 1
	preN := m1.N - 1

	for i := 0; i < res.M; i++ {
		for j := 0; j < res.N; j++ {
			res.Data[i][j] -= rowCoefs[i] + columnCoefs[j]

			for k := 0; k < halfN; k++ {
				res.Data[i][j] += (m1.Data[i][k<<1] + 
				m2.Data[(k<<1)+1][j])*
				(m1.Data[i][(k<<1)+1] + m2.Data[k<<1][j])
			}

			if isOdd {
				res.Data[i][j] += m1.Data[i][preN] *
				m2.Data[preN][j]
			}
		}
	}

	return res, nil
}
\end{minted}
\end{code}

\newpage

\begin{code}
\caption{Листинг функции реализации алгоритма поиска произведений соседних элементов строк матрицы}
\label{code:rc}
\begin{minted}{go}
func rowCoefs(m matrix.Matrix) []int {
	coefs := make([]int, m.M)
	for i := 0; i < m.M; i++ {
		for j := 0; j < m.N/2; j++ {
			coefs[i] = coefs[i] + m.Data[i][2*j]*m.Data[i][2*j+1]
		}
	}
	return coefs
}
\end{minted}
\end{code}

\begin{code}
\caption{Листинг функции реализации алгоритма поиска произведений соседних элементов столбцов матрицы}
\label{code:cc}
\begin{minted}{go}
func columnCoefs(m matrix.Matrix) []int {
	coefs := make([]int, m.N)
	for j := 0; j < m.N; j++ {
		for i := 0; i < m.M/2; i++ {
			coefs[j] = coefs[j] + m.Data[2*i][j]*m.Data[2*i+1][j]
		}
	}
	return coefs
}
\end{minted}
\end{code}

\newpage

\begin{code}
\caption{Листинг функции реализации оптимизированного алгоритма поиска произведений соседних элементов строк матрицы}
\label{code:rcBetter}
\begin{minted}{go}
func betterRowCoefs(m matrix.Matrix) []int {
	coefs := make([]int, m.M)
	halfN := m.N >> 1
	for i := 0; i < m.M; i++ {
		for j := 0; j < halfN; j++ {
			coefs[i] += m.Data[i][j<<1] * m.Data[i][(j<<1)+1]
		}
	}
	return coefs
}
\end{minted}
\end{code}

\begin{code}
\caption{Листинг функции реализации оптимизированного алгоритма поиска произведений соседних элементов столбцов матрицы}
\label{code:ccBetter}
\begin{minted}{go}
func betterColumnCoefs(m matrix.Matrix) []int {
	coefs := make([]int, m.N)
	halfM := m.M >> 1
	for j := 0; j < m.N; j++ {
		for i := 0; i < halfM; i++ {
			coefs[j] += m.Data[i<<1][j] * m.Data[(i<<1)+1][j]
		}
	}
	return coefs
}
\end{minted}
\end{code}

\newpage

\section{Тестирование}
В таблице  представлены тесты для алгоритмов умножения матриц: стандартного, Винограда и оптимизированного алгоритма Винограда. Тестирование проводилось по методологии чёрного ящика. Все тесты пройдены успешно.

\begin{table}[H]
	\begin{center}
		\caption{Тесты для алгоритмов умножения матриц: стандартного, Винограда и оптимизированного алгоритма Винограда}
		\label{table:tests}
		\begin{tabular}{|c | c | c| c |} 
			\hline
			\multicolumn{1}{|c|}{№} & \multicolumn{1}{c|}{Матрица A}
			&  \multicolumn{1}{c|}{Матрица B}
			& \multicolumn{1}{c|}{Результат} \\
			\hline
		
			
			1 & $\begin{pmatrix}
				1\\
			\end{pmatrix}$ & $\begin{pmatrix}
				1\\
			\end{pmatrix}$ & $\begin{pmatrix}
				1\\
			\end{pmatrix}$ \\ \hline
			
			2 & $\begin{pmatrix}
				1\\
			\end{pmatrix}$ & $\begin{pmatrix}
				0\\
			\end{pmatrix}$ & $\begin{pmatrix}
				0\\
			\end{pmatrix}$ \\ \hline
			
			3 & $\begin{pmatrix}
				0 & 0 & 0\\
				0 & 0 & 0\\
			\end{pmatrix}$ & $\begin{pmatrix}
				0& 0\\
				0& 0\\
				0& 0\\
			\end{pmatrix}$ &$\begin{pmatrix}
			0& 0\\
			0& 0\\
			\end{pmatrix}$ \\ \hline
			
			4 & $\begin{pmatrix}
				0 & 3 & 0\\
				0 & 0 & 4 \\
			\end{pmatrix}$ & $\begin{pmatrix}
				0& 5\\
				0& 0\\
			\end{pmatrix}$ & mul sizes error \\ \hline
			
			5 & $\begin{pmatrix}
				1 & 2 \\
				3 &4\\
				5& 6\\
				7 & 8 \\
			\end{pmatrix}$ & $\begin{pmatrix}
				1& 2\\
				3& 4\\
			\end{pmatrix}$ &$\begin{pmatrix}
				7&10\\
				15&22\\
				23&34\\
				31&46\\
			\end{pmatrix}$ \\ \hline
		
			6 & $\begin{pmatrix}
				1 & 2& 3\\
				4& 5&6 \\
				7&8&9\\
			\end{pmatrix}$ & $\begin{pmatrix}
				1\\
				2\\
				3\\
			\end{pmatrix}$ &$\begin{pmatrix}
				14\\
				32\\
				50\\
			\end{pmatrix}$ \\ \hline
		
			7 & $\begin{pmatrix}
				5& 0& 0& 1\\
				9& 7& 0& 8\\
				3& 6& 4& 7\\
				0& 0& 0& 0\\
				2& 2& 8& 2\\
			\end{pmatrix}$ & $\begin{pmatrix}
				0& 7& 5\\
				0& 0& 0\\
				0& 9& 5\\
				0& 1& 0\\
			\end{pmatrix}$ &$\begin{pmatrix}
				0& 36& 25\\
				0& 71& 45\\
				0& 64& 35\\
				0& 0& 0\\
				0& 88& 50\\
			\end{pmatrix}$ \\ \hline

			
		\end{tabular}
	\end{center}
\end{table}

\newpage