\chapter*{Заключение}
\addcontentsline{toc}{chapter}{Заключение} 
Для каждого алгоритма, кроме стандартного, время работы и объём потребляемой памяти для которого зависит только от размерности обрабатываемой матрицы, были выявлены лучший и худший случаи (данные, при которых алгоритмы работают быстрее всего и медленнее всего, соответственно). 

Для алгоритма Винограда лучшим случаем считается тот, при котором на вход подаются матрицы с нечётной размерностью, так как это влечёт за собой выполнение дополнительного цикла для коррекции результата, полученного после выполнения основного цикла.

В результате проведения замеров времени выполнения и потребляемой памяти с учётом разных случаев были сформулированы нижеперечисленные выводы.

Для любых случаев время работы стандартного алгоритма превосходит время работы алгоритмов Винограда. Например, для лучших случаев алгоритмов Винограда, стандартный на размерности матриц в 10 элементов работает медленнее в 1.2 раза, а на размерности 500 в 1,7 раза. Для худших случаев при тех же условиях разница составляет 1,3 и 1,7 раза.

В лучших случаях скорость работы алгоритма Винограда превышает скорость работы этого же алгоритма для худших случаев: для размерности 10 в 1.95 раза, для размерности 500 в 1.003 раза. Для оптимизированного алгоритма Винограда такая разница составляет 1.009 раза для размерности 10 и 1.004 раза для размерности 500.

В лучших случаях скорость работы оптимизированного алгоритма Винограда превышает скорость работы неоптимизированного алгоритма в 1.17 раза для размерности 10 и в 1.02 раза для размерности 500. Для худших случаев такая разница составляет 1.08 раза для размерности 10 и 1.03 раза для размерности 500.

Если говорить об объёме занимаемой памяти, то стандартный алгоритм выигрывает по сравнению с реализациями алгоритма Винограда только на размерностях матриц до 10 элементов. Например, при размерности 2 стандартный алгоритм потребляет примерно в 1.2 раза меньше памяти, чем реализации алгоритма Винограда. На размерности в 50 элементов стандартный алгоритм потребляет в 1.9 раза больше памяти, чем реализации алгоритма Винограда. Разница между двумя реализациями незначительна и составляет в среднем 1.01 раза в пользу неоптимизированного алгоритма.

В ходе выполнения лабораторной работы была достигнута поставленная цель: были получены навыки программирования, тестирования полученного программного продукта и проведения замеров по результатам работы программы на примере реализации алгоритмов умножения матриц.

В процессе выполнения лабораторной работы были также реализованы все поставленные задачи, а именно:
\begin{itemize}
	\item были изучены алгоритмы стандартного умножения матриц, Винограда и оптимизированный алгоритм Винограда;
	\item были разработаны схемы данных алгоритмов и проведён анализ их трудоёмкости;
	\item была выполнена программная реализация данных алгоритмов;
	\item были проведены замеры потребления памяти (в байтах) для данных алгоритмов; 
	\item были проведены замеры времени работы (в нс) данных алгоритмов; 
	\item была получены графическая зависимость замеряемых величин от размерности квадратной матрицы предоставляемой на вход алгоритмам;
	\item был проведён сравнительный анализ данных алгоритмов на основе полученных зависимостей.
\end{itemize}

\newpage