\setcounter{page}{3}
\chapter*{Введение}
\addcontentsline{toc}{chapter}{Введение} 
Матрица~---~это математический объект, представляемый в виде прямоугольной таблицы элементов (например, целых чисел), имеющей определённое количество строк и столбцов, на пересечении которых находятся его элементы. Количество строк и столбцов задает размерность матрицы.

Для матрицы определены нижеперечисленные алгебраические операции:
\begin{itemize}
	\item умножение матриц;
	\item сложение матриц;
	\item умножение матрицы на скаляр.
\end{itemize}

Матричное умножение считается одной из самых фундаментальных операций в современных вычислениях \cite{web_item1}. Например, оно широко применяется в:
\begin{itemize}
	\item задачах линейной алгебры;
	\item задачах полилинейной алгебры;
	\item задачах полиномиальной алгебры;
	\item решении обыкновенных дифференциальных уравнений;
	\item решении уравнений в частных производных;
	\item решении интегральных уравнений;
	\item комбинаторике;
	\item статистике;
	\item биоинформатике.
\end{itemize}

Цель работы: получение навыков программирования, тестирования полученного программного продукта и проведения замеров времени выполнения и объёма потребляемой памяти по результатам работы программы на примере реализации алгоритмов умножения матриц.

\newpage

Задачи работы:
\begin{enumerate}[label={\arabic*)}]
	\item изучение алгоритмов стандартного умножения матриц, Винограда и оптимизированного алгоритма Винограда;
	\item разработка схем данных алгоритмов и анализ их трудоёмкости;
	\item реализация данных алгоритмов;
	\item проведение замеров потребления памяти (в байтах) для данных алгоритмов; 
	\item проведение замеров времени работы (в нс) данных алгоритмов; 
	\item получение графической зависимости замеряемых величин от размерности квадратной матрицы предоставляемой на вход алгоритмам;
	\item проведение сравнительного анализа данных алгоритмов на основе полученных зависимостей.
\end{enumerate}

\newpage