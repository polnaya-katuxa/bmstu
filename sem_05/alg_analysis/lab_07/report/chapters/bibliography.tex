\renewcommand{\bibname}{Список использованных источников}
\begin{thebibliography}{100}
\addcontentsline{toc}{chapter}{Список использованных источников}

\bibitemweb{item1}{Документация по языку программирования $Go$}{https://go.dev/doc/}{20.09.2022}

\bibitemweb{item2}{Документация по пакетам языка программирования $Go$}{https://pkg.go.dev}{20.09.2022}

\bibitemweb{item3}{GoLand: IDE для профессиональной разработки на $Go$}{https://www.jetbrains.com/ru-ru/go/}{20.09.2022}

\bibitemweb{item4}{Техническая спецификация ноутбука $MacBook$ $Air$}{https://support.apple.com/kb/SP869}{20.09.2022}

\bibitemweb{item5}{$Apple$ $M2$}{https://www.notebookcheck.net/Apple-M2-Processor-Benchmarks-and-Specs.632312.0.html}{10.10.2022}

\bibitem{item6} Donovan A. A. A., Kernighan B. W. The Go programming language. – Addison-Wesley Professional, 2015.

\bibitemweb{web_item7}{$gnuplot$ $homepage$}{http://www.gnuplot.info}{10.10.2022}

\bibitemweb{item8}{Исходный код $src/testing/benchmark.go$}{https://go.dev/src/testing/benchmark.go}{10.10.2022}

\bibitem{item9} Lawler Е. L., Lenslra J. К., Rinnooy Kan А. Н. G., Shmoys D. В. ed. The traveling salesman problem. A guided tour of combinatorial optimization. N. Y.: J. Wiley \& Sons, 1985.

\bibitem{item10} Меламед И. И., Сергеев С. И., Сигал И. Х. Задача коммивояжера. Вопросы теории //Автоматика и телемеханика. – 1989. – №. 9. – С. 3-33.

\bibitem{item11} Ульянов М. В. Ресурсно-эффективные компьютерные алгоритмы. Разработка и анализ //М.: ФИЗМАТЛИТ. – 2008. – Т. 304.

\bibitem{item12} Пархоменко П. П. Классификация гамильтоновых циклов в двоичных гиперкубах //Автоматика и телемеханика. – 2001. – №. 6. – С. 136-150.

\bibitemweb{item13}{Яндекс.Карты}{https://yandex.ru/maps}{10.12.2022}

\bibitem{item14} Штовба С. Д. Муравьиные алгоритмы //Exponenta Pro. Математика в приложениях. – 2003. – Т. 4. – №. 4. – С. 70-75.

\bibitemweb{item15}{clock\_gettime(3) — Linux manual page}{https://man7.org/linux/man-pages/man3/clock\_gettime.3.html}{10.10.2022}

\end{thebibliography}

\newpage