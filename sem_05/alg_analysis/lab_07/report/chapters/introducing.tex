\setcounter{page}{3}
\chapter*{Введение}
\addcontentsline{toc}{chapter}{Введение} 
В октябре 1985 года появилась первая в мировой литературе монография \cite{item9}, посвященная целиком задаче коммивояжёра. Она была написана большим коллективом известных специалистов \cite{item10}. 
Сейчас задача коммивояжёра занимает центральное место среди труднорешаемых задач дискретной оптимизации \cite{item10}. 

Задача имеет разные формулировки, но в контексте данной лабораторной работы будет рассмотрена графовая: необходимо найти кратчайший путь прохода по всем заданным пунктам такой, чтобы каждый пункт был посещён ровно один раз и конечным пунктом оказался тот, с которого был начат обход. Условия данной задачи можно реализовать при помощи взвешенного графа, вершины которого представляют пункты, а веса рёбер~---~расстояния между соответствующими пунктами \cite{item11}. Тогда задачу можно свести к поиску кратчайшего Гамильтонова цикла \cite{item12} для неориентированного графа.

Цель работы: получение навыков программирования, тестирования полученного программного продукта и проведения замеров времени выполнения по результатам работы программы на примере реализации решения задачи коммивояжёра при помощи алгоритма полного перебора и муравьиного алгоритма.

Задачи работы:
\begin{enumerate}[label={\arabic*)}]
	\item изучение теоретических основ задачи коммивояжёра, алгоритма полного перебора и муравьиного алгоритма;
	\item описание алгоритма полного перебора и муравьиного алгоритма;
	\item реализация данных алгоритмов;
	\item выполнение оценки трудоёмкости реализаций алгоритмов;
	\item проведение параметризации для муравьиного алгоритма;
	\item проведение замеров времени работы (в мкс) данных алгоритмов на наилучшей комбинации параметров; 
	\item получение графической зависимости замеряемой величины от количества вершин в графе;
	\item проведение сравнительного анализа двух представленных реализаций  на основе полученной зависимости.
\end{enumerate}

\newpage