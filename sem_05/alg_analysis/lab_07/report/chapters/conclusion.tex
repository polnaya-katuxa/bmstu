\chapter*{Заключение}
\addcontentsline{toc}{chapter}{Заключение} 
Из проведённых замеров времени работы реализаций решения задачи коммивояжёра с помощью алгоритма полного перебора и муравьиного алгоритма  можно сделать следующие выводы.
Муравьиный алгоритм работает дольше при малых количествах вершин в графе. Например, на количестве вершин, равном 5, муравьиный алгоритм выполняется  в 230 раз медленнее алгоритма полного перебора. При больших значениях количества вершин в графе муравьиный алгоритм по временной эффективности начинает значительно превосходить алгоритм полного перебора. Например, на количестве вершин, равном 10, муравьиный алгоритм работает в 38 раз быстрее алгоритма полного перебора.

В ходе выполнения лабораторной работы была достигнута поставленная цель: были получены навыки программирования, тестирования полученного программного продукта и проведения замеров времени выполнения по результатам работы программы на примере реализации решения задачи коммивояжёра при помощи алгоритма полного перебора и муравьиного алгоритма.

В процессе выполнения лабораторной работы были также реализованы все поставленные задачи, а именно:
\begin{itemize}
	\item были изучены теоретические основы задачи коммивояжёра, алгоритма полного перебора и муравьиного алгоритма;
	\item были описаны алгоритм полного перебора и муравьиный алгоритм;
	\item была выполнена программная реализация данных алгоритмов;
	\item была выполнена оценка трудоёмкости реализаций алгоритмов;
	\item была проведена параметризация для муравьиного алгоритма;
	\item была выполнена программная реализация параллельных конвейерных вычислений с не менее чем тремя линиями;
	\item были проведены замеры времени работы (в мкс) для данных алгоритмов;
	\item была получена зависимость замеряемой величины от количества вершин в графе, предоставляемом на вход алгоритму;
	\item был проведен сравнительный анализ двух представленных реализаций решения задачи коммивояжёра на основе полученной зависимости.
\end{itemize}

\newpage