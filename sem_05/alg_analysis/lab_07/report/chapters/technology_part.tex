\chapter{Технологическая часть}

В данном разделе будет представлена реализация алгоритмов решения задачи коммивояджёра. Также будут указаны обязательные требования к ПО, средства реализации алгоритмов и результаты проведённого тестирования программы.

\section{Требования к ПО}
Для программы выделен перечень требований:
\begin{itemize}
	\item программой обрабатываются графы, записанные в файле указанной директории в виде симметричной матрицы смежности с предварительным указанием размерности;
	\item программой принимается режим запуска (обычный или для замеров), значения параметров $\alpha$, $\beta$, $\rho$, $time$, описанные ранее;
	\item программой производится аварийное завершение с текстом об ошибке при ошибках;
	\item программой проводится модульное тестирование функций реализации решения задачи коммивояджёра;
	\item программой производятся замеры времени выполнения;
	\item программой на экран выводится результат работы~---~последовательность вершин кратчайшего маршрута и его длина.
\end{itemize}

\section{Средства реализации}
Для реализации данной работы был выбран язык программирования Go \cite{item1}. Выбор обусловлен наличием в $Go$ библиотек для тестирования ПО и проведения замеров времени выполнения в том числе при помощи вставок на других языках программирования, а также необходимых для реализации поставленных цели и задач средств. В качестве среды разработки была выбрана $GoLand$ \cite{item3}.

\section{Реализация алгоритмов}
В листингах \ref{code:bf}~--~\ref{code:go2} представлены реализации функций для решения задачи коммивояджёра и необходимые для реализации муравьиного алгоритма подпрограммы: поиска муравьём маршрута и поиска муравьём следующей вершины маршрута. 

\begin{code}
\caption{Листинг функции реализации алгоритма полного перебора для решения задачи коммивояджёра}
\label{code:bf}
\begin{minted}{go}
func TravellingSalesmanBF(g graph.Graph) ([]int, int) {
	var minRoute []int
	
	verts := make([]int, g.Size)
	for i := range verts {
		verts[i] = i
	}

	permuts := permutations(verts)
	minTax := math.MaxInt
	
	for i := range permuts {
		if g.IsOKRoute(permuts[i]) {
			cost := g.RouteTotalTax(permuts[i])
			
			if cost < minTax {
				minTax = cost
				minRoute = permuts[i]
			}
		}
	}
	
	return minRoute, minTax
}
\end{minted}
\end{code}

\newpage

\begin{code}
\caption{Листинг функции реализации муравьиного алгоритма для решения задачи коммивояджёра}
\label{code:aco}
\begin{minted}{go}
func TravellingSalesmanACO(g graph.Graph, alpha, beta, k float64, time int) 
([]int, int, error) { 
	pheromone := createPheromoneMatrix(g.Size)
	vision := createVisionMatrix(g)
	minRoute := make([]int, g.Size)
	minTax := math.MaxInt
	q := getQ(g)

	for t := 0; t < time; t++ {
		c := createColony(g.Size)
		for _, v := range c.Members {
			err := v.FindRoute(g, pheromone, vision, alpha, 
			beta)
			if err != nil {
				return nil, -1, fmt.Errorf("route error: 
				%w", err)
			}
			if v.Tax < minTax && (len(v.Route) == g.Size+1 || 
			g.Size == 1) {
				minTax = v.Tax
				minRoute = v.Route
			}
		}
		vaporize(pheromone, k)
		increase(pheromone, c, q)
		correction(pheromone)
	}
	if len(minRoute) == 0 {
		return nil, -1, errors.New("min route error")
	}
	return minRoute[:len(minRoute)-1], minTax, nil
}
\end{minted}
\end{code}

\begin{code}
\caption{Листинг функции реализации поиска муравьём маршрута}
\label{code:route}
\begin{minted}{go}
func (a *Ant) FindRoute(g graph.Graph, ph [][]float64, vis [][]float64, 
alpha, beta float64) error {
	toGo := make([]int, 0)
	for i := range g.Connection {
		if i != a.Pos {
			toGo = append(toGo, i)
		}
	}
	cycle := false
	for len(toGo) != 0 {
		next, err := a.Go(toGo, ph, vis, alpha, beta)
		if err != nil {
			return err
		}
		a.Move(g.Connection[a.Pos][next], next)
		toGo = removeElement(toGo, next)
		if len(toGo) == 0 && !cycle {
			toGo = append(toGo, a.Route[0])
			cycle = true
		}
	}
	return nil
}
\end{minted}
\end{code}

\begin{code}
\caption{Листинг функции реализации поиска муравьём следующей вершины (начало)}
\label{code:go1}
\begin{minted}{go}
func (a *Ant) Go(toGo []int, ph [][]float64, vis [][]float64, alpha, beta 
float64) (int, error) {
	probs := make([]float64, 0)
	sumProbs := 0.0
	for _, v := range toGo {
		if vis[a.Pos][v] != -1 {
			greed := math.Pow(vis[a.Pos][v], beta)
\end{minted}
\end{code}

\begin{code}
\caption{Листинг функции реализации поиска муравьём следующей вершины (продолжение листинга \ref{}code:go1)}
\label{code:go2}
\begin{minted}{go}
			herd := math.Pow(ph[a.Pos][v], alpha)
			prob := greed * herd
			probs = append(probs, prob)
			sumProbs += prob
		}
	}
	if sumProbs <= 0 {
		return -1, errors.New("probability error")
	}
	
	maxProb := 0.0
	greedChoice := 0
	for i, v := range probs {
		v /= sumProbs
		if v > maxProb {
			maxProb = v
			greedChoice = toGo[i]
		}
	}
	if a.IsElite {
		return greedChoice, nil
	}
	
	choice := 0
	curSum := 0.0
	rand.Seed(time.Now().UnixNano())
	randPoint := rand.Float64() * sumProbs
	for curSum < randPoint {
		curSum += probs[choice]
		choice++
	}
	return toGo[choice-1], nil
}
\end{minted}
\end{code}

\newpage

\section{Тестирование}
В таблице \ref{table:tests} представлены тесты для решения задачи коммивояджёра с помощью алгоритма полного перебора и муравьиного алгоритма. Все тесты пройдены успешно. Тестирование проводилось на матрицах, заполненных случайными числами по методологии чёрного ящика. Для муравьиного алгоритма вводились следующие параметры: $\alpha = 0.5$, $\beta = 0.5$, $\rho = 0.5$, $time = 10$.

\begin{table}[H]
  \caption{\label{table:tests} Тесты для решения задачи коммивояджёра}
  \begin{center}
    \begin{tabular}{|c|c|c|}
      \hline
      № &  Матрица смежности графа & Результат \\ \hline
      1 & $\begin{pmatrix}
      	0
      \end{pmatrix}$ & [] 0 \\ \hline
      2 & $\begin{pmatrix}
      	0 & 34 \\
      	34 & 0
      \end{pmatrix}$ & [0 1] 68 \\ \hline
      3 & $\begin{pmatrix}
      	0 & 24 & 6 \\
      	24 & 0 & 13 \\
      	6 & 13 & 0
      \end{pmatrix}$ & [0 2 1] 43 \\ \hline
      4 & $\begin{pmatrix}
      	0 & 12 & 5 & 23 & 56 \\
      	12 & 0 & 31 & 4 & 13 \\
      	5 & 31 & 0 & 8 & 2 \\
      	23 & 4 & 8 & 0 & 11 \\
      	56 & 13 & 2 & 11 & 0
      \end{pmatrix}$ & [3 4 2 0 1] 34  \\ \hline
      5 & $\begin{pmatrix}
      	0 & 45 & 12 & 67 & 88 & 22 & 14 & 4 \\
      	45 & 0 & 5 & 7 & 89 & 34 & 121 & 7 \\
      	12 & 5 & 0 & 23 & 45 & 32 & 43 & 12 \\
      	67 & 7 & 23 & 0 & 44 & 44 & 32 & 2 \\
      	88 & 89 & 45 & 44 & 0 & 4 & 56 & 21 \\
      	22 & 34 & 32 & 44 & 4 & 0 & 22 & 47 \\
      	14 & 121 & 43 & 32 & 56 & 22 & 0 & 9 \\
      	4 & 7 & 12 & 2 & 21 & 47 & 9 & 0
      \end{pmatrix}$ & [5 4 7 3 1 2 0 6] 87 \\ \hline
    \end{tabular}
  \end{center}
\end{table}

\newpage