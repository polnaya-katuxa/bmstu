\setcounter{page}{4}
\chapter*{Введение}
\addcontentsline{toc}{chapter}{Введение} 
В наши дни, в связи с повсеместной компьютеризацией, востребованность технологии компьютерного графического моделирования растёт \cite{web_item11}. Синтез изображения с использованием информационных технологий применяется в таких сферах, как:
\begin{itemize}
	\item медицина;
	\item архитектура;
	\item кинематограф;
	\item разработка компьютерных игр.
\end{itemize}

Данный способ представления данных является более наглядным и позволяет получить полное визуальное представление какого-либо разрабатываемого проекта. 
%Это делает компьютерную графику применимой практически для любых задач проектирования, где требуется согласовывать полученный результат, например, с заказчиком для дальнейшего обсуждения деталей и планов развития проекта.

Построение трёхмерного изображения мыльных пузырей вещества может быть применимо в контексте изучения физических свойств мыльных пузырей и химических свойств вещества для повышения наглядности представления. Также, подобные изображения могут быть использованы при разработке компьютерных игр или в процессе постпроизводства отснятых кинолент.

Цель работы: разработка программы для построения трёхмерного изображения мыльных пузырей вещества.\\

Для достижения поставленной цели необходимо выполнить следующие задачи:
\begin{enumerate}[label={\arabic*)}]
	\item формальное описание заданных объектов сцены;
	\item изучение и сравнение алгоритмов построения реалистичных изображений, необходимых для визуализации мыльных пузырей вещества;
	\item изучение оптических свойств мыльных пузырей вещества для синтеза реалистичного изображения;
	\item разработка алгоритма, позволяющего построить реалистичное изображение мыльных пузырей вещества;
	\item описание структуры разрабатываемого ПО;
	\item выбор средств разработки, позволяющих реализовать выбранный алгоритм;
	\item программная реализация данного алгоритма;
	\item проведение исследования зависимости времени работы программы от количества выделяемых потоков.
\end{enumerate}

\newpage