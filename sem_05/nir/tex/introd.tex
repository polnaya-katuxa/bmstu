\chapter*{Определения}
\addcontentsline{toc}{chapter}{Определения}
В настоящей расчетно-пояснительной записке применяют следующие термины с соответствующими определениями.

Ритм~---~организация музыки во времени, последовательность длительностей~---~звуков и пауз (другими словами, последовательность ударов воображаемого метронома, согласованная в каждый момент времени с музыкальным рисунком проигрываемой композиции) \cite{bib16}.

Распознавание музыки~---~перевод звукового сигнала в цифровое представление, с которым, в дальнейшем, и происходит взаимодействие (цифровое) \cite{bib18}.

Интерпретация ритма~---~вариант восприятия ритмического паттерна музыкального фрагмента человеком.

Транскрибирование~---~это перевод содержания аудиофайла в символьный формат \cite{bib20}.

Вероятностная модель~---~это математическая модель реального явления, содержащего элементы принципиально неустранимой неопределённости (случайности).

Нейронная сеть~---~математическая модель, а также её программное или аппаратное воплощение, построенная по принципу организации и функционирования биологических нейронных сетей — сетей нервных клеток живого организма.

$MIDI$-файл~---~стандарт цифровой звукозаписи на формат обмена данными между электронными музыкальными инструментами, интерфейс позволяет единообразно кодировать в цифровой форме такие данные как нажатие клавиш, настройку громкости и других акустических параметров, выбор тембра, темпа, тональности и др., с точной привязкой во времени \cite{bib21}.

Коннекционизм~---~один из подходов в области искусственного интеллекта, когнитивной науки, нейробиологии, психологии и философии сознания, моделирующий мыслительные или поведенческие явления процессами становления в сетях из связанных между собой простых элементов.

Скрытая переменная~---~переменные, которые не могут быть измерены в явном виде, а могут быть только выведены через математические модели с использованием наблюдаемых переменных.

Скрытая модель Маркова~---~статистическая модель, имитирующая работу процесса, похожего на марковский процесс с неизвестными параметрами, и задачей ставится разгадывание неизвестных параметров на основе наблюдаемых \cite{bib17}.

Марковский процесс~---~случайный процесс, эволюция которого после любого заданного значения временного параметра $t$ не зависит от эволюции, предшествовавшей $t$, при условии, что значение процесса в этот момент фиксировано \cite{bib19}. 

Онсет~---~начало ноты или удар.

\chapter*{Обозначения и сокращения}
\addcontentsline{toc}{chapter}{Обозначения и сокращения}

$MIDI$~---~$Musical$ $Instrument$ $Digital$ $Interface$

$HMM$~---~$Hidden$ $Markov$ $Model$

\chapter*{Реферат}
\addcontentsline{toc}{chapter}{Реферат}

Целью данной работы является анализ известных методов распознавания музыкальных фрагментов на основе интерпретации ритма и их классификация.

В результате проведён анализ заданной предметной области, были рассмотрены существующие методы распознавания музыкальных фрагментов на основе интерпретации ритма и была проведена их классификация на основе сформулированных критериев сравнения.

Ключевые слова: распознавание музыкальных фрагментов, интерпретация ритма, транскрибирование, поиск расстояния, ноты, ритм, расстояние Хэмминга, расстояние Евклида, интервально-разностное расстояние, редакционное расстояние, хронотоническое расстояние.

\newpage

\chapter*{Введение}
\addcontentsline{toc}{chapter}{Введение}

Уже долгое время музыка является неотъемлемой частью жизни человека: прослушивание музыки может повысить уровень концентрации, поднять настроение, настроить на определённую деятельность и способствовать продуктивности. Научно доказано, что прослушивание музыки оказывает значительное влияние на состояние человеческого организма \cite{bib1}.

Благодаря работе мозга и слухового аппарата, человек может запоминать ритм музыки и впоследствии распознавать её заново. 

С развитием техники способностью распознавать музыку смог обладать не только человек, но и различные устройства, появились новые возможности обработки звуковых сигналов. 

Музыка скаладывается из многих компонентов, таких как мелодия, темп, громкость и т.д. Одним из основных элементов, характеризующих музыкальный фрагмент, является также его ритм \cite{bib15}.

Целью данной работы является анализ известных методов распознавания музыкальных фрагментов на основе интерпретации ритма и их классификация.

Для достижения поставленной цели необходимо решить следующие задачи:
\begin{itemize}
    \item провести анализ предметной области распознавания музыкальных \\фрагментов на основе интерпретации ритма;
    \item провести обзор существующих методов распознавания музыкальных \\фрагментов на основе интерпретации ритма;
    \item сформулировать критерии их сравнения;
    \item классифицировать данные методы.
\end{itemize}