\chapter{Контрольные вопросы}

\section{Что такое операторная грамматика?}

Операторная грамматика~---~КС-грамматика без $\epsilon$-правил, в которой правые части всех правил не содержат смежных нетерминальных символов

\section{Что такое грамматика операторного предшествования?}

Операторная грамматика G называется грамматикой операторного предшествования, если между любыми двумя терминальными символами выполняется не более одного отношения операторного предшествования.

% Грамматики операторного предшествования~---~операторные грамматики, в которых:
% \begin{enumerate}
% 	\item для каждой упорядоченной пары терминальных символов выполняется не более чем одно из отношений предшествования;
% 	\item различные правила имеют разные правые части.
% \end{enumerate}

\section{Как определяются отношения операторного предшествования?}

\begin{enumerate}
	\item $a \equaldot b$, если $A \rightarrow \alpha a \gamma b \beta \in P$ и $\gamma \in N \cup \{\epsilon\}$.
	\item $a \lessdot b$, если $A \rightarrow \alpha a B \beta \in P$ и $B \implies \gamma b \delta$, где $\gamma \in N \cup \{\epsilon\}$.
	\item $a \gtrdot b$, если $A \rightarrow \alpha B b \beta \in P$ и $B \implies \delta a \gamma$, где $\gamma \in N \cup \{\epsilon\}$.
	\item $\$ \lessdot a$, если $S \implies \gamma a \alpha$ и $\gamma \in N \cup \{\epsilon\}$.
	\item $a \gtrdot \$$, если $S \implies \alpha a \gamma$ и $\gamma \in N \cup \{\epsilon\}$.
\end{enumerate}

\section{Как выделяется основа в процессе синтаксического разбора операторного предшествования?}

Основу правовыводимой цепочки грамматики можно выделить, просматривая эту цепочку слева направо до тех пор, пока впервые не встретится отношение $\gtrdot$. Для нахождения левого конца основы надо возвращаться назад, пока не встретится отношение $\lessdot$. Цепочка, заключенная между $\lessdot$ и $\gtrdot$ , будет основой. Если грамматика предполагается обратимой, то основу можно однозначно свернуть. Этот процесс продолжается до тех пор, пока входная цепочка не свернется к начальному символу (либо пока дальнейшие свертки окажутся невозможными).

\section{Какие виды синтаксических ошибок не обнаруживаются в предложенном примере?}

\begin{enumerate}
	\item Ошибки, связанные с ограниченным размером контекста. Пример: если указать после последнего оператора в блоке невалидный идентификатор вместо точки с запятой, то будет ошибка "отсутствует end блок".
	\item Ошибки, связанные с неоднозначностью. Пример: если вместо валидной операции отношения указать неизвестный символ, то, будет ошибка, связанная с другим нетерминалом, а не с операцией отношения.
	\item Пропуск множественных ошибок в одном выражении.
	\item Ошибки, связанные с контекстом.
\end{enumerate}

\section{Какие действия надо предпринять для обнаружения всех синтаксических ошибок в предложенном примере?}

\begin{enumerate}
	\item Увеличить количество просматриваемых символов.
	\item Ручное восстановление после ошибок.
	\item Реализовать метод правого разбора.
\end{enumerate}

\section{Как сформулировать синтаксически управляемые определения для перевода инфиксного выражения в последовательность команд стековой машины?}

Установить последовательность команд, которые будут установлены относительно правил грамматики.
	Например, для правила $E \rightarrow E + T$ может соответствовать команда \texttt{ADD;}, правилу 
	$T \rightarrow T * F$ соответствует команда \texttt{MUL;}, правилу $F \rightarrow a$ соответствует команда \texttt{LOAD a;}.

\section{Как сформулировать синтаксически управляемые определения для перевода инфиксного выражения в абстрактное синтаксическое дерево?}

Правилам будет соответствовать создание узлов дерева.
Например, для правила $E \rightarrow E + T$ 
может соответствовать команда создания узла \texttt{NewSumNode(nodeE, nodeT)}.

