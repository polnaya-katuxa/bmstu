\chapter{Контрольные вопросы}

\section{Какие из следующих множеств регулярны? Для тех, которые регулярны, напишите регулярные выражения}

\begin{enumerate}
	\item Множество цепочек с равным числом нулей и единиц
	
	Не является регулярным.
	
	\item Множество цепочек из {0, 1}* с четным числом нулей и нечетным числом единиц
	
	((0(00)*1)(1(00)*1)*1(00)*01|(0(00)*1)(1(00)*1)*0|0(00)*01|1)
(((10)(00)*1|0)(1(00)*1)*1(00)*01|((10)(00)*1|0)(1(00)*1)*0|(10)(00)*01|11)*

	\item Множество цепочек из {0, 1}*, длины которых делятся на 3
	
	((0|1)(0|1)(0|1))*
	
	\item Множество цепочек из {0, 1}*, не содержащих подцепочки 101
	
	0*(1|00|000)*0*
\end{enumerate}

\section{Найдите праволинейные грамматики для тех множеств из вопроса 1, которые регулярны}

\begin{enumerate}
	\item[2.] Множество цепочек из {0, 1}* с четным числом нулей и нечетным числом единиц
	
	S -> 1C|0A 
	
	A -> 0S|1B
	
	B -> 1A|0C
	
	C -> 1S|0B|$\epsilon$
	
	\item[3.] Множество цепочек из {0, 1}*, длины которых делятся на 3
	
	S -> 0A|1A|$\epsilon$
	
	A -> 0B|1B
	
	B -> 0S|1S
	
	\item[4.] Множество цепочек из {0, 1}*, не содержащих подцепочки 101
	
	S -> 0S|$\epsilon$A
	
	A -> 1A|00A|000A|$\epsilon$B
	
	B -> 0B|$\epsilon$
	
\end{enumerate}

\section{Найдите детерминированные и недетерминированные конечные автоматы для тех множеств из вопроса 1, которые регулярны}

\begin{enumerate}
	\item[2.] Множество цепочек из {0, 1}* с четным числом нулей и нечетным числом единиц
	
	\begin{enumerate}
		\item НКА
		
		\includeimage
    	{nfa_2}
    	{f}
    	{h!}
    	{0.9\textwidth}
    	{НКА}
		
		\item ДКА
		
		\includeimage
    	{dfa_2}
    	{f}
    	{h!}
    	{0.9\textwidth}
    	{ДКА}
		
	\end{enumerate}
	
	\item[3.] Множество цепочек из {0, 1}*, длины которых делятся на 3
	
	\begin{enumerate}
		\item НКА
		
		\includeimage
    	{nfa_3}
    	{f}
    	{h!}
    	{0.9\textwidth}
    	{НКА}
		
		\item ДКА
		
		\includeimage
    	{dfa_3}
    	{f}
    	{h!}
    	{0.9\textwidth}
    	{ДКА}
		
	\end{enumerate}
	
	\item[4.] Множество цепочек из {0, 1}*, не содержащих подцепочки 101
	
	\begin{enumerate}
		\item НКА
		
		\includeimage
    	{nfa_4}
    	{f}
    	{h!}
    	{0.9\textwidth}
    	{НКА}
		
		\item ДКА
		
		\includeimage
    	{dfa_4}
    	{f}
    	{h!}
    	{0.9\textwidth}
    	{ДКА}
		
	\end{enumerate}
	
\end{enumerate}

\section{Найдите конечный автомат с минимальным числом состояний для языка, определяемого автоматом M = (\{A, B, C, D, E\}, \{0, 1\}, d, A, \{E, F\}), где функция задается таблицей}

\newpage

% Please add the following required packages to your document preamble:
% \usepackage{multirow}
\begin{table}[h!]
\caption{}
\label{tab:my-table}
\begin{tabular}{|l|ll|}
\hline
\multirow{2}{*}{\textbf{Состояние}} & \multicolumn{2}{l|}{\textbf{Вход}}           \\ \cline{2-3} 
                                    & \multicolumn{1}{l|}{\textbf{0}} & \textbf{1} \\ \hline
A                                   & \multicolumn{1}{l|}{B}          & C          \\ \hline
B                                   & \multicolumn{1}{l|}{E}          & F          \\ \hline
C                                   & \multicolumn{1}{l|}{A}          & A          \\ \hline
D                                   & \multicolumn{1}{l|}{F}          & E          \\ \hline
E                                   & \multicolumn{1}{l|}{D}          & F          \\ \hline
F                                   & \multicolumn{1}{l|}{D}          & E          \\ \hline
\end{tabular}
\end{table}

Исходный конечный автомат:

\includeimage
    {last_dfa}
    {f}
    {h!}
    {0.9\textwidth}
    {Конечный автомат}

Классы 0-эквивалентности:

\begin{equation*}
\{A, B, C, D\}, \{E, F\}.
\end{equation*}

Классы 1-эквивалентности:

\begin{equation*}
\{A, C\}, \{B, D\}, \{E, F\}.
\end{equation*}

Классы 2-эквивалентности:

\begin{equation*}
\{A\}, \{C\}, \{B, D\}, \{E, F\}.
\end{equation*}

Классы 3-эквивалентности:

\begin{equation*}
\{A\}, \{C\}, \{B, D\}, \{E, F\}.
\end{equation*}

Классы 3-эквивалентности и 2-эквивалентности совпадают. Таким образом, в минимизированном конечном автомате будет 4 состояния:

\begin{equation*}
\{A\}, \{C\}, \{B, D\}, \{E, F\}.
\end{equation*}

Таким образом, минимизированный конечный автомат:

\includeimage
    {last_minimized}
    {f}
    {h!}
    {0.7\textwidth}
    {Минимизированный конечный автомат}