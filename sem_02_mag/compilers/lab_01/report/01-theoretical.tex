\chapter{Теоретическая часть}

\textbf{Цель работы:} приобретение практических навыков реализации важнейших элементов лексических анализаторов на примере распознавания цепочек регулярного языка.

\textbf{Задачи работы:}

\begin{enumerate}
	\item Ознакомиться с основными понятиями и определениями, лежащими в основе построения лексических анализаторов.
	\item Прояснить связь между регулярным множеством, регулярным выражением, праволинейным языком, конечно-автоматным языком и недетерминированным конечно-автоматным языком.
	\item Разработать, тестировать и отладить программу распознавания цепочек регулярного или праволинейного языка в соответствии с предложенным вариантом грамматики.
\end{enumerate}

\section{Задание}

Напишите программу, которая в качестве входа принимает произвольное регулярное выражение, и выполняет следующие преобразования:

\begin{enumerate}
	\item Преобразует регулярное выражение непосредственно в ДКА.
	\item По ДКА строит эквивалентный ему КА, имеющий наименьшее возможное количество состояний. Указание. Воспользоваться алгоритмом, приведенным по адресу \textit{http://neerc.ifmo.ru/wiki/index.php?title=Алгоритм\_Бржозовского}
	\item Моделирует минимальный КА для входной цепочки из терминалов исходной грамматики.
\end{enumerate}



%\includeimage
%    {general}
%    {f}
%    {h}
%    {0.9\textwidth}
%    {Общая схема алгоритма решения задачи о назначениях}
   
