\chapter{Технологический раздел}

%В данном разделе описан выбор системы управления базой данных, средств разработки серверной и клиентской частей ПО, интерфейсов и представлена общая схема взаимодействия компонентов системы. Также, описаны методы тестирования и обеспечения безопасности данных.

\section{Выбор средств программной реализации}

В качестве языка программирования был выбран язык Go, ввиду следующих причин.
\begin{itemize}
	\item Компилятор, написанный на Go может быть запущен на различных платформах.
	\item Для языка Go существуют библиотеки, позволяющие генерировать код для LLVM.
	\item Выбранный генератор лексических и синтаксических анализаторов ANTLR позволяет генерировать код на Go.
\end{itemize}

\section{Основные компоненты программы}

В результате работы ANTLR были сгенерированы интерфейсы для Visitor и Listener, файлы с данными для интерпретатора ANTLR, файлы с токенами и реализации анализаторов.

На языке Go был реализован интерфейс Visitor, так как, несмотря на большие требования по контролю за обходом дерева, он предоставляет больше гибкости в анализе дерева и позволяет возвращать значения из обработчиков.
Для каждой семантической единицы была прописана собственная логика анализа. 

\includelistingpretty{semantic}{}{Пример реализации посещения для оператора логического "или"}

Примеры программ на Lua и соответствующий им результат работы компилятора на LLVM IR представлены в Приложении~Б.

Язык Lua имеет обладает специфическими особенностями, что вызвало необходимость в реализации дополнительного функционала. Далее представлено более подробное описание особенностей Lua.

\textbf{Динамическая типизация}

Язык Lua динамически типизирован, а Go и LLVM IR~---~строго типизированы. Чтобы компилятор поддерживал работу с динамическими типами, был реализован специальный тип Generic.

\includelistingpretty{generic}{}{Реализация типа Generic на LLVM IR}

Этот тип хранит в себе идентификатор одного из доступных типов данных и указатель на область памяти со значением.
Значение любого типа языка Lua, преобразуется в переменную типа Generic на LLVM IR.

Go и LLVM IR, в отличие от Lua, не позволяют реализовывать операции для работы с разными типами данных. Например, нельзя напрямую сравнить целое число с вещественным.
Для поддержания такого функционала для каждой функции, которую необходимо применять к операндам разного типа, было реализовано множество
подфункций для работы с каждой комбинацией типов операндов, внутри которой операнды приводятся к одному типу, когда это необходимо и возможно.

\textbf{Работа с таблицами}

В языке Lua таблицы, массивы и структуры представлены одним типом таблиц или ассоциативных массивов. 

\includelistingpretty{tables}{}{Таблицы, массивы и структуры в языке Lua}

С помощью типа Generic, описанного ранее, также можно представить и таблицу языка Lua. 
На LLVM IR были реализованы стандартные функции для работы с таблицами: создание объекта таблицы, добавление элемента, получение элемента по ключу и пр.

\textbf{Возврат значений из функций}

Язык Lua, как и многие другие языки, поддерживает определение функций, не возвращающих никаких значений.
В LLVM IR функция всегда должна возвращать ровно одно значение. Для реализации функционала по возврату нескольких значений из функции
была предусмотрена возможность возврата таблицы как набора возвращаемых функцией значений, формально представленным одним значением.
Если функция на Lua не возвращает значений, то из функции на LLVM IR возвращается значение null.

\includelistingpretty{llvm}{}{Реализация функции создания null-переменной на LLVM IR}

\textbf{Возврат ошибок}

Для вывода ошибок в процессе компиляции в структуру Visitor был добавлен список ошибок, который наполняется в ходе компиляции и выводится в стандартный поток ввода-вывода.
Также, реализована проверка областей видимости переменных: в структуру Visitor добавлен словарь объявленных на текущий момент переменных, содержание которого меняется в зависимости от анализируемого блока программы.

Для обработки ошибочных ситуаций в процессе исполнения скомпилированного кода была реализована функция panic, которая обеспечивает вывод причины сбоя в программе в стандартный поток ввода-вывода. 
Например, такой ошибочной ситуацией может быть передача целочисленного операнда функции конкатенации строк.

\includelistingpretty{panic}{}{Реализация функции panic на LLVM IR}

\textbf{Базовые функции языка}

На LLVM IR были написаны базовые функции. В том числе:
\begin{itemize}
	\item реализация арифметических операций;
	\item реализация логических операций;
	\item реализация операций сравнения;
	\item реализация функций для строкового типа: конкатенация, получение длины;
	\item реализация функций для таблиц, массивов, структур: создание объекта, добавление элемента, получение элемента по ключу и пр.;
	\item реализация функции вывода в стандартный поток ввода-вывода.
\end{itemize} 

\section{Тестирование}

Было проведено тестирование работы компилятора для базовых конструкций языка Lua в соответствии с грамматикой.

Тестовые коды программ представлены в Приложении~В.

Ниже представлен пример работы программы.

\includelistingpretty{small}{}{Пример исходного кода на Lua}

\includelistingpretty{res}{}{Результат работы программы для примера}

В результате исполнения программы в стандартный поток ввода-вывода было выведено число 5.

\section{Вывод}
В данном разделе был описан выбор средств программной реализации, описано тестирование разработанной программы и приведены примеры работы компилятора.







