\chapter{Технологический раздел}

%В данном разделе описан выбор системы управления базой данных, средств разработки серверной и клиентской частей ПО, интерфейсов и представлена общая схема взаимодействия компонентов системы. Также, описаны методы тестирования и обеспечения безопасности данных.

\section{Выбор средств программной реализации}

В качестве языка программирования был выбран язык Go, ввиду следующих причин.
\begin{itemize}
	\item Компилятор, написанный на Go может быть запущен на различных платформах.
	\item Для языка Go существуют библиотеки, позволяющие генерировать код для LLVM.
	\item Выбранный генератор лексических и синтаксических анализаторов ANTLR позволяет генерировать код на Go.
\end{itemize}

\section{Основные компоненты программы}

В результате работы ANTLR были сгенерированы интерфейсы для Visitor и Listener, файлы с данными для интерпретатора ANTLR, файлы с токенами и реализации анализаторов.

На языке Go был реализован интерфейс Visitor, где для каждой смантической единицы была прописана логика анализа. 

\includelistingpretty{semantic}{}{Пример реализации посещения для цикла while}

На LLVM IR были написаны базовые функции для обработки типов и базовых операций с ними. Для работы с динамической типизацией был реализован тип Generic.

\includelistingpretty{generic}{}{Реализация типа Generic на LLVM IR}

\includelistingpretty{llvm}{}{Реализация функции получения длины строки на LLVM IR}

\section{Тестирование}

Было проведено тестирование работы компилятора для базовых конструкций языка Lua в соответствии с грамматикой.

Тестовые коды программ представлены в Приложении~В.

Ниже представлен пример работы программы.

\includelistingpretty{small}{}{Пример исходного кода на Lua}

\includelistingpretty{res}{}{Результат работы программы для примера}

В результате исполнения программы в стандартный поток ввода-вывода было выведено число 5.

\section{Вывод}
В данном разделе был описан выбор средств программной реализации, описано тестирование разработанной программы и приведены примеры работы компилятора.







