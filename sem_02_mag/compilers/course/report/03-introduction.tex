\chapter*{ВВЕДЕНИЕ}
\addcontentsline{toc}{chapter}{ВВЕДЕНИЕ}

Транслятор~---~это программа перевода текста программы с исходного языка на объектный~\cite{bib1}. Если исходный язык является языком программирования высокого уровня, и, если объектный язык~---~язык ассемблера или машинный язык, то транслятор называют \textbf{компилятором}. Трансляция исходной программы в объектную выполняется во время компиляции, а фактическое выполнение объектной программы во время выполнения готовой программы.

Целью данной работы является разработка компилятора языка Lua. Компилятор должен выполнять чтение текстового файла, содержащего код на языке Lua и генерировать на выходе программу, пригодную для запуска.


В ходе работы необходимо решить следующие задачи:
\begin{enumerate}
	\item Проанализировать грамматику языка Lua и выделить ее ключевые составляющие.
	\item Изучить существующие средства для анализа исходных кодов программ, системы для генерации низкоуровневого кода, запуск которого возможен на большинстве из используемых платформ и операционных систем.
	\item Разработать прототип компилятора на языке Go, выполняющий синтаксический анализ исходного текста программы и построение абстрактного синтаксического дерева.
	\item Провести преобразование абстрактного синтаксического дерева в IR с использованием LLVM.
\end{enumerate}