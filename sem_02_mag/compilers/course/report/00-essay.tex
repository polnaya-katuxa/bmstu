\begin{essay}{}

\par \textbf{Ключевые слова:} Компилятор, Lua, Go, ANTLR4, LLVM, IR.

\par Объектом разработки является компилятор языка Lua.

% \par В процессе разработки был проведён анализ предметной области и сравнительный анализ предложенного решения относительно существующих. Формализованы варианты использования, хранимые данные и предоставляемые сервисом возможности. Проведён выбор модели данных, базы данных по способу хранения и системы управления базой данных по способу доступа к базе данных. Составлена ER-диаграмма сущностей проектируемой базы данных в нотации Чена и спроектированы сущности базы данных и накладываемые ограничения целостности. Была описана проектируемая ролевая модель на уровне базы данных, определены права доступа к внутренним структурам и обоснован выбор средств реализации базы данных и приложения. Описан интерфейс доступа к базе данных. Также, проведено исследование производительности строковых и колоночных СУБД на агрегационных запросах.

% \par По результатам исследования сформулирован вывод о большей производительности колоночных баз данных по сравнению со строковыми для выполнения агрегационных запросов.

\end{essay}