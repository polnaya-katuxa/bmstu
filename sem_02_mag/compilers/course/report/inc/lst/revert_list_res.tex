define i64 @main() {
0:
	%1 = call %Generic* @lua_table_new()
	%2 = call %Generic* @create(i32 0, i8* inttoptr (i64 0 to i8*))
	%3 = call %Generic* @create(i32 0, i8* inttoptr (i64 0 to i8*))
	call void @lua_table_set(%Generic* %1, %Generic* %3, %Generic* %2)
	%4 = call %Generic* @create(i32 0, i8* inttoptr (i64 1 to i8*))
	%5 = call %Generic* @create(i32 0, i8* inttoptr (i64 1 to i8*))
	call void @lua_table_set(%Generic* %1, %Generic* %5, %Generic* %4)
	%6 = call %Generic* @create(i32 0, i8* inttoptr (i64 2 to i8*))
	%7 = call %Generic* @create(i32 0, i8* inttoptr (i64 2 to i8*))
	call void @lua_table_set(%Generic* %1, %Generic* %7, %Generic* %6)
	%8 = call %Generic* @create(i32 0, i8* inttoptr (i64 3 to i8*))
	%9 = call %Generic* @create(i32 0, i8* inttoptr (i64 3 to i8*))
	call void @lua_table_set(%Generic* %1, %Generic* %9, %Generic* %8)
	%10 = call %Generic* @create(i32 0, i8* inttoptr (i64 4 to i8*))
	%11 = call %Generic* @create(i32 0, i8* inttoptr (i64 4 to i8*))
	call void @lua_table_set(%Generic* %1, %Generic* %11, %Generic* %10)
	%12 = call %Generic* @revert_array(%Generic* %1)
	call void @print(%Generic* %12)
	ret i64 0
}

define %Generic* @revert_array(%Generic* %arr) {
0:
	%1 = call %Generic* @create(i32 0, i8* inttoptr (i64 0 to i8*))
	%2 = call %Generic* @create(i32 0, i8* inttoptr (i64 0 to i8*))
	%3 = call %Generic* @create(i32 0, i8* inttoptr (i64 1 to i8*))
	%4 = call %Generic* @create_nil()
	%5 = call %Generic* @create_nil()
	br label %19

6:
	%7 = call %Generic* @lua_table_new()
	%8 = call %Generic* @create(i32 0, i8* inttoptr (i64 0 to i8*))
	%9 = call %Generic* @create(i32 0, i8* inttoptr (i64 0 to i8*))
	%10 = call %Generic* @create(i32 0, i8* inttoptr (i64 1 to i8*))
	%11 = call %Generic* @sub(%Generic* %1, %Generic* %10)
	%12 = call %Generic* @create(i32 0, i8* inttoptr (i64 1 to i8*))
	%13 = call %Generic* @neg(%Generic* %12)
	%14 = call %Generic* @create(i32 0, i8* inttoptr (i64 1 to i8*))
	%15 = call %Generic* @neg(%Generic* %14)
	br label %28

16:
	%17 = call %Generic* @lua_table_get_key_at(%Generic* %arr, %Generic* %2)
	%18 = call %Generic* @lua_table_get_value_at(%Generic* %arr, %Generic* %2)
	call void @copy(%Generic* %17, %Generic* %4)
	call void @copy(%Generic* %18, %Generic* %5)
	br label %25

19:
	%20 = call %Generic* @lua_table_len(%Generic* %arr)
	%21 = call %Generic* @ge(%Generic* %2, %Generic* %20)
	%22 = call i1 @check(%Generic* %21)
	br i1 %22, label %6, label %16

23:
	%24 = call %Generic* @add(%Generic* %2, %Generic* %3)
	call void @copy(%Generic* %24, %Generic* %2)
	br label %19

25:
	%26 = call %Generic* @create(i32 0, i8* inttoptr (i64 1 to i8*))
	%27 = call %Generic* @add(%Generic* %1, %Generic* %26)
	call void @copy(%Generic* %27, %Generic* %1)
	br label %23

28:
	%29 = call %Generic* @lt(%Generic* %15, %Generic* %9)
	%30 = call i1 @check(%Generic* %29)
	br i1 %30, label %41, label %38

31:
	ret %Generic* %7

32:
	%33 = call %Generic* @add(%Generic* %11, %Generic* %15)
	call void @copy(%Generic* %33, %Generic* %11)
	br label %28

34:
	%35 = call %Generic* @lua_table_get(%Generic* %arr, %Generic* %11)
	call void @lua_table_set(%Generic* %7, %Generic* %8, %Generic* %35)
	%36 = call %Generic* @create(i32 0, i8* inttoptr (i64 1 to i8*))
	%37 = call %Generic* @add(%Generic* %8, %Generic* %36)
	call void @copy(%Generic* %37, %Generic* %8)
	br label %32

38:
	%39 = call %Generic* @lt(%Generic* %11, %Generic* %13)
	%40 = call i1 @check(%Generic* %39)
	br i1 %40, label %34, label %31

41:
	%42 = call %Generic* @gt(%Generic* %11, %Generic* %13)
	%43 = call i1 @check(%Generic* %42)
	br i1 %43, label %34, label %31
}