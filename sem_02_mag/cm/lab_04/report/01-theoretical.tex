\chapter{Теоретический раздел}

Цель работы: изучение метода парабол для решения задачи одномерной минимизации.

\textbf{Задание:}
\begin{enumerate}
	\item Реализовать модифицированный метод Ньютона с конечно-разностной аппроксимацией производных в виде программы на ЭВМ.
	\item Провести решение задачи 
	\begin{equation*}
		\begin{cases}
			f(x) \rightarrow \min, \\
			x \in [a, b],
		\end{cases}
	\end{equation*}
	для данных индивидуального варианта.
	\item Организовать вывод на экран графика целевой функции, найденной точки минимума $(x^{*}, f(x^{*}))$ и и последовательности точек $(x_i, f (x_i))$, аппроксимирующих точку искомого минимума (для последовательности точек следует предусмотреть возможность <<отключения>> вывода ее на экран);
	\item провести решение задачи с использованием стандартной функции \texttt{fminbnd} пакета MatLAB.
\end{enumerate}

\section{Исходные данные варианта №7}

\begin{equation*}
	f(x) = \arctg(x^3 - 5x + 1) + \left( \frac{x^2}{3x - 2} \right)^{\sqrt{3}}.
\end{equation*}

\begin{equation*}
	x \in [1, 2].
\end{equation*}

\section{Метод Ньютона}

Пусть $f(x)$~---~дважды дифференцируема на отрезке $[a;b]$ и выпукла. Следовательно, она унимодальна, и условие $f'(x)$~---~необходимо и достаточно для поиска ее минимума.

Рассмотрим решение уравнения $f'(x) = 0$ и его решение методом Ньютона.

Основная идея метода Ньютона: за очередное приближение корня уравнения принимается точка пересечения с осью $Оx$ касательной к графику функции в точке, отвечающей текущему приближению.

Уравнение касательной к графику функции $f'(x)$ в точке $x_0$ имеет вид $y = f'(x_0) + f''(x_0) * (x - x_0)$.
Тогда расчетное соотношение можно записать как $x_k = x_{k-1} - \frac{f'(x_{k-1})}{f''(x_{k-1})}$, где $x_k$~---~текущее приближение, $x_{k-1}$~---~предыдущее.

Вычисления продолжаются до тех пор, пока не выполнится одно из условий: $|\overline{x} - \overline{x}'| <= \epsilon$, где $\overline{x}'$~---~приближение $x^{*}$ с предыдущей итерации, или $|f'(x_i)| <= \epsilon$.

\section{Модифицированный метод Ньютона}

Если вычисление производных трудоемко, то используется модифицированный метод Ньютона. Тогда в качестве очередного приближения $x^{*}$ используется точка пересечения с осью $Оx$ прямой, параллельной касательной к графику функции в точке $x_0$~---~первого приближения.

Тогда расчетное соотношение можно записать как $x_k = x_{k-1} - \frac{f'(x_{k-1})}{f''(x_0)}$, где $x_k$~---~текущее приближение, $x_{k-1}$~---~предыдущее, $x_0$~---~первое приближение.

Вместо вычисления производных используются конечно-разностные аппроксимации:

\begin{equation*}
 f'(x_i) = \frac{f(x_i + h) - f(x_i - h)}{2h},
\end{equation*}

\begin{equation*}
 f''(x_i) = \frac{f(x_i + h) - 2f(x_i) + f(x_i - h)}{h^2},
\end{equation*}

где $h$~---~малая величина.

\img{0.8\textwidth}{1.pdf}{Схема алгоритма модицифированного метода Ньютона}

