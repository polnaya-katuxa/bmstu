# Лабораторная работа 1. Вариант 7.

function main()
  clc;

  debug = true;

  a = 1;
  b = 2;
  eps = 0.001;

  draw_plot(a, b, eps);

  [x_min, f_min, n] = find_min(debug, a, b, eps);
  fprintf('\n\033[36mТочка минимума (x*, f(x*)) = (%f, %f), количество вычислений функции: %d.\033[0m\n', x_min, f_min, n);
end

function [x_min, f_min, n] = find_min(debug, a, b, eps)
  x0 = a;
  f0 = f(x0);
  delta = (b - a)/4;

  if debug
    fprintf('(x0, f(x0)) = (%f, %f).\n', x0, f0);
  endif

  i = 1;

  while true
    x1 = x0 + delta;
    f1 = f(x1);
    i = i + 1;

    if debug
      fprintf('(x%d, f(x%d)) = (%f, %f).\n', i-1, i-1, x1, f1);
    endif

    if f0 > f1
      if a <= x1 <= b
        x0 = x1;
        f0 = f1;
        continue;
      endif
    endif

    if abs(delta) < eps
      x_min = x0;
      f_min = f0;
      n = i;
      return;
    endif

    delta = -delta / 4;
    x0 = x1;
    f0 = f1;
  endwhile
end

function draw_plot(a, b, step)
  x=a:step:b;
  y=f(x);
  plot(x,y);
end

function y = f(x)
  y = atan(x .^ 3 - 5 * x + 1) + ((x .^ 2) / (3 * x - 2)) .^ sqrt(3);
end