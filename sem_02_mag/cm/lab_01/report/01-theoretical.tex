\chapter{Теоретический раздел}

Цель работы: изучение метода поразрядного поиска для решения задачи одномерной минимизации.

\textbf{Задание:}
\begin{enumerate}
	\item Реализовать метод поразрядного поиска в виде программы на ЭВМ.
	\item Провести решение задачи 
	\begin{equation*}
		\begin{cases}
			f(x) \rightarrow \min, \\
			x \in [a, b],
		\end{cases}
	\end{equation*}
	для данных индивидуального варианта.
	\item Организовать вывод на экран графика целевой функции, найденной точки минимума $(x^{*}, f(x^{*}))$ и последовательности точек $(x_i, f(x_i))$, приближающих точку исходного минимума (для последовательности точек следует предусмотреть возможность «отключения» вывода ее на экран).
\end{enumerate}

\section{Исходные данные варианта №7}

\begin{equation*}
	f(x) = arctg(x^3 - 5x + 1) + \left( \frac{x^2}{3x - 2} \right)^{\sqrt{3}}.
\end{equation*}

\begin{equation*}
	x \in [1, 2].
\end{equation*}

\section{Краткое описание метода поразрядного поиска}

Метод поразрядного поиска является усовершенствованием метода перебора с целью уменьшения количества значений целевой функции $f$, которое необходимо найти для достижения заданной точности.

В основе метода поразрядного поиска лежат две идеи.

\begin{enumerate}
	\item Свойство унимодальной функции:
	если $a \le x_1 \le x_2 \le b$, то

	\begin{enumerate}
		\item если $f(x_1) \le f(x_2)$, то $x^* \in [a, x_2]$,
		\item иначе $x \in [x_1, b]$.
	\end{enumerate}

	\item Целесообразно сначала найти грубое приближением точки $x^*$ минимума с достаточно большим шагом, а затем уточнить это значение с меньшим шагом.
\end{enumerate}

Обычно сначала выбирают шаг $\Delta = \frac{b - a}{4}$, и последовательно вычисляют значения $f(x_0), f(x_1), \dots$, где $x_i = a + \Delta i, i = 0, 1, \dots$, до тех пор, пока не будет выполнено равенство $f(x_i) \le f(x_{i+1})$. В этом случае направление поиска изменяют на противоположное, а величину шага уменьшают (обычно в 4 раза).

\img{1\textwidth}{1.pdf}{Схема алгоритма поразрядного поиска}

