\chapter{Теоретический раздел}

Цель работы: изучение метода парабол для решения задачи одномерной минимизации.

\textbf{Задание:}
\begin{enumerate}
	\item Реализовать метод парабол в виде программы на ЭВМ.
	\item Провести решение задачи 
	\begin{equation*}
		\begin{cases}
			f(x) \rightarrow \min, \\
			x \in [a, b],
		\end{cases}
	\end{equation*}
	для данных индивидуального варианта.
	\item Организовать вывод на экран графика целевой функции, найденной точки минимума $(x^{*}, f(x^{*}))$ и последовательности отрезков $[x_{1i}, x_{3i}]$, содержащих точку искомого минимума (для последовательности отрезков следует предусмотреть возможность <<отключения>> вывода ее на экран).
\end{enumerate}

\section{Исходные данные варианта №7}

\begin{equation*}
	f(x) = \arctg(x^3 - 5x + 1) + \left( \frac{x^2}{3x - 2} \right)^{\sqrt{3}}.
\end{equation*}

\begin{equation*}
	x \in [1, 2].
\end{equation*}

\section{Краткое описание метода парабол}

Метод парабол является представителем группы методов, основанных на аппроксимации целевой функции некоторой более простой функцией (как правило полиномом), минимум которой можно легко найти. Точка минимума этой аппроксимируещей функции и принимается за очередное приближение точки минимума целевой функции.

Пусть

\begin{enumerate}
	\item $f$ унимодальна на $[a; b]$,
	\item $f$ минимума во внутренней точке отрезка $[a; b]$.
\end{enumerate}

Выберем три точки $x_1, x_2, x_3 \in [a; b]$, так чтобы (*):

\begin{enumerate}
	\item $x_1 < x_2 < x_3$,
	\item $f(x_1) \ge f(x_2) \le f(x_3)$~---~хотя бы одно из неравенств строгое.
\end{enumerate}

Тогда в силу унимодальности функции $f$ точка минимума $x^{*} \in [x_1; x_3]$.

Аппроксимируем целевую функцию параболой, проходящей через точки $(x_1, f_1)$, $(x_2, f_2)$, $(x_3, f_3)$, где $f_i = f(x_i)$.

В силу условий (*) ветви параболы направленны вверх. Это значит, что точка $\overline{x}$ минимума этой параболы также принадлежит отрезку [x1, x3].
Точка $\overline{x}$ принимается за очередное приближение точки $x^{*}$.

Пусть $q(x) = a_0 + a_1 \cdot (x - x_1) + a_2 \cdot (x - x_1) \cdot (x - x_2)$~---~уравнение параболы.

Можно показать, что условия $q(x_i) = f_i$, приводят к (**):

\begin{equation*}
	a_0 = f_1,
\end{equation*}

\begin{equation*}
	a_1 = \frac{f_2 - f_1}{x_2 - x_1},
\end{equation*}

\begin{equation*}
	a_2 = \frac{1}{x_3 - x_2} \cdot \left( \frac{f_3 - f_1}{x_3 - x_1} - \frac{f_2 - f_1}{x_2 - x_1} \right),
\end{equation*}

\begin{equation*}
	\overline{x} = \frac{1}{2} \cdot \left( x_1 + x_2 - \frac{a_1}{a_2} \right).
\end{equation*}

О выборе точек.

\begin{enumerate}
	\item На первой итерации для выбора точек $x_1$, $x_2$, $x_3$ обычно достаточно использование нескольких пробных точек. Если это не получается за разумное время, можно выполнить несколько итераций метода золотого сечения до тех пор, пока пробные точки этого метода и одна из граничных точек текущего отрезка не будут удовлетворять условиям (**).
	\item На второй и последующих итерациях на отрезке $[x_1, x_3]$ рассматриваются две пробные точки $x_2$ и $\overline{x}$, для которых используется метод исключения отрезков. В новом отрезке $[x_1, x_3]$ в качестве $x_2$ выбирается та точка из $x_2$ и $\overline{x}$, которая оказалась внутри.
\end{enumerate}

Вычисления продолжаются до тех пор, пока не выполнится условие $|\overline{x} - \overline{x}'| <= \epsilon$, где $\overline{x}'$~---~приближение $x^{*}$ с предыдущей итерации.

\img{0.8\textwidth}{1.pdf}{Схема алгоритма метода парабол}

