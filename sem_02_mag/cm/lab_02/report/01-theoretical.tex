\chapter{Теоретический раздел}

Цель работы: изучение метода золотого сечения для решения задачи одномерной минимизации.

\textbf{Задание:}
\begin{enumerate}
	\item Реализовать метод золотого сечения в виде программы на ЭВМ.
	\item Провести решение задачи 
	\begin{equation*}
		\begin{cases}
			f(x) \rightarrow \min, \\
			x \in [a, b],
		\end{cases}
	\end{equation*}
	для данных индивидуального варианта.
	\item Организовать вывод на экран графика целевой функции, найденной точки минимума $(x^{*}, f(x^{*}))$ и последовательности точек $(x_i, f(x_i))$, приближающих точку исходного минимума (для последовательности точек следует предусмотреть возможность «отключения» вывода ее на экран).
\end{enumerate}

\section{Исходные данные варианта №7}

\begin{equation*}
	f(x) = arctg(x^3 - 5x + 1) + \left( \frac{x^2}{3x - 2} \right)^{\sqrt{3}}.
\end{equation*}

\begin{equation*}
	x \in [1, 2].
\end{equation*}

\section{Краткое описание метода золотого сечения}

Методы исключения отрезков основаны на следующих принципах.

\begin{enumerate}
	\item Выбираем две произвольные точки $x_1$ и $x_2$ такие, что $a < x_1 < x_2 < b$.
	\item Свойство унимодальной функции:
	если $a \le x_1 \le x_2 \le b$, то

	\begin{enumerate}
		\item если $f(x_1) \le f(x_2)$, то $x^* \in [a, x_2]$,
		\item иначе $x \in [x_1, b]$.
	\end{enumerate}

	\item Проверяем условия (a) и (b) и по результатам этой проверки отбрасываем часть отрезка $[a, b]$.
	\item Вычисления продолжаются до тех пор, пока длина текущего отрезка не станет меньше $\epsilon$~---~заданной точности.
\end{enumerate}

Способ выбора $x_1$ и $x_2$ определяет конкретный метод поиска минимума. В методе золотого сечения для уменьшения количества значений целевой функции, которые приходится вычислять в ходе реализации алгоритма, выбирают пробные точки $x_1$ и $x_2$ внутри отрезка $[a, b]$ так, чтобы при переходе к очередному отрезку одна из этих точек стала новой пробной точкой.

При этом будем считать, что отношение длины нового отрезка к длине текущего отрезка не зависит от номера итерации и равно $\tau$. Также, будем считать, что $x_1$ и $x_2$ располагаются симметрично относительно середины отрезка $[a, b]$.

\img{1\textwidth}{1.pdf}{Схема алгоритма исключения отрезков методом золотого сечения}

