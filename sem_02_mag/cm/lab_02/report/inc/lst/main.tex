# Лабораторная работа 2. Вариант 7.

function main()
  clc;

  debug = true;

  a = 1;
  b = 2;
  eps = 1e-2;

  draw_plot(a, b, eps);

  [x_min, f_min, n] = find_min(debug, a, b, eps);
  fprintf('\n\033[36mТочка минимума (x*, f(x*)) = (%f, %f), количество вычислений функции: %d.\033[0m\n', x_min, f_min, n);
end

function [x_min, f_min, n] = find_min(debug, a, b, eps)
  tau = (sqrt(5) - 1) / 2;
  l = b - a;

  x1 = b - tau*l;
  f1 = f(x1);

  if debug
    fprintf('(x0, f(x0)) = (%f, %f).\n', x1, f1);
  endif

  x2 = a + tau*l;
  f2 = f(x2);

  if debug
    fprintf('(x1, f(x1)) = (%f, %f).\n', x2, f2);
  endif

  i = 2;

  while true
    if l <= 2 * eps
      x_min = (a + b) / 2;
      f_min = f(x_min);
      n = i + 1;
      return;
    endif

    if f1 <= f2
      b = x2;
      l = b - a;

      x2 = x1;
      f2 = f1;

      x1 = b - tau*l;
      f1 = f(x1);
      i = i + 1;

      if debug
        fprintf('(x%d, f(x%d)) = (%f, %f).\n', i-1, i-1, x1, f1);
      endif
    else
      a = x1;
      l = b - a;

      x1 = x2;
      f1 = f2;

      x2 = a + tau*l;
      f2 = f(x2);
      i = i + 1;

      if debug
        fprintf('(x%d, f(x%d)) = (%f, %f).\n', i-1, i-1, x2, f2);
      endif
    endif
  endwhile
end

function draw_plot(a, b, step)
  x=a:step:b;
  y=f(x);
  plot(x,y);
end

function y = f(x)
  y = atan(x .^ 3 - 5 * x + 1) + ((x .^ 2) / (3 * x - 2)) .^ sqrt(3);
end



