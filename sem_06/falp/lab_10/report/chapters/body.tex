\setcounter{page}{3}

\chapter{Практическая часть}
\section{Задание}
Используя хвостовую рекурсию, разработать программу, позволяющую найти

\begin{enumerate}
	\item $n!$;
	\item $n$-ое число Фибонначи.
\end{enumerate}

Убедиться в правильности результатов. Для одного из вариантов ВОПРОСА и каждого задания составить таблицу, отражающую конкретный порядок работы системы. Так как резольвента хранится в виде стека, то состояние резольвенты требуется отображать в столбик: вершина~---~сверху! Новый шаг надо начинать с нового состояния резольвенты! Для одного из вариантов ВОПРОСА составить таблицу, отражающую конкретный порядок работы системы.

  
\begin{lstlisting}[label=div,caption=Реализация программы на языке Prolog]
fact1(0, Acc, Acc) :- !.
fact1(N, Acc, Res) :- N #> 0, Acc #=< Res, Res #> 0, Acc1 #= N * Acc, N1 #= N - 1, fact1(N1, Acc1, Res).

fact(N, Res) :- fact1(N, 1, Res).

fib1(0, AccA, _, AccA) :- !.
fib1(1, _, AccB, AccB) :- !.
fib1(N, AccA, AccB, Res) :- N #> 1, AccA #< Res, Res #> 0, AccB1 #= AccA + AccB, N1 #= N - 1, fib1(N1, AccB, AccB1, Res).

fib(N, Res) :- fib1(N, 0, 1, Res).
\end{lstlisting}

\begin{lstlisting}[label=div,caption=Варианты вопроса]
?- fact(5, N).
N = 120.

?- fact(N, 3628800).
N = 10.

?- fact(N, 121).
false.

?- fib(N, 55).
N = 10.

?- fib(8, N).
N = 21.

?- fib(N, 11).
false.

?- fib(0, N).
N = 0.
\end{lstlisting}


\newcommand{\specialcell}[2][c]{%
  \begin{tabular}[#1]{@{}l@{}}#2\end{tabular}}
  
\begin{table}[]
\resizebox{\textwidth}{!}{
\begin{tabular}{|l|l|l|l|}
\hline
\specialcell{№ шага} & \specialcell{Резольвента} & \specialcell{Сравниваемые термы; \\ результат; подстановки}                                                                                                                                                                                                                                                                                                                                                                                                                     & \specialcell{Дальнейшие \\ действия} \\ \hline

1   
& \specialcell{fact(3, What)} 
& \specialcell{fact(3, What) = fact1(0, Acc, Acc) \\ 
\textbf{Нет} \\ 
Подстановка пуста} 
& \specialcell{Прямой ход} \\ \hline

2
& \specialcell{fact(3, What)} 
& \specialcell{fact(3, What) = fact1(N, Acc, Res) \\ 
\textbf{Нет} \\ 
Подстановка пуста} 
& \specialcell{Прямой ход} \\ \hline

3
& \specialcell{fact1(3, 1, Res)} 
& \specialcell{fact(3, What) = fact(N, Res) \\ 
\textbf{Успех} \\ 
\textbf{N = 3} \\ 
\textbf{Res, What~---~сцепленные}
} 
& \specialcell{Прямой ход} \\ \hline

4
& \specialcell{fact1(3, 1, Res)} 
& \specialcell{fact1(3, 1, Res) = fact1(0, Acc, Acc) \\ 
\textbf{Нет} \\ 
} 
& \specialcell{Прямой ход} \\ \hline

5
& \specialcell{
N \#> 0 \\
Acc \#=< Res \\
Res \#> 0 \\
Acc1 \#= N * Acc \\
N1 \#= N - 1 \\
fact1(N1, Acc1, Res)
} 
& \specialcell{fact1(3, 1, Res) = fact1(N, Acc, Res) \\ 
\textbf{Успех} \\ 
\textbf{N = 3} \\ 
\textbf{Acc = 1} \\ 
\textbf{Res, Res~---~сцепленные}
} 
& \specialcell{Прямой ход} \\ \hline

6
& \specialcell{
Acc \#=< Res \\
Res \#> 0 \\
Acc1 \#= N * Acc \\
N1 \#= N - 1 \\
fact1(N1, Acc1, Res)
} 
& \specialcell{N \#> 0 \\ 
\textbf{Успех} \\ 
\textbf{N = 3} \\ 
\textbf{Acc = 1} \\ 
} 
& \specialcell{Прямой ход} \\ \hline

7
& \specialcell{
Res \#> 0 \\
Acc1 \#= N * Acc \\
N1 \#= N - 1 \\
fact1(N1, Acc1, Res)
} 
& \specialcell{Acc \#=< Res \\ 
\textbf{Успех} \\ 
\textbf{N = 3} \\ 
\textbf{Acc = 1} \\ 
} 
& \specialcell{Прямой ход} \\ \hline

8
& \specialcell{
Acc1 \#= N * Acc \\
N1 \#= N - 1 \\
fact1(N1, Acc1, Res)
} 
& \specialcell{Res \#> 0 \\ 
\textbf{Успех} \\ 
\textbf{N = 3} \\ 
\textbf{Acc = 1} \\ 
} 
& \specialcell{Прямой ход} \\ \hline
\end{tabular}
}
\end{table}


\newpage


\begin{table}[]
\resizebox{\textwidth}{!}{
\begin{tabular}{|l|l|l|l|}
\hline
\specialcell{№ шага} & \specialcell{Резольвента} & \specialcell{Сравниваемые термы; \\ результат; подстановки}                                                                                                                                                                                                                                                                                                                                                                                                                     & \specialcell{Дальнейшие \\ действия} \\ \hline

9
& \specialcell{
N1 \#= N - 1 \\
fact1(N1, Acc1, Res)
} 
& \specialcell{Acc1 \#= N * Acc \\ 
\textbf{Успех} \\ 
\textbf{N = 3} \\ 
\textbf{Acc = 1} \\ 
\textbf{Acc1 = 3}
} 
& \specialcell{Прямой ход} \\ \hline

10
& \specialcell{
fact1(N1, Acc1, Res)
} 
& \specialcell{N1 \#= N - 1 \\ 
\textbf{Успех} \\ 
\textbf{N = 3} \\ 
\textbf{Acc = 1} \\ 
\textbf{N1 = 2} \\
\textbf{Acc1 = 3}
} 
& \specialcell{Прямой ход} \\ \hline

11
& \specialcell{
fact1(2, 3, Res)
} 
& \specialcell{fact1(2, 3, Res) = fact1(0, Acc, Acc) \\ 
\textbf{Нет} \\ 
} 
& \specialcell{Прямой ход} \\ \hline

12
& \specialcell{
N \#> 0 \\
Acc \#=< Res \\
Res \#> 0 \\
Acc1 \#= N * Acc \\
N1 \#= N - 1 \\
fact1(N1, Acc1, Res)
} 
& \specialcell{fact1(2, 3, Res) = fact1(N, Acc, Res) \\ 
\textbf{Успех} \\
\textbf{N = 2} \\ 
\textbf{Acc = 3} \\
} 
& \specialcell{Прямой ход} \\ \hline


13
& \specialcell{
Acc \#=< Res \\
Res \#> 0 \\
Acc1 \#= N * Acc \\
N1 \#= N - 1 \\
fact1(N1, Acc1, Res)
} 
& \specialcell{N \#> 0 \\ 
\textbf{Успех} \\ 
\textbf{N = 2} \\ 
\textbf{Acc = 3} \\
} 
& \specialcell{Прямой ход} \\ \hline

14
& \specialcell{
Res \#> 0 \\
Acc1 \#= N * Acc \\
N1 \#= N - 1 \\
fact1(N1, Acc1, Res)
} 
& \specialcell{Acc \#=< Res \\ 
\textbf{Успех} \\ 
\textbf{N = 2} \\ 
\textbf{Acc = 3} \\
} 
& \specialcell{Прямой ход} \\ \hline

15
& \specialcell{
Acc1 \#= N * Acc \\
N1 \#= N - 1 \\
fact1(N1, Acc1, Res)
} 
& \specialcell{Res \#> 0 \\ 
\textbf{Успех} \\ 
\textbf{N = 2} \\ 
\textbf{Acc = 3} \\
} 
& \specialcell{Прямой ход} \\ \hline

\end{tabular}
}
\end{table}


\newpage


\begin{table}[]
\resizebox{\textwidth}{!}{
\begin{tabular}{|l|l|l|l|}
\hline
\specialcell{№ шага} & \specialcell{Резольвента} & \specialcell{Сравниваемые термы; \\ результат; подстановки}                                                                                                                                                                                                                                                                                                                                                                                                                     & \specialcell{Дальнейшие \\ действия} \\ \hline

16
& \specialcell{
N1 \#= N - 1 \\
fact1(N1, Acc1, Res)
} 
& \specialcell{Acc1 \#= N * Acc \\ 
\textbf{Успех} \\ 
\textbf{N = 2} \\ 
\textbf{Acc = 3} \\
\textbf{Acc1 = 6} \\ 
} 
& \specialcell{Прямой ход} \\ \hline


17
& \specialcell{
fact1(N1, Acc1, Res)
} 
& \specialcell{N1 \#= N - 1 \\ 
\textbf{Успех} \\ 
\textbf{N = 2} \\ 
\textbf{Acc = 3} \\
\textbf{N1 = 1} \\ 
\textbf{Acc1 = 6} \\ 
} 
& \specialcell{Прямой ход} \\ \hline

18
& \specialcell{
fact1(1, 6, Res)
} 
& \specialcell{fact1(1, 6, Res) = fact1(0, Acc, Acc) \\ 
\textbf{Нет} \\ 
} 
& \specialcell{Прямой ход} \\ \hline

19
& \specialcell{
N \#> 0 \\
Acc \#=< Res \\
Res \#> 0 \\
Acc1 \#= N * Acc \\
N1 \#= N - 1 \\
fact1(N1, Acc1, Res)
} 
& \specialcell{fact1(1, 6, Res) = fact1(N, Acc, Res) \\ 
\textbf{Успех} \\ 
\textbf{N = 1} \\ 
\textbf{Acc = 6} \\ 
} 
& \specialcell{Прямой ход} \\ \hline

20
& \specialcell{
Acc \#=< Res \\
Res \#> 0 \\
Acc1 \#= N * Acc \\
N1 \#= N - 1 \\
fact1(N1, Acc1, Res)
} 
& \specialcell{N \#> 0 \\ 
\textbf{Успех} \\ 
\textbf{N = 1} \\ 
\textbf{Acc = 6} \\ 
} 
& \specialcell{Прямой ход} \\ \hline

21
& \specialcell{
Res \#> 0 \\
Acc1 \#= N * Acc \\
N1 \#= N - 1 \\
fact1(N1, Acc1, Res)
} 
& \specialcell{Acc \#=< Res \\ 
\textbf{Успех} \\ 
\textbf{N = 1} \\ 
\textbf{Acc = 6} \\ 
} 
& \specialcell{Прямой ход} \\ \hline

22
& \specialcell{
Acc1 \#= N * Acc \\
N1 \#= N - 1 \\
fact1(N1, Acc1, Res)
} 
& \specialcell{Res \#> 0 \\ 
\textbf{Успех} \\ 
\textbf{N = 1} \\ 
\textbf{Acc = 6} \\ 
} 
& \specialcell{Прямой ход} \\ \hline

\end{tabular}
}
\end{table}


\newpage


\begin{table}[]
\resizebox{\textwidth}{!}{
\begin{tabular}{|l|l|l|l|}
\hline
\specialcell{№ шага} & \specialcell{Резольвента} & \specialcell{Сравниваемые термы; \\ результат; подстановки}                                                                                                                                                                                                                                                                                                                                                                                                                     & \specialcell{Дальнейшие \\ действия} \\ \hline

23
& \specialcell{
N1 \#= N - 1 \\
fact1(N1, Acc1, Res)
} 
& \specialcell{Acc1 \#= N * Acc \\ 
\textbf{Успех} \\ 
\textbf{N = 1} \\ 
\textbf{Acc = 6} \\ 
\textbf{Acc1 = 6} \\ 
} 
& \specialcell{Прямой ход} \\ \hline

24
& \specialcell{
fact1(0, 6, Res)
} 
& \specialcell{N1 \#= N - 1 \\ 
\textbf{Успех} \\ 
\textbf{N = 1} \\ 
\textbf{Acc = 6} \\ 
\textbf{N1 = 0} \\ 
\textbf{Acc1 = 6} \\ 
} 
& \specialcell{Прямой ход} \\ \hline

25
& \specialcell{
fact1(0, 6, Res)
} 
& \specialcell{fact1(0, 6, Res) = fact1(0, Acc, Acc) \\ 
\textbf{Успех} \\ 
\textbf{Acc = 6} \\ 
\textbf{Res = 6} \\ 
} 
& \specialcell{Откат к 19} \\ \hline

26-31
& \specialcell{
fact1(1, 6, Res)
} 
& \specialcell{... \\ 
\textbf{Нет} \\ 
\textbf{N = 1} \\ 
\textbf{Acc = 6} \\ 
} 
& \specialcell{Прямой ход} \\ \hline

32
& \specialcell{
fact1(1, 6, Res)
} 
& \specialcell{
\textbf{Конец базы знаний} \\ 
} 
& \specialcell{Откат к 12} \\ \hline

23-38
& \specialcell{
fact1(2, 3, Res)
} 
& \specialcell{... \\ 
\textbf{Нет} \\ 
\textbf{N = 2} \\ 
\textbf{Acc = 3} \\ 
} 
& \specialcell{Прямой ход} \\ \hline

39
& \specialcell{
fact1(2, 3, Res)
} 
& \specialcell{
\textbf{Конец базы знаний} \\ 
} 
& \specialcell{Откат к 5} \\ \hline

40-45
& \specialcell{
fact1(3, 1, Res)
} 
& \specialcell{... \\ 
\textbf{Нет} \\ 
\textbf{N = 3} \\ 
\textbf{Acc = 1} \\ 
} 
& \specialcell{Прямой ход} \\ \hline



\end{tabular}
}
\end{table}

\begin{table}[]
\resizebox{\textwidth}{!}{
\begin{tabular}{|l|l|l|l|}
\hline
\specialcell{№ шага} & \specialcell{Резольвента} & \specialcell{Сравниваемые термы; \\ результат; подстановки}                                                                                                                                                                                                                                                                                                                                                                                                                     & \specialcell{Дальнейшие \\ действия} \\ \hline

46
& \specialcell{
fact1(3, 1, Res)
} 
& \specialcell{
\textbf{Конец базы знаний} \\ 
} 
& \specialcell{Откат к 3} \\ \hline


47-51
& \specialcell{
fact(3, What)
} 
& \specialcell{... \\ 
\textbf{Нет} \\ 
Подстановка пуста
} 
& \specialcell{Прямой ход} \\ \hline

52
& \specialcell{
fact(3, What)
} 
& \specialcell{
\textbf{Конец базы знаний}
} 
& \specialcell{Конец базы знаний} \\ \hline

53
& \specialcell{
Резольвента пуста
} 
& \specialcell{
Подстановка пуста
} 
& \specialcell{Конец базы знаний} \\ \hline


\end{tabular}
}
\end{table}

\newpage

\begin{table}[]
\resizebox{\textwidth}{!}{
\begin{tabular}{|l|l|l|l|}
\hline
\specialcell{№ шага} & \specialcell{Резольвента} & \specialcell{Сравниваемые термы; \\ результат; подстановки}                                                                                                                                                                                                                                                                                                                                                                                                                     & \specialcell{Дальнейшие \\ действия} \\ \hline

1
& \specialcell{
fib(2, What)
} 
& \specialcell{fib(2, What) = fact1(0, Acc, Acc) \\
\textbf{Нет} \\ 
Подстановка пуста
} 
& \specialcell{Прямой ход} \\ \hline


2-6
& \specialcell{
fib(2, What)
} 
& \specialcell{... \\
\textbf{Нет} \\ 
Подстановка пуста
} 
& \specialcell{Прямой ход} \\ \hline


7
& \specialcell{
fib1(2, 0, 1, Res)
} 
& \specialcell{fib(2, What) = fib(N, Res) \\
\textbf{Успех} \\ 
\textbf{N = 2} \\ 
\textbf{Res, What - сцепленные} \\ 
} 
& \specialcell{Прямой ход} \\ \hline

8
& \specialcell{
fib1(2, 0, 1, Res)
} 
& \specialcell{fib1(2, 0, 1, Res) = fact1(0, Acc, Acc) \\
\textbf{Нет} \\ 
\textbf{N = 2} \\ 
\textbf{Res, What - сцепленные} \\ 
} 
& \specialcell{Прямой ход} \\ \hline

9-12
& \specialcell{
fib1(2, 0, 1, Res)
} 
& \specialcell{... \\
\textbf{Нет} \\ 
\textbf{N = 2} \\ 
\textbf{Res, What - сцепленные} \\ 
} 
& \specialcell{Прямой ход} \\ \hline

13
& \specialcell{
N \#> 1 \\
AccA \#< Res \\
Res \#> 0 \\
AccB1 \#= AccA + AccB \\
N1 \#= N - 1 \\
fib1(N1, AccB, AccB1, Res) \\
} 
& \specialcell{fib1(2, 0, 1, Res) = fib1(N, AccA, AccB, Res) \\
\textbf{Успех} \\ 
\textbf{N = 2} \\ 
\textbf{AccA = 0} \\ 
\textbf{AccB = 1} \\ 
\textbf{Res, Res - сцепленные} \\ 
} 
& \specialcell{Прямой ход} \\ \hline

14
& \specialcell{
AccA \#< Res \\
Res \#> 0 \\
AccB1 \#= AccA + AccB \\
N1 \#= N - 1 \\
fib1(N1, AccB, AccB1, Res) \\
} 
& \specialcell{N \#> 1 \\
\textbf{Успех} \\ 
\textbf{N = 2} \\ 
\textbf{AccA = 0} \\ 
\textbf{AccB = 1} \\ 
\textbf{Res, Res - сцепленные} \\ 
} 
& \specialcell{Прямой ход} \\ \hline

15
& \specialcell{
Res \#> 0 \\
AccB1 \#= AccA + AccB \\
N1 \#= N - 1 \\
fib1(N1, AccB, AccB1, Res) \\
} 
& \specialcell{AccA \#< Res \\
\textbf{Успех} \\ 
\textbf{N = 2} \\ 
\textbf{AccA = 0} \\ 
\textbf{AccB = 1} \\ 
\textbf{Res, Res - сцепленные} \\ 
} 
& \specialcell{Прямой ход} \\ \hline

\end{tabular}
}
\end{table}


\begin{table}[]
\resizebox{\textwidth}{!}{
\begin{tabular}{|l|l|l|l|}
\hline
\specialcell{№ шага} & \specialcell{Резольвента} & \specialcell{Сравниваемые термы; \\ результат; подстановки}                                                                                                                                                                                                                                                                                                                                                                                                                     & \specialcell{Дальнейшие \\ действия} \\ \hline

16
& \specialcell{
AccB1 \#= AccA + AccB \\
N1 \#= N - 1 \\
fib1(N1, AccB, AccB1, Res) \\
} 
& \specialcell{Res \#> 0 \\
\textbf{Успех} \\ 
\textbf{N = 2} \\ 
\textbf{AccA = 0} \\ 
\textbf{AccB = 1} \\ 
\textbf{Res, Res - сцепленные} \\ 
} 
& \specialcell{Прямой ход} \\ \hline

17
& \specialcell{
N1 \#= N - 1 \\
fib1(N1, AccB, AccB1, Res) \\
} 
& \specialcell{AccB1 \#= AccA + AccB \\
\textbf{Успех} \\ 
\textbf{N = 2} \\ 
\textbf{AccA = 0} \\ 
\textbf{AccB = 1} \\ 
\textbf{AccB1 = 1} \\ 
\textbf{Res, Res - сцепленные} \\ 
} 
& \specialcell{Прямой ход} \\ \hline

18
& \specialcell{
fib1(N1, AccB, AccB1, Res) \\
} 
& \specialcell{N1 \#= N - 1 \\
\textbf{Успех} \\ 
\textbf{N = 2} \\ 
\textbf{AccA = 0} \\ 
\textbf{AccB = 1} \\ 
\textbf{AccB1 = 1} \\ 
\textbf{N1 = 1} \\ 
\textbf{Res, Res - сцепленные} \\ 
} 
& \specialcell{Прямой ход} \\ \hline

19
& \specialcell{
fib1(1, 1, 1, Res)
} 
& \specialcell{fib1(1, 1, 1, Res) = fact1(0, Acc, Acc) \\
\textbf{Нет} \\ 
} 
& \specialcell{Прямой ход} \\ \hline

20-22
& \specialcell{
fib1(1, 1, 1, Res)
} 
& \specialcell{... \\
\textbf{Нет} \\ 
} 
& \specialcell{Прямой ход} \\ \hline

23
& \specialcell{
fib1(2, 0, 1, Res)
} 
& \specialcell{fib1(1, 1, 1, Res) = fib1(1, \_, AccB, AccB) \\
\textbf{Успех} \\ 
\textbf{AccB = 1} \\ 
\textbf{Res = 1} \\ 
} 
& \specialcell{Откат к 13} \\ \hline

24
& \specialcell{
fib1(2, 0, 1, Res)
} 
& \specialcell{fib1(2, 0, 1, Res) = fib(N, Res) \\
\textbf{Нет} \\ 
\textbf{N = 2} \\ 
\textbf{AccA = 0} \\ 
\textbf{AccB = 1} \\ 
\textbf{Res, Res - сцепленные} \\ 
} 
& \specialcell{Прямой ход} \\ \hline

25
& \specialcell{
fib(2, What)
} 
& \specialcell{\textbf{Конец базы знаний} \\
} 
& \specialcell{Откат к 7} \\ \hline

26
& \specialcell{
fib(2, What)
} 
& \specialcell{\textbf{Конец базы знаний} \\
} 
& \specialcell{Конец базы знаний} \\ \hline

27
& \specialcell{
Резольвента пуста
} 
& \specialcell{Подстановка пуста \\
} 
& \specialcell{Конец базы знаний} \\ \hline



\end{tabular}
}
\end{table}