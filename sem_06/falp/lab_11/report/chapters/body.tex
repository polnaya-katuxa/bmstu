\setcounter{page}{3}

\chapter{Практическая часть}
\section{Задание}
Используя хвостовую рекурсию, разработать (комментируя назначение
аргументов) эффективную программу , позволяющую:
\begin{enumerate}
	\item Найти длину списка (по верхнему уровню);
	\item Найти сумму элементов числового списка;
	\item Найти сумму элементов числового списка, стоящих на нечетных позициях исходного списка (нумерация от 0);
	\item Сформировать список из элементов числового списка, больших заданного значения;
	\item Удалить заданный элемент из списка (один или все вхождения);
	\item Объединить два списка.
\end{enumerate}

Убедиться в правильности результатов.
Для одного из вариантов ВОПРОСА уметь составить таблицу, отражающую конкретный порядок работы системы
  
\begin{code}
\caption{Задания №1-6}
\label{code:bf1}
\begin{minted}{lisp}
domains 
 list = integer*.

predicates
 nondeterm length1(list, integer, integer).
 nondeterm length(list, integer).
 
 nondeterm sum1(list, integer, integer).
 nondeterm sum(list, integer).
 
 nondeterm sum_odd_pos1(list, integer, integer, integer).
 nondeterm sum_odd_pos(list, integer).
 nondeterm sum_odd_pos2(list, integer, integer).
 nondeterm sum_odd_pos_new(list, integer).
\end{minted}
\end{code}

\newpage

\begin{code}
\caption{Задание №1-2}
\label{code:bf2}
\begin{minted}{lisp}		
 nondeterm list_of_bigger(list, integer, list).
 
 nondeterm del_all(list, integer, list).
 nondeterm del_single(list, integer, list).
 
 nondeterm union(list, list, list).
 
clauses
 length1([_ | T], Acc, Res) :- !, Acc1 = Acc + 1, length1(T, Acc1, Res).
 length1([], Acc, Acc) :- !.
 length(L, Res) :- !, length1(L, 0, Res).
 
 sum1([], Acc, Acc) :- !.
 sum1([H | T], Acc, Res) :- !, Acc1 = Acc + H, sum1(T, Acc1, Res).
 sum(L, Res) :- !, sum1(L, 0, Res).
 
 sum_odd_pos1([], _, Acc, Acc) :- !.
 sum_odd_pos1([_ | T], Pos, Acc, Res) :- Pos mod 2 = 0, !, Pos1 = Pos + 1,
  sum_odd_pos1(T, Pos1, Acc, Res).
 sum_odd_pos1([H | T], Pos, Acc, Res) :- !, Pos1 = Pos + 1,
  Acc1 = Acc + H, sum_odd_pos1(T, Pos1, Acc1, Res).
 sum_odd_pos(L, Res) :- !, sum_odd_pos1(L, 0, 0, Res).
 
 sum_odd_pos2([], Acc, Res) :- Res = Acc, !.
 sum_odd_pos2([_ | [H | T]], Acc, Res) :- !, Acc1 = Acc + H,
  sum_odd_pos2(T, Acc1, Res).
 sum_odd_pos_new(L, Res) :- !, sum_odd_pos2(L, 0, Res).
 
 list_of_bigger([], _, []) :- !.
 list_of_bigger([H | T], N, [H | ResT]) :- H > N, !,
  list_of_bigger(T, N, ResT).
 list_of_bigger([_ | T], N, Res) :- !, list_of_bigger(T, N, Res).
\end{minted}
\end{code}

\newpage

\begin{code}
\caption{Задание №1-2}
\label{code:bf3}
\begin{minted}{lisp}
 del_all([], _, []) :- !.
 del_all([H | T], N, Res) :- H = N, !, del_all(T, N, Res).
 del_all([H | T], N, [H | ResT]) :- !, del_all(T, N, ResT).
 
 del_single([], _, []) :- !.
 del_single([H | T], N, T) :- H = N, !.
 del_single([H | T], N, [H | ResT]) :- !, del_single(T, N, ResT).
 
 union([], [], []) :- !.
 union([H1 | T1], [], [H1 | ResT]) :- !, union(T1, [], ResT).
 union([], [H2 | T2], [H2 | ResT]) :- !, union([], T2, ResT).
 union([H1 | T1], [H2 | T2], [H1, H2 | ResT]) :- !, union(T1, T2, ResT).
 
 union_new([], L, L) :- !.
 union_new([H | T], L, [H | ResT]) :- !, union(T, L, ResT).

goal
 %length([1, 2, 3], Is).
 %sum([1, 2, 3], Is).
 %sum_odd_pos([1, 0, 2, 0, 5, 3], Is).
 %sum_odd_pos_new([1, 0, 2, 0, 5, 3], Is).
 %list_of_bigger([1, 0, 2, 0, 5, 3], 2, Is).
 %del_single([1, 0, 2, 0, 5, 2, 6], 2, Is).
 %del_all([1, 0, 2, 0, 5, 2, 6], 2, Is).
 %union([1, 2, 3], [1, 0, 2, 0, 5, 2, 6], Is).
 union_new([1, 2, 3], [1, 0, 2, 0, 5, 2, 6], Is).
\end{minted}
\end{code}

\newcommand{\specialcell}[2][c]{%
  \begin{tabular}[#1]{@{}l@{}}#2\end{tabular}}
  
\begin{table}[]
\resizebox{\textwidth}{!}{
\begin{tabular}{|l|l|l|l|}
\hline
\specialcell{№ шага} & \specialcell{Резольвента} & \specialcell{Сравниваемые термы; \\ результат; подстановки}                                                                                                                                                                                                                                                                                                                                                                                                                     & \specialcell{Дальнейшие \\ действия} \\ \hline

1   
& \specialcell{length([1, 2], Is)} 
& \specialcell{length([1, 2], Is) = \\ length1([\_ | T], Acc, Res) \\ 
\textbf{Нет} \\ 
Подстановка пуста} 
& \specialcell{Прямой ход} \\ \hline

2   
& \specialcell{...} 
&  
&  \\ \hline

3   
& \specialcell{! \\ length1([1, 2], 0, Res)} 
& \specialcell{length([1, 2], Is) = \\ length(L, Res) \\ 
\textbf{Успех} \\ 
\textbf{L = [1, 2]} \\ 
\textbf{Is и Res - сцепленные}} 
& \specialcell{Прямой ход} \\ \hline

4   
& \specialcell{!\\Acc1 = 0 + 1\\length1([2], Acc1, Res)} 
& \specialcell{length1([1, 2], 0, Res) = \\ length1([\_ | T], Acc, Res) \\ 
\textbf{Успех} \\ 
\textbf{Acc = 0} \\
\textbf{T = [2]} \\ 
\textbf{L = [1, 2]} \\ 
\textbf{Is, Res и Res - сцепленные}}
& \specialcell{Прямой ход} \\ \hline

5  
& \specialcell{length1([2], 1, Res)} 
& \specialcell{Acc1 = 0 + 1 \\ 
\textbf{Успех} \\ 
\textbf{Acc1 = 1} \\
\textbf{Acc = 0} \\
\textbf{T = [2]} \\ 
\textbf{L = [1, 2]} \\ 
\textbf{Is, Res и Res - сцепленные}}
& \specialcell{Прямой ход} \\ \hline

6  
& \specialcell{!\\Acc1 = 1 + 1\\length1([], Acc1, Res)} 
& \specialcell{length1([2], Acc1, Res) = \\ length1([\_ | T], Acc, Res) \\ 
\textbf{Успех} \\ 
\textbf{Acc = 1} \\
\textbf{T = []} \\ 
\textbf{Acc1 = 1} \\
\textbf{Acc = 0} \\
\textbf{T = [2]} \\ 
\textbf{L = [1, 2]} \\ 
\textbf{Is, Res, Res и Res - сцепленные}}
& \specialcell{Прямой ход} \\ \hline

\end{tabular}
}
\end{table}

\newpage

\begin{table}[]
\resizebox{\textwidth}{!}{%
\begin{tabular}{|l|l|l|l|}
\hline
\specialcell{№ шага} & \specialcell{Резольвента} & \specialcell{Сравниваемые термы; \\ результат; подстановки}                                                                                                                                                                                                                                                                                                                                                                                                                     & \specialcell{Дальнейшие \\ действия} \\ \hline

7  
& \specialcell{length1([], 2, Res)} 
& \specialcell{Acc1 = 1 + 1 \\ 
\textbf{Успех} \\ 
\textbf{Acc1 = 2} \\
\textbf{Acc = 1} \\
\textbf{T = []} \\ 
\textbf{Acc1 = 1} \\
\textbf{Acc = 0} \\
\textbf{T = [2]} \\ 
\textbf{L = [1, 2]} \\ 
\textbf{Is, Res, Res и Res - сцепленные}}
& \specialcell{Прямой ход} \\ \hline

8  
& \specialcell{length1([], 2, Res)} 
& \specialcell{length1([], 2, Res) = \\ length1([\_ | T], Acc, Res) \\ 
\textbf{Нет} \\ 
\textbf{Acc1 = 2} \\
\textbf{Acc = 1} \\
\textbf{T = []} \\ 
\textbf{Acc1 = 1} \\
\textbf{Acc = 0} \\
\textbf{T = [2]} \\ 
\textbf{L = [1, 2]} \\ 
\textbf{Is, Res, Res и Res - сцепленные}}
& \specialcell{Прямой ход} \\ \hline

9  
& \specialcell{!} 
& \specialcell{length1([], 2, Res) = \\ length1([], Acc, Acc) \\ 
\textbf{Успех} \\ 
\textbf{Res = Acc = 2} \\
\textbf{Acc = 2} \\
\textbf{Acc1 = 2} \\
\textbf{Acc = 1} \\
\textbf{T = []} \\ 
\textbf{Acc1 = 1} \\
\textbf{Acc = 0} \\
\textbf{T = [2]} \\ 
\textbf{L = [1, 2]} \\ 
\textbf{Is, Res, Res и Res - сцепленные}}
& \specialcell{Откат к пункту 6} \\ \hline

\end{tabular}
}
\end{table}

\newpage


\begin{table}[]
\resizebox{\textwidth}{!}{
\begin{tabular}{|l|l|l|l|}
\hline
\specialcell{№ шага} & \specialcell{Резольвента} & \specialcell{Сравниваемые термы; \\ результат; подстановки}                                                                                                                                                                                                                                                                                                                                                                                                                     & \specialcell{Дальнейшие \\ действия} \\ \hline

10  
& \specialcell{Acc1 = 1 + 1\\length1([], Acc1, Res)} 
& \specialcell{! (\textbf{ложь}) \\ 
\textbf{Нет} \\  
\textbf{Acc1 = 1} \\
\textbf{Acc = 0} \\
\textbf{T = [2]} \\ 
\textbf{L = [1, 2]} \\ 
\textbf{Is, Res, Res и Res - сцепленные}}
& \specialcell{Откат к пункту 4} \\ \hline

11 
& \specialcell{Acc1 = 0 + 1\\length1([2], Acc1, Res)} 
& \specialcell{! (\textbf{ложь}) \\ 
\textbf{Нет} \\  
\textbf{L = [1, 2]} \\ 
\textbf{Is, Res и Res - сцепленные}}
& \specialcell{Откат к пункту 3} \\ \hline

12 
& \specialcell{length1([1, 2], 0, Res)} 
& \specialcell{! (\textbf{ложь}) \\ 
\textbf{Нет}\\
Подстановка пуста} 
& \specialcell{Завершение работы} \\ \hline

13   
& \specialcell{Пусто} 
& \specialcell{Подстановка пуста} 
& \specialcell{Завершение работы} \\ \hline

\end{tabular}
}
\end{table}
