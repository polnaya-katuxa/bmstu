\setcounter{page}{3}

\chapter{Практическая часть}
\section{Задание №1}
Запустить среду Visual Prolog 5.2. Настроить утилиту TestGoal.
Запустить тестовую программу, проанализировать реакцию системы и множество ответов.
Разработать свою программу~---~«Телефонный справочник». Абоненты могут иметь несколько телефонов. Протестировать работу программы, используя разные вопросы.

\begin{itemize}
 \item «Телефонный справочник»: фамилия, номер телефона, адрес~---~структура (город, улица, номер дома, номер квартиры).
 \item «Автомобили»: фамилия владельца, марка, цвет, стоимость, номер.
\end{itemize}

Владелец может иметь несколько телефонов, автомобилей (Факты). В разных городах есть однофамильцы, в одном городе~---~фамилия уникальна.
Используя конъюнктивное правило и простой вопрос, обеспечить возможность поиска:

\begin{itemize}
 \item По Марке и Цвету автомобиля найти Фамилию, Город, Телефон. Лишней информации не находить и не передавать!!!
\end{itemize}
  
\begin{code}
\caption{Задание №1}
\label{code:bf2}
\begin{minted}{lisp}
domains
 name, phone, city, street = symbol.
        building, flat = integer.
        address = address(city, street, building, flat).
 model, color, number = symbol.
 price = integer.

predicates
 nondeterm phone_note(name, phone, address).
 nondeterm car_note(name, model, color, price, number).
 nondeterm note_by_car(model, color, name, phone, city).

clauses
 phone_note("Eliza", "111111", address("New-York", "One", 1, 2)).
 phone_note("Dabby", "222222", address("London", "Two", 3, 4)).
 phone_note("Eliza", "333333", address("Paris", "Three", 5, 6)).
 phone_note("Donnie", "444444", address("Oslo", "Four", 7, 8)).
        phone_note("Darwin", "555555", address("Minsk", "Five", 7, 8)).
 
 car_note("Eliza", "BMW", "black", 1000, "AAA111").
 car_note("Donnie", "Ford", "yellow", 1500, "BBB222").
 car_note("Donnie", "Mercedes", "pink", 2000, "CCC333").
 car_note("Darwin", "BMW", "black", 1000, "DDD444").
        car_note("Dabby", "Mercedes", "white", 2000, "EEE555").
        car_note("Dabby", "Lada", "black", 500, "FFF666").
        car_note("Dabby", "Ford", "pink", 1500, "GGG777").
 
 note_by_car(Model, Color, Name, Phone, City) :- car_note(Name, Model, 
 Color, _, _), phone_note(Name, Phone, address(City, _, _, _)).
 
goal
 %car_note("Eliza", Model, Color, Price, Number).
 %phone__note(Name, Phone, address("Oslo", Street, House, Room)).
 note_by_car("BMW", "black", Name, Phone, City).
\end{minted}
\end{code}

\section{Вопросы}

\subsection{Что собой представляет программа «Телефонный справочник» на Prolog, какова ее структура. }

Программа <<Телефонный справочник>> представляет собой набор фактов и правило. Она состоит из четырех разделов:
\begin{itemize}
	\item domains --- раздел описания доменов;
	\item predicates --- раздел описания предикатов;
	\item clauses --- раздел описания предложений базы знаний;
	\item goal --- раздел описания внутренней цели (вопроса).
\end{itemize}

Предикат --- логическая функция (отображение множества значений на аргументы \{"да", "нет"\}).
Домен --- пользовательский тип данных.

\subsection{Как она реализуется, как формируются результаты работы программы.}

Программа реализуется с помощью описания базы знаний и задания вопроса.

В процессе выполнения программы система, используя базу знаний, описанную в разделе clauses, будет пытаться найти такие значения переменных, при которых на поставленный вопрос можно ответить "да".
