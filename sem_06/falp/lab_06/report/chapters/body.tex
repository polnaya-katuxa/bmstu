\setcounter{page}{3}

\chapter{Практическая часть}
\section{Задание №1}
Написать хвостовую рекурсивную функцию my-reverse, которая развернет верхний уровень своего списка-аргумента lst.
  
\begin{code}
\caption{Задание №1}
\label{code:bf2}
\begin{minted}{lisp}
(defun get_reversed (l r) 
	(cond
		((null l) r)
		(t (get_reversed (cdr l) (cons (car l) r)))  
	)
)

(defun my_reverse (l)
	(get_reversed l nil)
)

; ИЛИ

(defun my_reverse (l &optional (r nil)) 
	(cond
		((null l) r)
		(t (get_reversed (cdr l) (cons (car l) r)))  
	)
)
\end{minted}
\end{code}

\section{Задание №2}
Написать функцию, которая возвращает первый элемент списка - аргумента, который сам является непустым списком.

\begin{code}
\caption{Задание №2}
\label{code:bf2}
\begin{minted}{lisp}
(defun get_inside (l)
	(
		cond ((null l) nil)
		((atom (car l)) (car l))
		(t (get_inside (car l)))
	)
)

(defun get_first_atom (l)
	(
		cond ((null l) nil)
		((atom (car l)) (get_first_atom (cdr l)))
		(t (get_inside (car l))
	)
))
\end{minted}
\end{code}

\section{Задание №3}
Напишите рекурсивную функцию, которая умножает на заданное число-аргумент все числа из заданного списка-аргумента, когда:
\begin{itemize}
	\item все элементы списка~---~числа;
	\item элементы списка~---~любые объекты.
\end{itemize}

\begin{code}
\caption{Задание №3}
\label{code:bf2}
\begin{minted}{lisp}
(defun mul_numbers1 (l num acc)
(	cond ((null l) acc)
	(t  (mul_numbers1 (cdr l) num (cons (* (car l) num) acc)))
))

(defun my_mul1 (l num)
	(mul_numbers1 l num nil))
\end{minted}
\end{code}

\newpage

\begin{code}
\caption{Задание №3}
\label{code:bf2}
\begin{minted}{lisp}
(defun mul_numbers2 (l num acc)
(	cond ((null l) acc)
	((numberp (car l)) (mul_numbers2 (cdr l) num 
		(cons (* (car l) num) acc)))
	((atom (car l)) (mul_numbers2 (cdr l) num (cons (car l) acc)))
	(t (mul_numbers2 (cdr l) num (cons (mul_numbers2 (car l) num nil) 
		acc)))
))

(defun my_mul2 (l num)
	(mul_numbers2 l num nil)
)
\end{minted}
\end{code}

\section{Задание №4}
Напишите функцию, select-between, которая из списка-аргумента, содержащего только числа, выбирает только те, которые расположены между двумя указанными границами- аргументами и возвращает их в виде списка.

\begin{code}
\caption{Задание №4}
\label{code:bf3}
\begin{minted}{lisp}
(defun select_between1 (lst l r acc)
	(cond ((null lst) acc)
	((or (< l (car lst) r) (< r (car lst) l)) 
		(select_between1 (cdr lst) l r (cons (car lst) acc)))
	(t (select_between1 (cdr lst) l r acc))
))

(defun select_between (lst l r)
	(select_between1 lst l r nil))
\end{minted}
\end{code}

\section{Задание №5}
Написать рекурсивную версию (с именем rec-add) вычисления суммы чисел заданного списка:
\begin{itemize}
	\item одноуровнего смешанного;
	\item структурированного.
\end{itemize}

\begin{code}
\caption{Задание №5}
\label{code:bf5}
\begin{minted}{lisp}
(defun add (lst acc)
(	cond ((null lst) acc)
	((numberp (car lst)) (add (cdr lst) (+ (car lst) acc)))
	(t (add (cdr lst) acc))
))

(defun rec_add (lst) (add lst 0))
\end{minted}
\end{code}

\begin{code}
\caption{Задание №5}
\label{code:bf5}
\begin{minted}{lisp}
(defun rec_add (lst)
(	cond ((null lst) 0)
	((numberp (car lst)) (+ (car lst) (rec_add (cdr lst))))
	((atom (car lst)) (rec_add (cdr lst)))
	(t (+ (rec_add (car lst)) (rec_add (cdr lst))))
))
\end{minted}
\end{code}
