\setcounter{page}{3}
\chapter{Теоретическая часть}
\section{Базис языка Lisp.}
Базис~---~это минимальный набор правил/конструкций языка, к которым могут быть сведены все остальные. Базис языка Lisp представлен атомами, структурами, базовыми функциями, базовыми функционалами. Некоторые базисные функции: $car$, $cdr$, $cons$, $quote$, $eq$, $eval$, $apply$, $funcall$.

\section{Классификация функций языка Lisp.}
Среди функций в языке Lisp выделяют бызисные, пользовательские и функции ядра. Также, по реализации функции можно разделить на:
\begin{itemize}
	\item чистые (не создающие побочных эффектов, принимающие фиксированное число аргументов, не получающие данные неявно, результат работы которых не зависит от внешних переменных);
	\item особые, или формы;
	\item функции более высоких порядков, или функционалы (функции, результатом и/или аргументом которых является функция).
\end{itemize}

\section{Способы создания функций в языке Lisp.}
Функцию можно определить двумя способами: неименованную с помощью $lambda$ и именованную с помощью $defun$.

\begin{equation}
	\nonumber (lambda \quad (x_1 \quad x_2 \quad ... \quad x_n) \quad f),
\end{equation}
где $f$~---~тело функции, $x_i, i = \overline{1, n}$~---~формальные параметры.

\begin{equation}
	\nonumber (defun \quad <\text{имя}> \quad [lambda] \quad (x_1 \quad x_2 \quad ... \quad x_n) \quad f),
\end{equation}
где $f$~---~тело функции, $x_i, i = \overline{1, n}$~---~формальные параметры. Тогда имя будет ссылкой на описание функции.

\section{Функции $cond$, $if$, $and$, $or$.}
Функции $cond$, $if$, $and$, $or$ являются основными условными функциями в $Lisp$.

\subsection{\texttt{cond}}
Форма $cond$ содержит некоторое (возможно нулевое) количество подвыражений, которые является списками форм. Каждое подвыражение содержит форму условия и ноль и более форм для выполнения. Например:

\begin{code}
\begin{minted}{lisp}
(cond (condition-1 expression-1-1 expression-1-2 ...) 
      (condition-2)
      (condition-3 expression-3-1 ...)
      ... )
\end{minted}
\end{code}

$cond$ обрабатывает свои подвыражения слева направо. Для каждого подвыражения, вычисляется форма условия. Если результат $nil$, $cond$ переходит к следующему подвыражению. Если результат $t$, $cdr$ подвыражения обрабатывается, как список форм. После выполнения списка форм, $cond$ возвращает управление без обработки оставшихся подвыражений. Оператор $cond$ возвращает результат выполнения последней формы из списка. Если этот список пустой, тогда возвращается значение формы условия. Если $cond$ вернула управление без вычисления какой-либо ветки (все условные формы вычислялись в $nil$), возвращается значение $nil$.

\subsection{$if$}
Оператор $if$ обозначает то же, что и конструкция $if-then-else$ в большинстве других языков программирования. Сначала выполняется форма $condition$. Если результат не равен $nil$, тогда выбирается форма $then$. Иначе выбирается форма $else$. Выбранная ранее форма выполняется, и $if$ возвращает то, что вернула это форма.

\begin{code}
\begin{minted}{lisp}
(if condition then else)
\end{minted}
\end{code}

\subsection{$and$}
$and$ последовательно слева направо вычисляет формы. Если какая-либо форма $expressionN$ вычислилась в $nil$, тогда немедленно возвращается значение $nil$ без выполнения оставшихся форм. Если все формы кроме последней вычисляются в не-$nil$ значение, and возвращает то, что вернула последняя форма. Таким образом, and может использоваться, как для логических операций, где $nil$ обозначает ложь, и не-$nil$~---~истину, так и для условных выражений.

\begin{code}
\begin{minted}{lisp}
(and expression1 expression2 ... )
\end{minted}
\end{code}

\subsection{$or$}
$or$ последовательно выполняет каждую форму слева направо. Если какая-либо непоследняя форма выполняется в что-либо отличное от $nil$, or немедленно возвращает это не-$nil$ значение без выполнения оставшихся форм. Если все формы кроме последней, вычисляются в $nil$, $or$ возвращает то, что вернула последняя форма. Таким образом $or$ может быть использована как для логических операций, в который $nil$ обозначает ложь, и не-$nil$~---~истину, так и для условного выполнения форм.

\begin{code}
\begin{minted}{lisp}
(or expression1 expression2 ... )
\end{minted}
\end{code}

\newpage

\chapter{Практическая часть}
\section{Задание №1}
Написать функцию, которая принимает целое число и возвращает первое четное число, не меньшее аргумента.

\begin{code}
\caption{Задание №1}
\label{code:bf2}
\begin{minted}{lisp}
[2]> (defun close_big_even (x) (if (evenp x) x (+ x 1)))
; или (defun close_big_even (x) (+ x (mod x 2)))
CLOSE_BIG_EVEN
[3]> (close_big_even 5)
6
[4]> (close_big_even 4)
4
\end{minted}
\end{code}

\section{Задание №2}
Написать функцию, которая принимает число и возвращает число того же знака, но с модулем на 1 больше модуля аргумента.

\begin{code}
\caption{Задание №2}
\label{code:bf2}
\begin{minted}{lisp}
[6]> (defun bigger_abs (x) (if (< x 0) (- x 1) (+ x 1)))
BIGGER_ABS
[7]> (bigger_abs 3)
4
[8]> (bigger_abs -3)
-4
\end{minted}
\end{code}

\section{Задание №3}
Написать функцию, которая принимает два числа и возвращает список из этих чисел, расположенный по возрастанию.

\begin{code}
\caption{Задание №3}
\label{code:bf3}
\begin{minted}{lisp}
[12]> (defun asc_list (x y) (if (< x y) (list x y) (list y x)))
ASC_LIST
[13]> (asc_list 2 3)
(2 3)
[14]> (asc_list 3 2)
(2 3)
[15]> (asc_list 0 -1)
(-1 0)
[16]> (asc_list 0 0)
(0 0)
\end{minted}
\end{code}

\section{Задание №4}
Написать функцию, которая принимает три числа и возвращает $T$ только тогда, когда первое число расположено между вторым и третьим.

\begin{code}
\caption{Задание №4}
\label{code:bf4}
\begin{minted}{lisp}
[8]> (defun first_between (x y z) (if (or (and (< z x) (< x y))
(and (< y x) (< x z))) t nil))
FIRST_BETWEEN
[9]> (first_between 1 2 3)
NIL
[10]> (first_between 2 1 3)
T
[11]> (first_between 2 3 1)
T
[12]> (first_between 0 -1 1)
T
[13]> (first_between 0 0 0)
NIL
\end{minted}
\end{code}

\section{Задание №5}
Каков результат вычисления следующих выражений?

\begin{enumerate}
	\item $(and \quad 'fee \quad 'fie \quad 'foe)$;
	\item $(or \quad nil \quad 'fie \quad 'foe)$;
	\item $(and \quad (equal \quad 'abc \quad 'abc) \quad 'yes)$;
	\item $(or \quad 'fee \quad 'fie \quad 'foe)$;
	\item $(and \quad nil \quad 'fie \quad 'foe)$;
	\item $(or \quad (equal \quad 'abc \quad 'abc) \quad 'yes)$.
\end{enumerate}

\begin{code}
\caption{Задание №5}
\label{code:bf5}
\begin{minted}{lisp}
[7]> (and 'fee 'fie 'foe)
FOE ; если все выражения не-nil, возвращает результат последнего

[8]> (or nil 'fie 'foe)
FIE ; вернёт первый не-nil результат

[9]> (and (equal 'abc 'abc) 'yes)
YES ; если все выражения не-nil, возвращает результат последнего

[10]> (or 'fee 'fie 'foe)
FEE ; возвращает первый не-nil результат

[11]> (and nil 'fie 'foe)
NIL ; если встречает nil результат, сразу его возвращает

[12]> (or (equal 'abc 'abc) 'yes)
T ; возвращает первый не-nil результат
\end{minted}
\end{code}

\section{Задание №6}
Написать предикат, который принимает два числа-аргумента и возвращает $T$, если первое число не меньше второго.

\begin{code}
\caption{Задание №6}
\label{code:bf4}
\begin{minted}{lisp}
[15]> (defun second_less (x y) (>= x y))
SECOND_LESS
[16]> (second_less 1 1)
T
[17]> (second_less 1 2)
NIL
[18]> (second_less 3 2)
T
\end{minted}
\end{code}

\section{Задание №7}
Какой из следующих двух вариантов предиката ошибочен и почему? 

\begin{enumerate}
	\item $(defun \quad pred1 \quad (x) \quad (and \quad (numberp \quad x) \quad (plusp \quad x)))$;
	\item $(defun \quad pred2 \quad (x) \quad (and \quad (plusp \quad x) \quad (numberp \quad x)))$.
\end{enumerate}

\begin{code}
\caption{Задание №7}
\label{code:bf4}
\begin{minted}{lisp}
[1]> (defun pred1 (x) (and (numberp x) (plusp x)))
PRED1
[2]> (pred1 2)
T
[3]> (pred1 -2)
NIL
[4]> (pred1 't)
NIL

[5]> (defun pred2 (x) (and (plusp x) (numberp x)))
PRED2
[6]> (pred2 2)
T
[7]> (pred2 -2)
NIL
[8]> (pred2 't)
*** - PLUSP: T is not a real number
\end{minted}
\end{code}

Второй предикат ошибочен. Это связано с тем, что в $and$ переданные выражения обрабатываются в порядке передачи, то есть проверка $plusp$ на положительность аргумента будет выполнена раньше, чем проверка на то, что аргумент является числом. Если аргумент~---~не число, то такой порядок проверок приведёт к ошибке.

\section{Задание №8}
Решить задачу 4, используя для ее решения конструкции: только $if$, только $cond$, только $and/or$.

Задание №4: написать функцию, которая принимает три числа и возвращает $Т$ только тогда, когда первое число расположено между вторым и третьим.

\begin{code}
\caption{Задание №8}
\label{code:bf4}
\begin{minted}{lisp}
; только if
[2]> (defun first_between (x y z) (if (< y x) (< x z) 
(if (< z x) (< x y) nil)))
FIRST_BETWEEN
[3]> (first_between 1 2 3)
NIL
[4]> (first_between 2 1 3)
T
[5]> (first_between 2 3 1)
T
[6]> (first_between 0 -1 1)
T
[7]> (first_between 0 0 0)
NIL
\end{minted}
\end{code}

\newpage

\begin{code}
\caption{Задание №8}
\label{code:bf4}
\begin{minted}{lisp}
; только cond
[14]> (defun first_between (x y z) (
cond ((< y x) (< x z))
((< z x) (< x y))
))
FIRST_BETWEEN
[15]> (first_between 1 2 3)
NIL
[16]> (first_between 2 1 3)
T
[17]> (first_between 2 3 1)
T
[18]> (first_between 0 -1 1)
T
[19]> (first_between 0 0 0)
NIL
\end{minted}
\end{code}

\begin{code}
\caption{Задание №8}
\label{code:bf4}
\begin{minted}{lisp}
; только and/or
[20]> (defun first_between (x y z) (or (and (< z x) (< x y)) 
(and (< y x) (< x z))))
FIRST_BETWEEN
[21]> (first_between 1 2 3)
NIL
[22]> (first_between 2 1 3)
T
[23]> (first_between 2 3 1)
T
[24]> (first_between 0 -1 1)
T
[25]> (first_between 0 0 0)
NIL
\end{minted}
\end{code}

\newpage

\section{Задание №9}
Переписать функцию $how-alike$, приведенную в лекции и использующую $cond$, используя только конструкции $if$, $and/or$.

Вариант с $cond$ из лекций:

\begin{code}
\caption{Задание №9}
\label{code:bf4}
\begin{minted}{lisp}
[8]> (defun how_alike (x y)
(cond
((or (= x y) (equal x y)) 'the_same)
((and (oddp x) (oddp y)) 'both_odd)
((and (evenp x) (evenp y)) 'both_even)
(t 'difference)
))
HOW_ALIKE
[9]> (how_alike 3 0)
DIFFERENCE
[10]> (how_alike 2 2)
THE_SAME
[11]> (how_alike 2 2.0)
THE_SAME
[12]> (how_alike 2 4)
BOTH_EVEN
[13]> (how_alike 3 3)
THE_SAME
[14]> (how_alike 3 1)
BOTH_ODD
\end{minted}
\end{code}

\newpage

Вариант с $if$, $and/or$:

\begin{code}
\caption{Задание №9}
\label{code:bf4}
\begin{minted}{lisp}
[15]> (defun how_alike (x y)
(if (or (= x y) (equal x y)) 'the_same
(if (and (oddp x) (oddp y)) 'both_odd
(if (and (evenp x) (evenp y)) 'both_even
'difference))
))
HOW_ALIKE
[16]> (how_alike 3 0)
DIFFERENCE
[17]> (how_alike 2 2)
THE_SAME
[18]> (how_alike 2 2.0)
THE_SAME
[19]> (how_alike 2 4)
BOTH_EVEN
[20]> (how_alike 3 3)
THE_SAME
[21]> (how_alike 3 1)
BOTH_ODD
\end{minted}
\end{code}
