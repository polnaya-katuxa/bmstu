\chapter{Коды ядра}

\section*{Прерывания}

\begin{lstlisting}
	struct cdev {
    ...
    struct module *owner;
    const struct file_operations *ops;
    struct list_head list;
    dev_t dev;
    ...
} __randomize_layout;
\end{lstlisting}

\begin{lstlisting}
	struct block_device {
    dev_t            bd_dev; 
    ...
    struct inode *        bd_inode; 
    struct super_block *    bd_super;
    ...
    struct gendisk *    bd_disk; // (определяет устройство (специализированный интерфейс), ссылается на struct block_device_operations)
    ...
} __randomize_layout;
\end{lstlisting}

\begin{lstlisting}
	struct tasklet_struct
	{
		struct tasklet_struct *next; // указатель на следующий тасклет в односвязном  списке
		unsigned long state; // состояние тасклета
		atomic_t count; // счетчик ссылок 
		bool use_callback; // флаг
		union {
			void (*func)(unsigned long data); // функция-обработчик тасклета
			void (*callback)(struct tasklet_struct *t);
		};
		unsigned long data; // аргумент функции-обработчика тасклет
	};
\end{lstlisting}

\begin{lstlisting}
	/*
	* Очередь отложенных действий, связанная с процессором:
	*/
	struct cpu_workqueue_struct 
	{
	   spinlock_t lock; /* Очередь для защиты данной структуры */
	   long remove_sequence; /* последний добавленный элемент
	  (следующий для запуска ) */
	   long insert_sequence; /* следующий элемент для добавления */
	   struct list_head worklist; /* список действий на каждое cpu */
	   wait_queue_head_t more_work;
	   wait_queue_head_t work_done;
	  struct workqueue_struct *wq; /* соответствующая структура
									workqueue_struct */
		task_t *thread; /* соответствующий поток (функция) */
		int run_depth; /* глубина рекурсии функции run_workqueue() */
	};
\end{lstlisting}

\begin{lstlisting}
	struct workqueue_struct {
	struct list_head	pwqs;		/* WR: all pwqs of this wq */
	struct list_head	list;		/* PR: list of all workqueues */

	struct mutex		mutex;		/* protects this wq */
	int			work_color;	/* WQ: current work color */
	int			flush_color;	/* WQ: current flush color */
	atomic_t		nr_pwqs_to_flush; /* flush in progress */
	struct wq_flusher	*first_flusher;	/* WQ: first flusher */
	struct list_head	flusher_queue;	/* WQ: flush waiters */
	struct list_head	flusher_overflow; /* WQ: flush overflow list */

	struct list_head	maydays;	/* MD: pwqs requesting rescue */
	struct worker		*rescuer;	/* MD: rescue worker */

	int			nr_drainers;	/* WQ: drain in progress */
	int			saved_max_active; /* WQ: saved pwq max_active */

	struct workqueue_attrs	*unbound_attrs;	/* PW: only for unbound wqs */
	struct pool_workqueue	*dfl_pwq;	/* PW: only for unbound wqs */

#ifdef CONFIG_SYSFS
	struct wq_device	*wq_dev;	/* I: for sysfs interface */
#endif
#ifdef CONFIG_LOCKDEP
	char			*lock_name;
	struct lock_class_key	key;
	struct lockdep_map	lockdep_map;
#endif
	char			name[WQ_NAME_LEN]; /* I: workqueue name */

	/*
	 * Destruction of workqueue_struct is RCU protected to allow walking
	 * the workqueues list without grabbing wq_pool_mutex.
	 * This is used to dump all workqueues from sysrq.
	 */
	struct rcu_head		rcu;

	/* hot fields used during command issue, aligned to cacheline */
	unsigned int		flags ____cacheline_aligned; /* WQ: WQ_* flags */
	struct pool_workqueue __percpu *cpu_pwqs; /* I: per-cpu pwqs */
	struct pool_workqueue __rcu *numa_pwq_tbl[]; /* PWR: unbound pwqs indexed by node */
};
\end{lstlisting}

\begin{lstlisting}
	typedef void (*work_func_t) (struct work_struct *work);
		struct work_struct {
			atomic_long_t data;
			struct list_head entry;
			work_func_t func; // обработчик работы
	#ifdef CONFIG_LOCKDEP
			struct lockdep_map lockdep_map;
	#endif
	};
	\end{lstlisting}

	\begin{lstlisting}
		extern void raise_softirq(unsigned int nr)
	   {
		  unsigned long flags;
		  local_irq_save(flags);\\ сохраняет состояние флага IF
		  \\(разрешает/запрещает прер-я на процессоре)
		  raise_softirq_irqoff(nr);
		  local_irq_restore(flags);
	   }
	  \end{lstlisting}

\section*{Сокеты}

\begin{lstlisting}
	#include <net/socket.c>
	asmlinkage long sys_socketcall(int call, unsigned long *args)
	// ее текст = switch, переключающий ядро на разные функции, связанные с сокетом
	{
	  int err;
	  if copy_from_user(a, args, nargs[call])
		return -EFAULT;
	  a0 = a[0];
	  a1 = a[1];
	  switch(call)
	  {
		  case SYS_SOCKET: err= sys_socket(a0, a1, a[2]); break;
		  case SYS_BIND: err= sys_bind(a0, (struct sockaddr*)a1, a[2]); break;
		  case SYS_CONNECT: err= sys_connect(...); break;
		  ...
		  default: err = -EINVAL; break;
	  }
	  return err;
	}
	\end{lstlisting}

	\begin{lstlisting}
		asmlinage long sys_socket(int family, int type, int protocol)
		{
		  int retval;
		  struct socket *sock;
		  ...
		  retval = sock_create(famaly, type, protocol, &sock);
		  ...
		  return retval;
		}
		\end{lstlisting}
		
\begin{lstlisting}
			struct socket // нет в 6 версии ядра
			{
			  socket_state state;
			  short type;
			  unsigned long flags;
			  const struct proto_ops *ops;
			  struct fasync_strcut *fasync_list;
			  struct file *file;
			  struct sock *sk;
			  wait_queue_head_t wait;
			}
\end{lstlisting}

\begin{lstlisting}
	struct sockaddr
	{
	  sa_family_t sa_family;
	  char sa_data[14]; 
	}
\end{lstlisting}

\begin{lstlisting}
struct soackaddr_in
{
  sa_family_t sa_family;
  unsigned short int sin_port;
  struct in_addr sin_addr;
  unsigned char sin_zero[sizeof(struct sockaddr) - sizeof(sa_family_t) - sizeof(uint16_t) - sizeof(struct in_addr)];
};
\end{lstlisting}			

\section*{Proc}

\begin{lstlisting}[language=C, label=lst:1, caption= Структура proc\_dir\_entry]
<linux/proc_fs.h>
struct proc_dir_entry {
 atomic_t in_use;
 refcount_t refcnt;
 struct list_head pde_openers; 
 spinlock_t pde_unload_lock; // Собственное средство взаимоисключения
 ...
 const struct inode_operations *proc_iops; // Операции определенные на inode фс proc
 union {
  const struct proc_ops *proc_ops;
  const struct file_operations *proc_dir_ops;
 };
 const struct dentry_operations *proc_dops; // Используетя для регистрации своих операций над файлом в proc
 union {
  const struct seq_operations *seq_ops;
  int (*single_show)(struct seq_file *, void *);
 };
 proc_write_t write;
 void *data;
 unsigned int state_size;
 unsigned int low_ino;
 nlink_t nlink;
        ...
 loff_t size;
 struct proc_dir_entry *parent;
 ...
 char *name;
 u8 flags;
        ...
} 
\end{lstlisting}

\begin{lstlisting}[language=C, label=lst:1, caption= Структура proc\_ops]
struct proc_ops {
 unsigned int proc_flags;
 int (*proc_open)(struct inode *, struct file *);
        // в таблице открытых файлов об одном файле находится столько записей, сколько раз он был открыт
 ssize_t (*proc_read)(struct file *, char __user *, size_t, loff_t *);
 ...
 ssize_t (*proc_write)(struct file *, const char __user *, size_t, loff_t *);
 loff_t (*proc_lseek)(struct file *, loff_t, int);
 int (*proc_release)(struct inode *, struct file *);
 ...
 long (*proc_ioctl)(struct file *, unsigned int, unsigned long);
}
\end{lstlisting}

\begin{lstlisting}[language=C, label=lst:1, caption= Функция proc\_create\_data]
extern struct proc_dir_entry *proc_create_data(const char *, umode_t,
  struct proc_dir_entry *,
  const struct proc_ops *, void *);
\end{lstlisting}

\begin{lstlisting}[language=C, label=lst:1, caption= Функция proc\_create]
static inline struct proc_dir_entry *proc_create(const char *name, umode_t mode,
       struct proc_dir_entry *parent,
       const struct proc_ops *proc_ops)
{
 return proc_create_data(name, mode, parent, proc_ops, NULL);
}
\end{lstlisting}

\begin{lstlisting}[language=C, label=lst:1, caption= Хз чо это]
#include <linux/types.h>
#include <linux/fs.h>

extern struct proc_dir_entry *proc_symlink(const char *, struct proc_dir_entry *, const char *);
extern struct proc_dir_entry *proc_mkdir(const char *, struct proc_dir_entry *);
struct proc_dir_entry *create_proc_read_entry( const char *name, mode_t mode, struct proc_dir_entry *base, read_proc_t *read_proc, void *data );
\end{lstlisting}

\begin{lstlisting}
	struct file_operations {
	struct module *owner;
	loff_t (*llseek) (struct file *, loff_t, int);
	ssize_t (*read) (struct file *, char __user *, size_t, loff_t *);
	ssize_t (*write) (struct file *, const char __user *, size_t, loff_t *);
	...
	int (*open) (struct inode *, struct file *);
	...
	int (*release) (struct inode *, struct file *);
	...
} __randomize_layout;
\end{lstlisting}

\begin{lstlisting}
unsigned long __copy_to_user(void __user *to, const void *from, unsigned long n);
unsigned long __copy_from_user(void *to, const void __user *from, unsigned long n);
\end{lstlisting}

\section*{Super\_block}

\begin{lstlisting}
struct super_block {
        struct list_head        s_list;            
        dev_t                   s_dev;            // устройство, на котором находится ФС
        unsigned long           s_blocksize;       // размер блока в байтах
        unsigned char           s_dirt;           // флаг изменения супрблока
        struct file_system_type s_type;            // в ядре есть структура описывающая тип ФС
        struct super_operations s_op;             // операции на суперблоке
        struct block_device *b_dev // описывает устройство, на котором находится файловая система (соответствует драйверу блочного устройства)
        unsigned long            s_magic;         // магический номер смонтированной ФС
        struct dentry            *s_root;      //  точка монтирования ФС
        ...
        int                      s_count;       // число ссылок
        struct list_head         s_dirty;         // список измененных inode'ов
        char                     s_id[32];      // имя?
};
\end{lstlisting}

\begin{lstlisting}
<linux/fs.h>

struct super_operations
{
  struct inode *(alloc_inode)(struct superblock *sb);
void (*destroy_inode) (struct inode *);
...
void (*dirty_inode)(struct inode *, int flags);
int (*write_inode)(struct inode*, struct wirteback_cintrol *wbc);
int (*drop_inode)(struct inode *);
...
void (*put_super)(struct super_block *);
}
\end{lstlisting}

\begin{lstlisting}
static struct super_block *alloc_super (
  struct file_system_type *type,
  int flags,
  struct user_namespace *user_ns
)
{
  struct superblock *s = kzalloc(sizeof(struct super_block), GFP_USER);
  static const struct super_operations default_ops;
  if (!s) return NULL;
  INIT_LIST_HEAD(&s->s_mounts);
  ...
  INIT_LIST_HEAD(&s->s_inodes);
  ...
}
\end{lstlisting}

\section*{Dentry}

\begin{lstlisting}
struct dentry_operations
{
  int (*drevalidate)(struct dentry *, unsigned int);
  ...
  int (*d_hash)(const struct dentry *, unsigned int);
  int (* d_compare)(const struct dentry *, unsigned int, const char *, const struct
  qstr *);
  int (* d_delete)(const struct dentry *);
  int (* d_init)(const struct dentry *);
  int (* d_release) (struct dentry *);
  void (* d_input)(struct dentry *, struct inode *);
  char *(* d_name)(struct dentry *, char *, int);
  ..
}
\end{lstlisting}

\section*{Inode}

\begin{lstlisting}
struct inode {
  struct list_head i_hash;
 struct list_head i_list;
 struct list_head i_dentry;
 ...
 unsigned long i_ino;
 atomic_t i_count;
 kdev_t i_rdev;
 umode_t i_mode;
 ...
 loff_t i_size;
 ...
 // информация о времени модификации и доступа к inode
 ...
 // 6 полей, связ с блоками(только для ядра)
 ...
 unsigned int i_blkbits;// битовая карта блока
 unsigned long i_blksize;// размер блоков
 unsigned long i_blocks;// кол-во блоков
 ...
 const struct inode_operations  *i_op; //перечень функций определенных для работы с inode и c открытыми файлами
 const struct file_operations  *i_fop; 
 struct super_block  *i_sb; 
 ...
 struct list_head i_devices;
 struct pipe_inode_info *i_pipe;
 struct block_device *i_bdev;
 struct char_device *i_cdev;
 ...
 unsigned long i_state;
 unsigned int i_flags;
 ...
 union //типы фс
 {
 struct minix_inode_info minix_i;
 struct ext2_inode_info ext2_i;
 ....
 struct ntfs_inode_info ntfs_i;
 struct msdos_inode_info msdos_i;
 ...
 struct nfs_inode_info nfs_i;// сетевая фс
 struct ufs_inode_info ufs_i;
 ...
 struct proc_inode_info proc_i;
 struct socket socket_i;
 ...
 
 }
};
\end{lstlisting}


\begin{lstlisting}
    struct inode_operations {
  struct dentry * (*lookup) (struct inode *,struct dentry *, unsigned int);
  
  int (*create) (struct inode *,struct dentry *,
           umode_t, bool);

  int (*mkdir) (struct inode *,struct dentry *,
          umode_t);
  
  int (*rename) ( struct inode *, struct dentry *,
      struct inode *, struct dentry *, unsigned int);
  ...
} ;
\end{lstlisting}

\section*{file}

\begin{lstlisting}
    struct file {
  struct path    f_path;
  struct inode    *f_inode;  /* cached value */
  const struct file_operations  *f_op;
        ...
  atomic_long_t    f_count;// кол-во жестких ссылок
  unsigned int     f_flags;
  fmode_t      f_mode;
  struct mutex    f_pos_lock;
  loff_t      f_pos;
  ...
  struct address_space  *f_mapping;
  ...
};
\end{lstlisting}

\begin{lstlisting}
	struct file_operations {
	struct module *owner;
	loff_t (*llseek) (struct file *, loff_t, int);
	ssize_t (*read) (struct file *, char __user *, size_t, loff_t *);
	ssize_t (*write) (struct file *, const char __user *, size_t, loff_t *);
	...
	int (*open) (struct inode *, struct file *);
	...
	int (*release) (struct inode *, struct file *);
	...
} __randomize_layout;
\end{lstlisting}

\section*{mount}

\begin{lstlisting}
typedef int (*fill_super_t)(struct super_block *, void *, int);
struct dentry *mount_bdev(struct file_system_type *fs_type, int flags, const char *dev_name, void *data, fill_super_t fill_super);
struct dentry *mount_nodev(struct file_system_type *fs_type, int flags, void *data, fill_super_t fill_super);
struct dentry *mount_single(struct file_system_type *fs_type, int flags, void *data, fill_super_t fill_super);
\end{lstlisting}

\section*{Simple \& generic}

\begin{lstlisting}
	int generic_delete_inode(struct inode *inode)
	{
		return 1;
	}
\end{lstlisting}

\begin{lstlisting}
	int simple_statfs(struct dentry *dentry, struct kstatfs *buf)
	{
		buf->f_type = dentry->d_sb->s_magic;
		buf->f_bsize = PAGE_CACHE_SIZE;
		buf->f_namelen = NAME_MAX;
		return 0;
	}
\end{lstlisting}

\section*{file\_system\_type}

\begin{lstlisting}
	struct file_system_type {
		const char *name;
		int fs_flags;
		#define FS_REQUIRES_DEV    1 
		...
		#define FS_USERNS_MOUNT    8  /* Can be mounted by userns root */
		...
		struct dentry *(*mount) (struct file_system_type *, int,
		const char *, void *);
		void (*kill_sb) (struct super_block *);
		struct file_system_type * next;
		struct hlist_head fs_supers;
		
		struct lock_class_key s_lock_key;
		struct lock_class_key s_umount_key;
		struct lock_class_key s_vfs_rename_key;
		struct module *owner;
		...
	};
\end{lstlisting}

\section*{Seqfile}
\begin{lstlisting}[language=C, label=lst:1, caption= Структуры]
struct seq_operations;

struct seq_file {
 char *buf;
 size_t size;
 size_t from;
 size_t count;
 ...
 loff_t index;
 loff_t read_pos;
 struct mutex lock;
 const struct seq_operations *op;
 int poll_event;
 const struct file *file;
 void *private;
};

struct seq_operations {
 void * (*start) (struct seq_file *m, loff_t *pos);
 void (*stop) (struct seq_file *m, void *v);
 void * (*next) (struct seq_file *m, void *v, loff_t *pos);
 int (*show) (struct seq_file *m, void *v);
};
\end{lstlisting}
