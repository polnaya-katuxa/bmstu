\setcounter{page}{3}

\chapter{Об игре}

\section{Название}
Ассоциации.

\section{Цель игры}
Развитие навыков общения, группового взаимодействия, адаптации в новом круге лиц, круге лиц разного возраста. Отработка навыков объяснения, грамотного выражения мыслей и донесения их до собеседника. Развитие внимания, памяти, ассоциативного мышления и умения слушать собеседника.

\section{Средства}
Для визуализации результатов работы участников понадобится игровое поле, содержащее любое количество полей, в зависимости от времени, выделенного на игру. Также, будут необходимы фишки, которые игроки будут перемещать по клеткам поля в случае успеха, и пронумерованные карточки для голосования. Количество карточек и фишек должно совпадать с количеством участников. Также, нужен набор карточек с изображениями разного уровня абстракции, позволяющих формировать на их основе ассоциативный ряд. Количество таких карт может быть до 100 штук.

\section{Количество и возраст игроков}
Количество игроков для комфортной игры может составлять от 4 до 7 человек. Участвовать могут игроки любого возраста - чем больше он отличается, тем разнообразнее будут ассоциации и тем интереснее играть.

\section{Время игры}
Зависит от количества игроков и карточек с изображениями, в среднем составляет около 30-40 минут.

\chapter{Ход игры}
Фишки всех игроков выставляются на начало игрового поля. Колода с карточками-иллюстрациями перемешивается и каждому игроку выдается на руки по 6 карточек. 

Каждый ход один из игроков, по очереди, становится ведущим. Ведущий загадывает ассоциацию на одну из своих карточек, произносит её вслух и выкладывает на стол загаданную карточку так, чтобы другие игроки не видели изображение. Остальные игроки ищут среди своих карточек один рисунок, который, по их мнению, наилучшим образом подходит под загаданную фразу, и кладет её на стол рубашкой вверх.

Ведущий собирает карточки, перемешивает их и выкладывает на стол в линию в случайном порядке, но уже так, чтобы все игроки видели изображения. Карты условно нумеруются по порядку слева направо.

Основная задача игроков — угадать, какую именно из выложенных на столе карточек загадал ведущий, и проголосовать за нее. Каждый игрок выбирает одну карточку для голосования с номером карты, за которую хочет проголосовать, кроме номера собственной карты. Ведущий не голосует. Когда все приняли решение и проголосовали, карточки для голосования переворачиваются и происходит подсчет очков.

\begin{enumerate}
	\item Если карточку ведущего угадали все игроки, то он идет на 3 хода назад, а остальные стоят на месте.
	\item Если карточку ведущего никто не угадал, то ведущий идет на 2 хода назад. Плюс очки получают игроки, за чьи карточки проголосовали (игрок получает столько очков, сколько голосов за его карточку).
	\item В любом другом случае по 2 очка получают все игроки, правильно угадавшие карточку. Ведущий получает по 1 очку за каждого угадавшего его карточку игрока. Все игроки получают по одному очку за каждого другого игрока, который проголосовал за их карточку.
\end{enumerate}

Игроки передвигают свои фишки на игровом поле на количество шагов, соответствующее количеству выигранных очков. Каждый игрок берет по одной карте из колоды на новом ходу. Ведущим становится следующий игрок по порядку.

Игра заканчивается, когда заканчиваются карты на руках у игроков. Победителем оказывается тот, кто заработал больше всего очков и продвинулся дальше всех.

\chapter{Обсуждение}
Во время игры участники знакомятся, узнают друг друга ближе, стараются как можно лучше понять друг друга, обсуждают изображения. Могут возникать споры, если ассоциации игроков оказались совсем разными, или, наоборот, моменты сближения, когда мысли сошлись. Зачастую попадаются комичные изображения на карточках, что также развивает навыки импровизации и ситуационного юмора игроков.

По окончании игры игроки могут также провести обсуждение касаемо возникавших в ходе игры ассоциаций, на их основе сформировать мнение о личностях и интересах друг друга, качествах и навыках общения.

\newpage