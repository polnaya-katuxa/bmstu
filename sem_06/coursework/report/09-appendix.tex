\begin{appendices}
\label{appendix:graph}
	\chapter{SQL-скрипты}
	
	В листингах \ref{createPG}~--~\ref{filterPG} приведены листинги SQL-скриптов для создания таблиц, функций, ролей и реализованные запросы для СУБД Postgres и ClickHouse.
	
	\begin{lstlisting}[label=createPG,caption=SQL-скрипт создания таблиц в СУБД Postgres]
create extension "uuid-ossp";

create table users (
    uuid uuid default uuid_generate_v4() primary key,
    login text unique not null,
    password text not null,
    is_admin bool default false
);

create type gender as enum (
    'female',
    'male'
    );

create table clients (
    uuid uuid default uuid_generate_v4() primary key,
    name text not null,
    surname text not null,
    patronymic text,
    gender gender not null,
    birth_date timestamp not null check (birth_date < current_timestamp::timestamp and birth_date > '01.01.1900 00:00:00'::timestamp),
    registration_date timestamp not null check (registration_date < current_timestamp::timestamp and registration_date > clients.birth_date),
    email text unique not null check ( email like '%@%.%' ),
    data json
);

create table event_types (
    uuid uuid default uuid_generate_v4() primary key,
    name text not null,
    alias text unique not null check ( length(alias) > 5 )
);

create table schedules (
    uuid uuid default uuid_generate_v4() primary key,
    next_time timestamp not null,
    finished boolean default false not null,
    periodic boolean default false not null,
    span text not null
);

create table ads (
    uuid uuid default uuid_generate_v4() primary key,
    content text not null,
    filters json,
    user_id uuid references users (uuid) on delete set null,
    schedule_id uuid unique references schedules (uuid) on delete cascade,
    creation_time timestamp not null check (creation_time < current_timestamp::timestamp)
);

create table events (
     uuid uuid default uuid_generate_v4() primary key,
     time timestamp not null,
     client_id uuid references clients (uuid) on delete set null,
     alias text references event_types (alias) not null check (length(alias) > 5)
);
	\end{lstlisting}
	
	\begin{lstlisting}[label=createRolePG,caption=SQL-скрипт создания ролей в системе управления базами данных Postgres]
create role admin login password 'password1';
create role targetologist login password 'password2';
create role client login password 'password3';

grant insert on clients to client;
grant insert on events to client;

grant insert, select on users to targetologist;
grant insert, select on ads to targetologist;
grant insert, select on event_types to targetologist;
grant select on events to targetologist;
grant insert, select, update on schedules to targetologist;
grant select on clients to targetologist;

grant insert, select, delete, update on users to admin;
grant select, delete on clients to admin;
	\end{lstlisting}
	
	\begin{lstlisting}[label=createCH,caption=SQL-скрипт создания таблицы events в СУБД ClickHouse]
CREATE TABLE coursework.events_queue (
    id UUID,
    client_id UUID,
    alias String,
    time DateTime('UTC')
) ENGINE = Kafka('127.0.0.1:9092', 'blurby-events', 'clickhouse-blurby-events', 'JSONEachRow');
		
CREATE TABLE coursework.events
(
    id UUID,
    client_id UUID,
    alias String,
    time DateTime('UTC')
) ENGINE = MergeTree ORDER BY (time);
		
CREATE MATERIALIZED VIEW coursework.events_mv TO coursework.events AS
		SELECT *
		FROM coursework.events_queue;
	\end{lstlisting}
	
	\begin{lstlisting}[label=createRoleCH,caption=SQL-скрипт создания ролей в СУБД ClickHouse]
CREATE USER targetologist IDENTIFIED WITH sha256_password BY 'password1';

CREATE ROLE targetologist_role;
GRANT SELECT ON coursework.events TO targetologist_role;
GRANT SELECT ON coursework.ad_send_times TO targetologist_role;

GRANT targetologist_role TO targetologist;
	\end{lstlisting}
	
	\begin{lstlisting}[label=createFuncPG,caption=SQL-скрипт табличной функции в СУБД Postgres]
create or replace function get_ads_by_span(finish timestamp) returns table(
   uuid uuid,
   content text,
   filters json,
   user_id uuid,
   schedule_id uuid,
   creation_time timestamp,
   next_time timestamp,
   span text,
   finished boolean,
   periodic boolean
)
as $$
begin
    return query (
        select a.uuid as uuid, a.content, a.filters, a.user_id, a.schedule_id, a.creation_time, s.next_time, s.span, s.finished, s.periodic
        from ads a join schedules s on s.uuid = a.schedule_id
        where s.next_time <= finish and s.finished = false
    );
end;
$$
    language plpgsql;
	\end{lstlisting}
	
	\begin{lstlisting}[label=filterCH,caption=SQL-скрипт запроса на фильтрацию пользователей в СУБД ClickHouse]
select distinct client_id
				from (
					 select
						 client_id,
						 alias,
						 count(*) as num
					 from coursework.events
					 where time between now() - interval 5 hour and now()
					 group by client_id, alias
					 order by num desc
					 limit 5 by client_id
				) where alias = 'alias'
	\end{lstlisting}
	
	\begin{lstlisting}[label=filterPG,caption=SQL-скрипт запроса на фильтрацию пользователей в СУБД Postgres]
select distinct cast sub1.client_id
			from (
			select distinct sub.client_id, sub.alias, sub.c,
			       row_number() over (partition by sub.client_id order by sub.c desc) as r
				from (
				select client_id, alias, count(*) as c from events
				    where time between current_timestamp - '5 hours'::interval and current_timestamp
					group by events.client_id, alias) as sub
				) as sub1
					where alias = 'alias' and r <= 5
	\end{lstlisting}
	
\end{appendices}