\begin{appendices}
\label{appendix:graph}
	\chapter{Тестирование}
	
	В листингах \ref{testMod}~--~\ref{testCon} приведены листинги функций необходимых для тестирования разработанного ПО.
	
	\begin{lstlisting}[label=testMod,caption=Модульный тест для функции входа в систему]
func TestProfile_Enter(t *testing.T) {
	mc := minimock.NewController(t)
	password := "password"
	hash, _ := bcrypt.GenerateFromPassword([]byte(password), bcrypt.DefaultCost)
	user := &models.User{
		ID:       uuid.New(),
		Login:    "uehfkjsyg",
		Password: string(hash),
		IsAdmin:  false,
	}
	type fields struct {
		userRepo        interfaces.UserRepository
		dailyBonus      int
		tokenExpiration time.Duration
		secretKey       string
	}
	type args struct {
		ctx      context.Context
		login    string
		password string
	}
	tests := []struct {
		name    string
		fields  fields
		args    args
		want    string
		wantErr bool
	}{
		{
			name: "successful enter",
			fields: fields{
				userRepo:        mocks.NewUserRepositoryMock(mc).GetByLoginMock. Return(user, nil),
				dailyBonus:      5,
				tokenExpiration: time.Hour,
				secretKey:       "secretkey",
			},
			args: args{
				ctx:      context.Background(),
				login:    "uehfkjsyg",
				password: password,
			},
			want:    "",
			wantErr: false,
		},
	}
	for _, tt := range tests {
		t.Run(tt.name, func(t *testing.T) {
			p := &Profile{
				userRepository:  tt.fields.userRepo,
				tokenExpiration: tt.fields.tokenExpiration,
				secretKey:       tt.fields.secretKey,
			}
			_, err := p.Login(tt.args.ctx, tt.args.login, tt.args.password)
			if (err != nil) != tt.wantErr {
				t.Errorf("Login() error = %v, wantErr %v", err, tt.wantErr)
				return
			}
		})
	}
}
	\end{lstlisting}
	
	\begin{lstlisting}[label=testBD,caption=SQL-скрипт заполнения БД тестовыми данными используемыми в интеграционных тестах]
insert into clients (uuid, name, surname, patronymic, gender, birth_date, registration_date, email, data) values
('a52b8aea-d751-4933-91bb-691132e3b760', '1', '1', '1', 'male', '2006-01-02 15:04:05', '2009-01-02 15:04:05', '1@mail.ru', null),
('3f010aca-5008-4aa5-a1a3-a061a876783f', '2', '2', '2', 'male', '2007-01-02 15:04:05', '2009-01-02 15:04:05', '2@mail.ru', null),
('77dcd288-79b2-4655-9584-cc9b5329665d', '3', '3', '3', 'female', '2008-01-02 15:04:05', '2009-01-02 15:04:05', '3@mail.ru', null);
	\end{lstlisting}
	
	\begin{lstlisting}[label=testInt,caption=Интеграционный тест для функции получения информации о клиенте]
func TestClient_Get(t *testing.T) {
	dbContainer, db, err := containers.SetupTestDatabase()
	if err != nil {
		t.Fatal(err)
	}
	defer func() { _ = dbContainer.Terminate(context.Background()) }()

	cr := postgres.NewCR(db, db, db)

	user := &models.User{
		ID:       uuid.New(),
		Login:    "uehfkjsyg",
		Password: "r3",
		IsAdmin:  true,
	}

	userNA := &models.User{
		ID:       uuid.New(),
		Login:    "uehfkjsyg",
		Password: "r3",
		IsAdmin:  false,
	}

	id, err := uuid.Parse("77dcd288-79b2-4655-9584-cc9b5329665d")
	if err != nil {
		t.Errorf("uuid parse: %v", err)
		return
	}

	idNE, err := uuid.Parse("77dcd288-79b2-4655-9584-cc9b5329665a")
	if err != nil {
		t.Errorf("uuid parse: %v", err)
		return
	}

	\end{lstlisting}
	
	\begin{lstlisting}[label=testInt,caption=Интеграционный тест для функции получения информации о клиенте (продолжение)]
	
	bd, err := time.Parse(time.RFC3339, "2008-01-02T15:04:05Z")
	if err != nil {
		t.Errorf("bd parse: %v", err)
		return
	}

	rd, err := time.Parse(time.RFC3339, "2009-01-02T15:04:05Z")
	if err != nil {
		t.Errorf("bd parse: %v", err)
		return
	}

	client := &models.Client{
		ID:               id,
		Name:             "3",
		Surname:          "3",
		Patronymic:       "3",
		Gender:           "female",
		BirthDate:        bd,
		RegistrationDate: rd,
		Email:            "3@mail.ru",
		Data:             nil,
	}

	type fields struct {
		clientRepository interfaces.ClientRepository
	}
	type args struct {
		ctx context.Context
		id  uuid.UUID
	}
	tests := []struct {
		name    string
		fields  fields
		args    args
		want    *models.Client
		wantErr bool
	}{
		{
			name:   "successful get test",
			fields: fields{cr},
			args: args{
				ctx: mycontext.UserToContext(context.Background(), user),
				id:  id,
			},
			want:    client,
			wantErr: false,
		},
	}
	for _, tt := range tests {
		t.Run(tt.name, func(t *testing.T) {
			c := &Client{
				clientRepository: tt.fields.clientRepository,
			}
			got, err := c.Get(tt.args.ctx, tt.args.id)
			if (err != nil) != tt.wantErr {
				t.Errorf("Get() error = %v, wantErr %v", err, tt.wantErr)
				return
			}
			if !reflect.DeepEqual(got, tt.want) {
				t.Errorf("Get() got = %v, want %v", got, tt.want)
			}
		})
	}
}
	\end{lstlisting}
	
	\begin{lstlisting}[label=testCon,caption=Функция запуска контейнера для интеграционного тестирования]
func SetupTestDatabase() (testcontainers.Container, *gorm.DB, error) {
	containerReq := testcontainers.ContainerRequest{
		Image:        "postgres:latest",
		ExposedPorts: []string{"5432/tcp"},
		WaitingFor:   wait.ForListeningPort("5432/tcp"),
		Env: map[string]string{
			"POSTGRES_DB":       "testdb",
			"POSTGRES_PASSWORD": "postgres",
	\end{lstlisting}
	
	\begin{lstlisting}[label=testCon,caption=Функция запуска контейнера для интеграционного тестирования (продолжение)]
	        "POSTGRES_USER":     "postgres",
		},
	}

	dbContainer, err := testcontainers.GenericContainer(
		context.Background(),
		testcontainers.GenericContainerRequest{
			ContainerRequest: containerReq,
			Started:          true,
		})
	if err != nil {
		return nil, nil, fmt.Errorf("generic container: %w", err)
	}

	host, err := dbContainer.Host(context.Background())
	if err != nil {
		return nil, nil, fmt.Errorf("host: %w", err)
	}

	port, err := dbContainer.MappedPort(context.Background(), "5432")
	if err != nil {
		return nil, nil, fmt.Errorf("port: %w", err)
	}

	dsn := fmt.Sprintf("host=%s port=%d user=postgres password=postgres dbname=testdb sslmode=disable", host, port.Int())
	pureDB, err := gorm.Open(postgres.Open(dsn), &gorm.Config{})
	if err != nil {
		return nil, nil, fmt.Errorf("gorm open: %w", err)
	}

	sqlDB, err := pureDB.DB()
	if err != nil {
		return nil, nil, fmt.Errorf("get db: %w", err)
	}
	\end{lstlisting}
	
	\begin{lstlisting}[label=testCon,caption=Функция запуска контейнера для интеграционного тестирования (продолжение)]
	if err = goose.Up(sqlDB, "../../../deployments/migrations/postgres"); err != nil {
		return nil, nil, fmt.Errorf("up migrations: %w", err)
	}

	text, err := os.ReadFile("../../../internal/containers/data.sql")
	if err != nil {
		return nil, nil, fmt.Errorf("read file: %w", err)
	}

	if err := pureDB.Exec(string(text)).Error; err != nil {
		return nil, nil, fmt.Errorf("exec: %w", err)
	}

	return dbContainer, pureDB, nil
}
	\end{lstlisting}

	
\end{appendices}