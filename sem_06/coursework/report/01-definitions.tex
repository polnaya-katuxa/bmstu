\begin{definitions}
	\definition{База данных}{совокупность данных из определённой предметной области, организованных по определённым правилам, предусматривающим общие принципы описания, хранения и манипулирования данными, независимая от прикладных программ~\cite{bib41}.}
	\definition{Модель данных}{абстрактная модель, определяющая совокупность правил создания и использования данных, описывающая способ представления данных, доступ к ним и связи между ними в той или иной области~\cite{bib13}.}
	\definition{Система управления базами данных}{это совокупность программ и языковых средств, предназначенных для управления данными в базе данных, ведения базы данных и обеспечения взаимодействия её с прикладными программами~\cite{bib41}.}
	\definition{Брокер сообщений}{программное обеспечение, которое позволяет приложениям, системам и службам взаимодействовать друг с другом и обмениваться информацией~\cite{bib43}.}
	\definition{Топик}{логическая категория сообщений~\cite{bib44}.}
	\definition{Партиция}{физическое хранилище сообщений в кластере Kafka~\cite{bib44}.}
	\definition{Движок таблицы}{тип таблицы, определяющий способ хранения и получения данных, поддерживаемые виды запросов и иные параметры работы с данными~\cite{bib45}.}
	\definition{Аутентификация}{это проверка соответствия субъекта и того, за кого он пытается себя выдать с помощью некой уникальной информации~\cite{bib42}.}
\end{definitions}